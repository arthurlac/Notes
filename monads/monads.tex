\documentclass{article}

\usepackage[utf8]{inputenc}
\usepackage[english]{babel}
\usepackage{amsfonts}
\usepackage{amssymb}
\usepackage{amsthm}
\usepackage{amsmath}
\usepackage{parskip}
\usepackage{mathtools}

\theoremstyle{definition}
\newtheorem{definition}{Definition}[section]
\newtheorem{example}{Example}[section]
\newtheorem{theorem}{Thereom}[section]

\usepackage{tikz-cd}
\usetikzlibrary{arrows}
\tikzset{
    commutative diagrams/.cd,
    arrow style=tikz,
    diagrams={>=space}
}
\tikzcdset{
    arrow style=tikz,
    diagrams={>={Straight Barb[scale=0.8]}}
}
\newcommand{\C}{\mathcal{C}}
\newcommand{\D}{\mathcal{D}}
\newcommand{\N}{\mathbb{N}}

\setlength{\parindent}{0pt}

\begin{document}

\section{Categories}

\begin{definition}
    A \textit{category} $\mathcal{C}$ consists of
    \begin{itemize}
        \item A set $Ob\,\mathcal{C}$, elements of which are called \textit{objects} of $\mathcal{C}$.
        \item For each $X, Y \in Ob\,\mathcal{C}$
            a set $\mathcal{C}(X,Y)$ called the \textit{homset} from $X$ to $Y$.
        \item For each $X, Y, Z \in Ob\,\mathcal{C}$ a \textit{composition function} $\circ$
            $$\mathcal{C}(X,Y) \circ \mathcal{C}(Y,Z) \rightarrow \mathcal{C}(X,Z)$$
        \item For all $X \in Ob\,\mathcal{C}$ an element $\iota_X$ of $\mathcal{C}(X,\ X)$,
            or equivalently a function $\iota_X : 1 \rightarrow \mathcal(X,\ X)$.
            Such that
            $$\forall f \in \mathcal{C}(X,Y)\ f \circ \iota_X = \iota_Y \circ f = f$$
        \item Composition is associative.
            $$\forall
              f \in \mathcal{C}(X,Y),
            \ g \in \mathcal{C}(Y,Z),
            \ h \in \mathcal{C}(X,Z)
            \ f \circ (g \circ h) = (f \circ g) \circ h
            $$
    \end{itemize}
\end{definition}

\par
An element $f$ of $\mathcal{C}(X,Y)$ is called an \textit{arrow},
or a \textit{morphism}. The object $X$ is called the \textit{domain} of $f$ and $Y$ is
the \textit{codomain}.\\

\begin{example}
    The foremost example of a category is the category of \textit{small sets}.
    We say small sets due to Russel's paradox, conversely phrased a category is small
    if its objects and arrow constitute sets; otherwise it is large. \textit{Cite BarrsWells?}

    The objects in $Ob\,\mathcal{C}$ are sets,
    a morphism from $X$ to $Y$ is a function $f : X \rightarrow Y$.
    The composition of Set is given by composition of functions,
    and the identity maps are given by the identity functions.
\end{example}

\section{Functors}
\begin{definition}
    A \textit{functor} $U : \mathcal{C} \rightarrow \mathcal{D}$ consists of
    \begin{itemize}
        \item A function $Ob\,U : Ob\,\C \rightarrow Ob\,\mathcal{D}$.
        \item For each $X, Y \in Ob\,\C$ a function
            $U_{X,Y} : \mathcal{C}(X,Y) \rightarrow \mathcal{D}(UX, UY)$
            such that $U$ respects both composition and identity.
            I.e.
            $$Uf \circ Ug \equiv U(f \circ g) \quad Uf \circ U1 \equiv U(f \circ 1)$$
    \end{itemize}
\end{definition}

\begin{definition}
    An \textit{endofunctor} is a functor $U : \C \rightarrow \C$;
    i.e. the domain and codomain of the functor are the same category $\C$.
\end{definition}

\section{Natural transformations}
\begin{definition}
    Given categories $\mathcal{C}$ and $\mathcal{D}$,
    with functors $U, V : \mathcal{C} \rightarrow \mathcal{D}$
    a \textit{natural transformation} $\alpha : U \rightarrow V$
    consists of
    $
    \forall\ X \in Ob\,\mathcal{C}\ \textrm{a map}
    \ \alpha_X : UX \rightarrow VX
    $
    such that $\forall\ f : X \rightarrow Y$ the following commutes
    \begin{center}
        \begin{tikzcd}[sep=large]
            UX \rar{\alpha_X} \dar[swap]{Uf} & VX \dar{Vf} \\
            UY \rar{\alpha_Y}                & VY
        \end{tikzcd}
    \end{center}
    A natural transformation can be considered a morphism of functors.
\end{definition}

\section{Monads}
\begin{definition}
    A \textit{monad} is defined as the triple
    \begin{itemize}
        \item An endofunctor $T : \C \rightarrow \C$
        \item A natural transformation $\eta : 1_{\C} \rightarrow T$
        \item A natural transformation $\mu : T^2 \rightarrow T$
    \end{itemize}
    Such that the following diagrams commute
    \begin{center}
        \begin{tikzcd}[sep=large]
            T \rar{\eta_T} \drar[swap]{1_{\C}} & T^2 \dar{\mu} \rar{\mu} & T \dlar{1_{\C}} \\
                                               & T                       &
        \end{tikzcd}
        \quad
        \begin{tikzcd}[sep=large]
            T^3 \rar{T\mu} \dar[swap]{\mu_T} & T^2 \dar{\mu} \\
            T^2 \rar[swap]{\mu}                    & T
        \end{tikzcd}
    \end{center}
\end{definition}

\section{Programming Monads}
In this section I will explain how the theory corresponds to in practice;
i.e. List Monad, Maybe/Option monad. And how programmers think of monads.
\end{document}
