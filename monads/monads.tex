\documentclass{article}

\usepackage[utf8]{inputenc}
\usepackage[english]{babel}
\usepackage{amsfonts}
\usepackage{amssymb}
\usepackage{amsthm}
\usepackage{amsmath}
\usepackage{parskip}
\usepackage{mathtools}
\usepackage{stackrel}

\theoremstyle{definition}
\newtheorem{definition}{Definition}[section]
\newtheorem{example}{Example}[section]
\newtheorem{theorem}{Thereom}[section]

\usepackage{tikz-cd}
\usetikzlibrary{arrows}
\tikzset{
    commutative diagrams/.cd,
    arrow style=tikz,
    diagrams={>=space}
}
\tikzcdset{
    arrow style=tikz,
    diagrams={>={Straight Barb[scale=0.8]}}
}

\setlength{\parindent}{0pt}

\begin{document}

\section{Categories}

\begin{definition}
    A \textit{category} $C$ consists of
    \begin{itemize}
        \item A set $Ob\,C$, elements of which are called \textit{objects} of $C$.
        \item For each $X, Y \in Ob\,C$
            a set $Hom(X,Y)$ called the \textit{homset} from $X$ to $Y$.
        \item For each $X, Y, Z \in Ob\,C$ a \textit{composition function} $\circ$
            $$Hom(X,Y) \times Hom(Y,Z) \rightarrow Hom(X,Z)$$
        \item For all $X \in Ob\,C$ an element $Id_X$ of $Hom(X,\ X)$
            such that $f \circ Id_X = Id_Y \circ f = f$
        \item Composition is associative.  $f \circ (g \circ h) = (f \circ g) \circ h$
    \end{itemize}
\end{definition}

\par
An element $f$ of $Hom(X,Y)$ is called an \textit{arrow},
or a \textit{morphism}. The object $X$ is called the \textit{domain} of $f$ and $Y$ is
the \textit{codomain}.\\

\begin{example}
    The immediate example of a category is the category of sets.
    The objects in $C$ are (small) sets,
    a morphism from $X$ to $Y$ is a function $f : X \rightarrow Y$.
    The composition of Set is given by composition of functions,
    and the identity maps are given by the identity functions.
\end{example}

\section{Functors}
\begin{definition}
    A \textit{functor} $U : C \rightarrow D$ consists of
    \begin{itemize}
        \item A function $Ob\,U : Ob\,C \rightarrow Ob\,D$.
        \item A function $U : Hom_C(X,Y) \rightarrow Hom_D(UX, UY)$
            such that $U$ respects both composition and identity.
            I.e.
            $$Uf \circ Ug = U(f \circ g) \quad U\,Id_X = Id_{UX}$$
    \end{itemize}
\end{definition}

\begin{definition}
    An \textit{endofunctor} is a functor $U : C \rightarrow C$;
    i.e. the domain and codomain of the functor are the same category $C$.
\end{definition}

\section{Natural transformations}
\begin{definition}
    Given categories $C$ and $D$,
    with functors $U, V : C \rightarrow D$
    a \textit{natural transformation} $\alpha : U \rightarrow V$
    consists of
    $
    \forall\ X \in Ob\,C\ \textrm{a map}
    \ \alpha_X : UX \rightarrow VX
    $
    such that $\forall\ f : X \rightarrow Y$ the following commutes
    \begin{center}
        \begin{tikzcd}[sep=large]
            UX \rar{\alpha_X} \dar[swap]{Uf} & VX \dar{Vf} \\
            UY \rar{\alpha_Y}                & VY
        \end{tikzcd}
    \end{center}
    A natural transformation can be considered a morphism of functors.
\end{definition}

\section{Monads}
\begin{definition}
    A \textit{monad} is defined as the triple
    \begin{itemize}
        \item An endofunctor $T : C \rightarrow C$
        \item A natural transformation $\eta : 1_{C} \rightarrow T$
        \item A natural transformation $\mu : T^2 \rightarrow T$
    \end{itemize}
    Such that the following diagrams commute
    \begin{center}
        \begin{tikzcd}[sep=large]
            T \rar{\eta_T} \drar[swap]{1_{C}} & T^2 \dar{\mu} & \lar[swap]{T\eta} \dlar{1_{C}} T \\
                                               & T            &
        \end{tikzcd}
        \quad
        \begin{tikzcd}[sep=large]
            T^3 \rar{T\mu} \dar[swap]{\mu_T} & T^2 \dar{\mu} \\
            T^2 \rar[swap]{\mu}                    & T
        \end{tikzcd}
    \end{center}
\end{definition}

\section{Kleisli Category}
\begin{definition}
    Given a monad $(T,\eta,\mu)$ over a category $C$,
    the \textit{Kleisli category} $C_T$ consists of
    \begin{itemize}
        \item Objects of $C_T$ are objects from the underlying category $C$.
        \item $Hom_{C_T}(X,Y) = Hom_C (X,TY)$
        \item Identity morphisms in $C_T$ are $\eta$ in $C$
        \item Composition $f \circ g$ is $\mu(Tf)g$
    \end{itemize}
\end{definition}

Composition in $C_T$ can be described in more detail via the operator
$$(-)^{*} : Hom(X, TY) \rightarrow Hom(TX, TY)$$
where given a morphism $f: X \rightarrow TY$ we have
$$f^{*} = \mu_{Y} \circ Tf$$
note that
$\eta_{X}^{*} = Id_{TX}$
and
$f^{*} \circ \eta _{X} = f$.
Then we can define the \textit{Kleisli operator} $\gg$ where
$$g \gg f = g^{*} \circ f$$
\begin{displaymath}
    x
    \stackrel{f}{\rightarrow}     T y
    \stackrel{T g}{\rightarrow}   T T z
    \stackrel{\mu z}{\rightarrow} T z
\end{displaymath}
where $\gg$ has the same axioms as a monoid operation
\begin{itemize}
    \item $(f \gg g) \gg h \equiv f \gg (g \gg h)$
    \item $\eta_Y \gg f \equiv f \equiv f \gg \eta_X$
\end{itemize}

\end{document}
