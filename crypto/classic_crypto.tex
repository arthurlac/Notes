\section{Base Definitions}
Cryptography allows for the secure exchange of information.
It is important to begin by noting that cryptography does not mean security.
What cryptography does is provide us \textit{primitives} for security; it also
allows us to analyse how secure we are being.

\vspace{5mm}
\textbf{Areas of Cryptography}\\
Firstly we should state that cryptography is not just limited to encryption
(and decryption of course).
However we will focus on encryption, most modern cryptography is encryption focused.
\begin{itemize}
    \item \textbf{Encryption} - Only the intended recipient can understand the message.
    \item \textbf{Steganography} - Only the intended recipient is aware that there is a message.
    \item \textbf{Chaffing and winnowing} - Only the intended recipient can find the real message.
\end{itemize}


\vspace{5mm}
\textbf{What do we use encryption for?}\\
Okay great, we know what encryption is but what is it used for?
These are the problems we want to tackle,
and encryption is going to be our tool.
\begin{itemize}
    \item \textbf{Confidentiality}: Keeping information a secret from those not authorised to have it.
    \item \textbf{Integrity}: Ensuring information has not been altered.
    \item \textbf{Authentication}: Confirmation of the identity of an entity.
    \item \textbf{Message authentication}: Confirmation of the source of information.
    \item \textbf{Signature}: A way of binding information to an entity.
    \item \textbf{Certification}: Endorsement of information by a trusted entity.
    \item \textbf{Non-repudiation}: Preventing an entity from denying previous actions or commitments.
    \item \textbf{Revocation}: Retracting certification or authorisation.
\end{itemize}

\vspace{5mm}
Now we can define a cryptosystem, which is the generic primitive of encryption.
\begin{definition}
    A \textit{cryptosystem} (or encryption system) is a five-tuple $(P,C,K,E,D)$, where
    \begin{enumerate}
        \item $P$ is a set of possible plaintexts
        \item $C$ is a set of possible ciphertexts
        \item $K$ is a set of possible keys called the keyspace
        \item For each key $k \in K$
            there is an encryption rule $e_k \in E$ where $e_k : P \rightarrow C$
            and a corresponding decryption rule where $d_k \in D$, $d_k : C \rightarrow P$
            such that
            $$\forall\ x \in P\ d_k(e_k(x)) = x$$
    \end{enumerate}
\end{definition}

For a cryptosystem to be useful in practice, we need:
\begin{enumerate}
    \item To be able to efficiently compute the encryption and the decryption functions.
    \item That an unauthorised party should not be able to determine the key or the plaintext from the ciphertext.
    \item We also need each encryption function $e_k$ to be injective or one to one.
\end{enumerate}

\subsection{Attack Models}
\begin{itemize}
    \item \textbf{Ciphertext only attack}
        Given: $y_1 = e_k(x_1),\dots y_i = e_k(x_i)$ deduce one of:
        \begin{itemize}
            \item $k$
            \item An algorithm that outputs $x_{i+1}$ given $y_{i+1} = e_k(x_{i+1})$
            \item $x_1,\dots x_i$
        \end{itemize}
    \item \textbf{Known plaintext attack}
        Given: $x_1, y_1 = e_k(x_1),\dots x_i,y_i = e_k(x_i)$ deduce:
        \begin{itemize}
            \item $k$
            \item An algorithm that outputs $x_{i+1}$ given $y_{i+1} = e_k(x_{i+1})$
        \end{itemize}
    \item \textbf{Chosen plaintext attack}
        Given: x1,y1 = ek(x1),...,xi,yi = ek(xi) where the attacker has chosen x1,...,xi deduce:
        \begin{itemize}
            \item k or
            \item an algorithm that outputs xi+1 given yi+1 = ek(xi+1)
        \end{itemize}
    \item \textbf{Adaptive chosen plaintext attack}\\
        This is a special case of the chosen plaintext attack.
        The attacker can modify his choice of plaintexts based on the results of earlier pairs.
    \item \textbf{Chosen ciphertext attack}
        Given: y1,x1 = dk(y1),...,yi,xi = dk(yi) deduce $k$
    \item \textbf{Non-cryptograph attacks}
        \begin{itemize}
            \item Bribery
            \item Physical theft
            \item Blackmail
            \item Threats
            \item Torture
        \end{itemize}
        Harder to analyse mathematically however.
\end{itemize}

\subsection{Security}
\begin{itemize}
    \item \textbf{Computational security}\\
        A cryptosystem is computationally secure if the best algorithm for breaking it requires a com- putational effort which is greater than the computational resources of the assumed attacker.
        • We need a measure of the computational effort to break the cryptosystem. • We can’t prove a system is computationally secure against all attacks.

    \item \textbf{Provable security}
        For some cryptosystems that provide confidentiality we can provide evidence that the system is secure by proving a theorem of the form:
        If the cryptosystem can be broken then we can efficiently solve problem A, where problem A is
        Well studied
        Thought to be “difficult”
        This is not an absolute proof of security but a proof of the security relative to another problem.

    \item \textbf{Unconditional security}
        A cryptosystem is said to be unconditionally secure if an attacker with infinite computational resources cannot break the system.

\end{itemize}

\textbf{An unconditionally secure cryptosystem}
In 1949 Shannon proved that the one-time pad is unconditionally secure.
Theorem 1.
%%Suppose (P, C, K, E, D) is a cryptosystem where |K| = |C| = |P|. Then the cryptosystem provides perfect secrecy if and only if every key is used with equal probability 1/|K|, and for everyx∈P andeveryy∈C,thereisauniquekeysuchthatek(x)=y.
(as stated in Cryptography Theory and Practice)

\textbf{Unconditional security is not practical}

Instead of aiming for unconditional security researchers have tried to develop cryptosystems where one shared key can be used to send many messages between to entities while still main- taining computational security.
