\section{Classical Cryptography}
\begin{definition}
	The \textit{subsitution cipher} has
	\begin{itemize}
		\item $P = C = \Z_{26}$
		\item $K$ is the set of all permutations of $\Z_{26}$
		\item
			$
			\forall\ \pi \in K
			\ e_\pi(x) = \pi(x) = y
			\ d_\pi(y) = \pi^{-1}(y) = x
			$
	\end{itemize}
\end{definition}

Consider

\begin{definition}
	The \textit{shift cipher}, also called the Caesar cipher,
\end{definition}

\begin{definition}
	The \textit{Vigen\'ere cipher cipher}
\end{definition}

\begin{definition}
	The \textit{permutation cipher}
\end{definition}

\subsection{What do we mean by secure?}

\subsection{Kerckhoff's desiderata (1883)}

\subsection{Attack Models}
\subsubsection{Ciphertext only attack}
\subsubsection{Known plaintext attack}
\subsubsection{Chosen plaintext attack}
\subsubsection{Adaptive chosen plaintext attack}
\subsubsection{Chosen ciphertext attack}
\subsubsection{Non-cryptograph attacks}

\subsection{Security}
\subsubsection{Computational security}
A cryptosystem is computationally secure if the best algorithm for breaking it requires a com- putational effort which is greater than the computational resources of the assumed attacker.
• We need a measure of the computational effort to break the cryptosystem. • We can’t prove a system is computationally secure against all attacks.

\subsubsection{Provable security}
For some cryptosystems that provide confidentiality we can provide evidence that the system is secure by proving a theorem of the form:
If the cryptosystem can be broken then we can efficiently solve problem A, where problem A is
 Well studied
 Thought to be “difficult”
This is not an absolute proof of security but a proof of the security relative to another problem.

\subsubsection{Unconditional security}
A cryptosystem is said to be unconditionally secure if an attacker with infinite computational resources cannot break the system.

\begin{definition}
	\textit{Vernam cipher}
\end{definition}

\begin{definition}
	\textit{One time pad}
\end{definition}

\subsubsection{An unconditionally secure cryptosystem}
In 1949 Shannon proved that the one-time pad is unconditionally secure.
Theorem 1.
%%Suppose (P, C, K, E, D) is a cryptosystem where |K| = |C| = |P|. Then the cryptosystem provides perfect secrecy if and only if every key is used with equal probability 1/|K|, and for everyx∈P andeveryy∈C,thereisauniquekeysuchthatek(x)=y.
(as stated in Cryptography Theory and Practice)

\subsubsection{Unconditional security is not practical}

Instead of aiming for unconditional security researchers have tried to develop cryptosystems where one shared key can be used to send many messages between to entities while still main- taining computational security.
