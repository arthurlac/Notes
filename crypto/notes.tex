\documentclass{article}

\usepackage[utf8]{inputenc}
\usepackage[english]{babel}
\usepackage{amsfonts}
\usepackage{amssymb}
\usepackage{amsthm}
\usepackage{mathtools}
\usepackage{color}

\DeclareUnicodeCharacter{2212}{-}

\addtolength{\evensidemargin}{-.5in}
\addtolength{\oddsidemargin}{-.5in}
\addtolength{\textwidth}{0.8in}
\addtolength{\textheight}{0.8in}
\addtolength{\topmargin}{-.4in}

\newtheoremstyle{quest}{\topsep}{\topsep}{}{}{\bfseries}{}{ }{\thmname{#1}\thmnote{ #3}.}
\theoremstyle{quest}
\newtheorem*{definition}{Definition}
\newtheorem*{theorem}{Theorem}
\newtheorem*{question}{Question}
\newtheorem*{example}{Example}

\setcounter{secnumdepth}{4}

\newcommand{\Hsh}{\mathcal{H}}
\newcommand{\K}{\mathcal{K}}
\newcommand{\X}{\mathcal{X}}
\newcommand{\Y}{\mathcal{Y}}
\newcommand{\N}{\mathbb{N}}
\newcommand{\Z}{\mathbb{Z}}
\newcommand{\mns}{ - }
\newcommand{\Mod}[1]{\ (\mathrm{mod}\ #1)}
\newcommand{\Modpn}{\Mod{\phi(n)}}

\setlength{\parindent}{0pt}

\begin{document}
\tableofcontents

\pagebreak
\section{What is the problem}
Cryptography allows for the secure exchange of information.
It is important to begin by noting that cryptography does not mean security.
What cryptography does is provide us \textit{primitives} for security; it also
allows us to analyse how secure we are being.

\subsection{Areas of cryptography}
\begin{itemize}
  \item Encryption - Only the intended recipient can understand the message.
  \item Steganography - Only the intended recipient is aware that there is a message.
  \item Chaffing and winnowing - Only the intended recipient can find the real message.
\end{itemize}

We will focus on encryption.

\subsection{Goals}

\begin{itemize}
	\item Confidentiality: Keeping information a secret from those not authorised to have it.
	\item Data integrity: Ensuring information has not been altered by those not authorised to do so.
	\item Authentication: Confirmation of the identity of an entity.
	\item Message authentication: Confirmation of the source of information.
	\item Signature: A way of binding information to an entity.
	\item Certification: Endorsement of information by a trusted entity.
	\item Non-repudiation: Preventing an entity from denying previous actions or commitments.
	\item Revocation: Retracting certification or authorisation.
\end{itemize}

\subsection{Cryptosystem}
\begin{definition}
	A \textit{cryptosystem} (or encryption system) is a five-tuple $(P,C,K,E,D)$, where
	\begin{enumerate}
		\item $P$ is a finite set of possible plaintexts
		\item $C$ is a finite set of possible ciphertexts
		\item $K$ is a finite set of possible keys called the keyspace
		\item For each key $k \in K$
			there is an encryption rule $e_k \in E$ where $e_k : P \rightarrow C$
			and a corresponding decryption rule where $d_k \in D$, $d_k : C \rightarrow P$
			such that
			$$\forall\ x \in P\ d_k(e_k(x)) = x$$
		\end{enumerate}
\end{definition}

For a cryptosystem to be useful in practice, we need:
1. to be able to efficiently compute the encryption and the decryption functions
2. that an unauthorised party should not be able to determine the key or the plaintext
We also need each encryption function ek to be injective or one to one. Why?

\pagebreak
\section{Classical Cryptography}
\begin{definition}
	The \textit{subsitution cipher} has
	\begin{itemize}
		\item $P = C = \Z_{26}$
		\item $K$ is the set of all permutations of $\Z_{26}$
		\item
			$
			\forall\ \pi \in K
			\ e_\pi(x) = \pi(x) = y
			\ d_\pi(y) = \pi^{-1}(y) = x
			$
	\end{itemize}
\end{definition}

Consider

\begin{definition}
	The \textit{shift cipher}, also called the Caesar cipher,
\end{definition}

\begin{definition}
	The \textit{Vigen\'ere cipher cipher}
\end{definition}

\begin{definition}
	The \textit{permutation cipher}
\end{definition}

\subsection{What do we mean by secure?}

\subsection{Kerckhoff's desiderata (1883)}

\subsection{Attack Models}
\subsubsection{Ciphertext only attack}
\subsubsection{Known plaintext attack}
\subsubsection{Chosen plaintext attack}
\subsubsection{Adaptive chosen plaintext attack}
\subsubsection{Chosen ciphertext attack}
\subsubsection{Non-cryptograph attacks}

\subsection{Security}
\subsubsection{Computational security}
A cryptosystem is computationally secure if the best algorithm for breaking it requires a com- putational effort which is greater than the computational resources of the assumed attacker.
• We need a measure of the computational effort to break the cryptosystem. • We can’t prove a system is computationally secure against all attacks.

\subsubsection{Provable security}
For some cryptosystems that provide confidentiality we can provide evidence that the system is secure by proving a theorem of the form:
If the cryptosystem can be broken then we can efficiently solve problem A, where problem A is
 Well studied
 Thought to be “difficult”
This is not an absolute proof of security but a proof of the security relative to another problem.

\subsubsection{Unconditional security}
A cryptosystem is said to be unconditionally secure if an attacker with infinite computational resources cannot break the system.

\begin{definition}
	\textit{Vernam cipher}
\end{definition}

\begin{definition}
	\textit{One time pad}
\end{definition}

\subsubsection{An unconditionally secure cryptosystem}
In 1949 Shannon proved that the one-time pad is unconditionally secure.
Theorem 1.
%%Suppose (P, C, K, E, D) is a cryptosystem where |K| = |C| = |P|. Then the cryptosystem provides perfect secrecy if and only if every key is used with equal probability 1/|K|, and for everyx∈P andeveryy∈C,thereisauniquekeysuchthatek(x)=y.
(as stated in Cryptography Theory and Practice)

\subsubsection{Unconditional security is not practical}

Instead of aiming for unconditional security researchers have tried to develop cryptosystems where one shared key can be used to send many messages between to entities while still main- taining computational security.

\pagebreak
\section{Private Key Cryptography}

\subsection{Combining basic components}
\subsection{Substitution-permutation networks}
\subsection{Attacks on substitution-permutation networks}
\subsection{The Data Encryption Standard (DES)}
\subsection{Modes of operation}
\subsection{After DES: 3DES and the Advanced Encryption Standard (AES)}


\pagebreak
\section{Cryptographic Hashes}

\textit{Hash functions} are functions which
map data of arbitrary size to a bit string of a fixed size,
the output being called a hash.
To qualify as hash functions these functions must have several qualities;
firstly that the function \textit{compresses} the input,
the output must always be the same size.
Secondly the function must be "easy" to compute;
i.e. computable algorithmically in polynomial time
with no massive hidden constants.

More formally we have
\begin{itemize}
	\item $\X$ is the set of messages.
	\item $\Y$ is the finite set of hashes,
		where each hash is of fixed size $n$ such that
		$\forall\ y \in \Y\ |y| = n$.
	\item $h : \X \rightarrow \Y$ is the hash function.
	\item A pair $(x,y) \in (\X,\Y)$ is valid iff $h(x) = y$.
\end{itemize}

\begin{definition}
	A \textit{keyed hash function} is the tuple $(\X,\Y,\K,\Hsh)$ where
	\begin{enumerate}
		\item $\X$ is the set of messages
		\item $\Y$ is the finite set of hashes.
		\item $\forall\ k \in \K$ there exists $h_k : \X \rightarrow \Y \in \Hsh$
			such that a pair $(x,y) \in (\X, \Y)$ is considered valid iff $h_k(x) = y$.
	\end{enumerate}
\end{definition}

\textit{Cryptographic hash functions} are a subset of hash functions
for which given $h(x) = y$ it is easy to verify that $h(x)$ is $y$
but given just $y$ it is not easy to find $x$.
The ideal cryptographic hash function has five main properties:
\begin{enumerate}
	\item it is deterministic so the same message always results in the same hash
	\item it is quick to compute the hash value for any given message
	\item it is infeasible to generate a message from its hash value except by trying all possible messages
	\item a small change to a message should change the hash value so extensively that the new hash value appears uncorrelated with the old hash value
	\item it is infeasible to find two different messages with the same hash value
\end{enumerate}
A hash function is considered to be secure if three problems are difficult to solve.
\begin{itemize}
	\item \textit{Preimage 1}
		Given $h : \X \rightarrow \Y$
		and $y \in Y$
		find $x \in X$
		such that $h(x) = y$.
	\item \textit{Preimage 2}
		Given $h : \X \rightarrow \Y$
		and $x \in \X$
		find $x\prime \in \X\ x \neq x\prime$
		such that $h(x) = h(x\prime)$.
	\item \textit{Collision}
		Given only $h : \X \rightarrow \Y$
		find $x, x\prime \in \X$
		such that $x \neq x\prime, h(x) = h(x\prime)$.
\end{itemize}

Na\"ive attack

Birthday attack problem

Summarily, a hash function is secure if
a preimage attack requires $2^n$ operations and
a collision attack requires attack $2^{n/2}$ operations.

\textit{Birthday paradox}

\subsection{Merkle-Damg\aa rd Construction}
The Merkle-Damg\aa rd construction is 
a method of building collision-resistant cryptographic hash functions 
from collision-resistant one-way compression functions.
A one-way compression function is a function
that transforms two fixed-length inputs into a fixed-length output
A hash function must be able to process an arbitrary-length message into a fixed-length output.
This can be achieved by breaking the input up into a series of equal-sized blocks,
and operating on them in sequence using a one-way compression function.
The Merkle–Damgård hash function first applies an MD-compliant padding function to create an input whose size is a multiple of a fixed number (e.g. 512 or 1024) — this is because compression functions cannot handle inputs of arbitrary size. The hash function then breaks the result into blocks of fixed size, and processes them one at a time with the compression function, each time combining a block of the input with the output of the previous round.

In order to make the construction secure, Merkle and Damgård proposed that messages be padded with a padding that encodes the length of the original message. This is called length padding or Merkle–Damgård strengthening.
The algorithm starts with an initial value, the initialization vector (IV). The IV is a fixed value (algorithm or implementation specific). For each message block, the compression (or compacting) function f takes the result so far, combines it with the message block, and produces an intermediate result. The last block is padded with zeros as needed and bits representing the length of the entire message are appended.

the padding scheme used in the Merkle–Damgård construction must be chosen carefully to ensure the security of the scheme
With these conditions in place, we find a collision in the MD hash function exactly when we find a collision in the underlying compression function. Therefore, the Merkle–Damgård construction is provably secure when the underlying compression function is secure


\subsection{Message Digest (MD) Hash functions}

\subsubsection{MD4}

\subsubsection{MD5}

\subsection{Secure Hash Algorithm (SHA)}

\subsubsection{SHA-1}

\subsubsection{SHA-2 Family}

\subsubsection{SHA-3}


\pagebreak
\section{Key Distribution Problem}
The problem: Distributing private keys
We will consider two ways that an organisation might distribute keys:
\begin{itemize}
    \item. A key predistribution scheme
    \item. A session key distribution scheme
\end{itemize}
25 A key predistribution scheme (PKS)
Definition 18.
\begin{definition}
    \begin{itemize}
        \item A trusted authority distributes keying information in advance
        \item Keying information is distributed using secure channels
        \item Keying information might be long lived keys or secret information that can later be used to produce keys
    \end{itemize}
\end{definition}
26 A session key distribution scheme (SKDS) Definition 19.
\begin{definition}
    \begin{itemize}
        \item A session key scheme involves a trusted authority and two users, Alice and Bob, who wish to communicate.
        \item The objective is the production of a session key known only to Alice and Bob (and possibly the TA).
        \item Session keys are used to encrypt information for a specified, short period of time.
        \item Often each network user will share a different long lived key with the TA.
    \end{itemize}
\end{definition}

\pagebreak
\section{Public-Key Cryptography}
\textit{Public key cryptography}, or asymmetrical cryptography,
is any cryptographic system that uses pairs of keys: public keys which may be disseminated widely,
and private keys which are known only to the owner.
The essential advantage of public key cryptography is that it provides secure communication between two people who had not met or exchanged securely a secret key.
This accomplishes two functions: authentication, where the public key verifies that a holder of the paired private key sent the message,
and encryption, where only the paired private key holder can decrypt the message encrypted with the public key.

\textbf{1976: Diffie and Hellman}
“We stand today on the brink of a revolution in cryptography.”
\begin{itemize}
    \item Gave an abstract way of providing secure communication between two people who had not met or exchanged securely a secret key.
    \item Argued how such a system could also provide secure digital signatures.
    \item Gave a practical method by which two people, without the aid of a trusted authority, can
        establish a shared secret key using an insecure channel.
\end{itemize}

\begin{definition}
    The essence of a \textit{public key cryptosystem} is defined as the tuple
    $(P,C,K,E,D)$
    \begin{itemize}
        \item For every key $k \in K$, $e_k$ is the inverse of $d_k$
        \item For every key $k \in K$,
            and for every $x \in P$ and $y \in C$
            $e_k(x)$ and $d_k(y)$ are easy to compute
        \item It is computationally infeasible for almost all $k \in K$
            to derive $d_k$ from $e_k$.
        \item For every $k \in K$ it is feasible to derive $e_K$ and $d_k$
    \end{itemize}
    We have $e_k$ as the public key which allows \textbf{anyone} to encrypt a message
    which only the holder of $d_k$ will be able to decrypt.
\end{definition}

Clearly a public-key cryptosystem can never be unconditionally secure.
\begin{itemize}
    \item Alice looks up Bob’s public key function $e_k$ and encrypts $x : y = e_k(x)$.
    \item Oscar encrypts each possible message in turn until he finds the unique $x$ such that $y = e_k(x)$.
\end{itemize}
Note that Oscar can always launch a chosen-plaintext attack.

\begin{definition}
    A \textit{one-way function} is a function $f : X \rightarrow Y$ such that 
    for all $x \in X$ it is easy to compute $f(x)$ but for (almost) all $y \in Y$ it is computationally
    infeasible to find an $x$ where $f(x) = y$.
\end{definition}
\begin{definition}
    A \textit{trap-door one-way function} is a one way function $f$ such that
    given some additional trap-door information it becomes feasible, 
    for all $y \in Y$ to find $x \in X$ such that $y = f(x)$.
\end{definition}

Given a public-key cryptosystem in which $P = C$
using the fact that $e_k$ and $d_k$ are inverses
we can define a secure digital signature scheme as such
we might define a mechanism to allow secure digital signatures:
\begin{itemize}
    \item
        Alice wishes to sign message $x$.
        Using her private decryption function she obtains $y$ where $y = d_k(x)$;
        sending $y$ to Bob.
    \item Bob can then encrypt $y$ with Alice’s public encryption function: $e_k(y) = x$
    \item Only Alice could have computed $y$ such that $e_k(y) = x$ hence Bob is convinced that
    Alice signed the message.
    Furthermore anyone could have checked Alice’s signature, not just Bob.
\end{itemize}

\subsection{Mathematical background}
\subsubsection{The Euler function}
\begin{definition}
    The \textit{Euler function} $\phi(x)$ is
\end{definition}
It has the following properties
\begin{itemize}
    \item If $p$ is prime then $\phi(p)$ is $p - 1$
    \item $\phi$ is multiplicative; that is,
        if $gcd(m,n) = 1$ then $\phi(m \cdot n) = \phi(m) \cdot \phi(n)$
    \item If $\phi(n) \geq 2$ then $\phi(n)$ is even.
\end{itemize}

\subsubsection{Fermat's Little Theorem}
\begin{definition}
    \textit{Fermat's little theorem} states that if $p$ is a prime number,
    then for any integer $a$ the number $a^p - a$ is an integer multiple of $p$.
    In the notation of modular arithmetic this is expressed as
    $$ a^p \equiv a \Mod p $$
\end{definition}
For example, if $a = 2$ and $p = 7$, then $2^7 = 128$, and 128 - 2 = 126 = 7 $\times$ 18.

\subsubsection{Chinese Remainder Theorem}
\begin{definition}
    \textit{Chinese Remainder Theorem} is a way of solving certain systems of congruences.
    %%Letm1,m2,...,mr bepairwiserelativelyprimepositiveintegers. Supposea1,a2,...,ar ∈Zandconsider:
    $$
    x \equiv a_1 (\Mod m_1) \\
    x \equiv a_2 (\Mod m_2) \\
    x \equiv a_r (\Mod m_r)
    $$
    %%The Chinese remainder theorem states that this system has a unique solution modulo M = m1m2 ···mr
\end{definition}

\subsubsection{The Euclidean Algorithm}

\subsubsection{The Extended Euclidean Algorithm}

\subsection{RSA Cryptosystem}
The Rivest-Shamir-Adleman cryptosystem,
named after its authors, was one of the first public key cryptosystems.
RSA is a relatively slow algorithm, and because of this,
it is less commonly used to directly encrypt user data.
More often, RSA passes encrypted shared keys for symmetric key cryptography
which in turn can perform bulk encryption-decryption operations at much higher speed.

\begin{definition}
    The \textit{RSA cryptosystem} is defined as such
    \begin{itemize}
        \item Let $p$ and $q$ be large primes, (in some implementations 1024 bits each)
        \item Let $n = p \cdot q$
        \item Let $P = C = \Z_n$
        \item $K = \{(n,p,q,a,b) | ba \equiv 1 \Mod{\phi(n)}\}$
        \item $(n,a)$ is the public key
        \item $(n,b)$ is the public key
        \item For $K = (n,p,q,a,b)$ we have
            $$e_k(x) = x^b \Mod n = y$$
            $$d_k(y) = y^a \Mod n = x$$
    \end{itemize}
\end{definition}

\subsection{RSA Toy Example}
Firstly Bob chooses $p = 127$ and $q = 131$ and calculates $n$ and $\phi(n)$
\begin{align*}
    n &= 127  \times 131  &\phi(n) &= (p \mns 1) \times (q \mns 1)\\
      &= 16637            &        &= 126        \times 130       \\
      &                   &        &= 16380
\end{align*}
Bob selects some $b$ such that $gcd(b, \phi(n)) = 1$;
we shall use $b = 4057$ and calculate $a$
\begin{align*}
    1      &= ab \Mod{\phi(n)}\\
    b^{-1} &= a \Mod{\phi(n)}\\
           &= 10453 \Mod{\phi(n)}\\
         a &= 10453
\end{align*}
Thus Bob has the public key $(n,b) = (16637,4057)$ and the private key $(p,q,a) = (127, 131, 10453)$.
Alice has Bob’s public key and she wishes to encrypt the plaintext $x = 9031$.
She computes:
\begin{align*}
   y &= x^b \Mod{n} \\
     &= 90314057 \Mod{16637} \\
     &= 1870
\end{align*}
So Alice sends $y = 1870$ over the insecure channel. Bob now uses his decryption exponent a to compute:
\begin{align*}
   x &= y^a \Mod{n} \\
     &= 187010453 \Mod{16637} \\
     &= 9031
\end{align*}

\subsection{RSA Implementation Concerns}
To be ready to use RSA we have to
\begin{enumerate}
    \item Generate two large primes $p, q$ such that $p \neq q$
    \item Calculate $n = p \cdot q$ and $\phi(n) = (p \mns 1) (q \mns 1)$
    \item Choose a random $1 < b < \phi(n)$ such that $gcd(b, \phi(n)) = 1$
    \item Calculate $a = b^{-1} \Mod{\phi(n)}$
    \item Set the public key to $(n, b)$ and the private key to $(p, q, a)$
\end{enumerate}
For RSA to be a practical algorithm we need these steps to be computationally feasible.
Firstly the issue of generating large primes
\begin{itemize}
    \item Basic method: generate a large random number, test for primality.
    \item PRIMES is in P: In 2002 Agrawal, Kayal and Saxena proved that
        there is a polynomial time deterministic algorithm for primality testing; but we don’t use this.
    \item Instead we use one of a variety of randomised polynomial time algorithms. The Miller-Rabin algorithm is the most prominent.
\end{itemize}

The other large cost involved in the parameter generation is the calculation of
$a = b^{-1} \Mod{\phi(n)}$
We use extended Euclidean algorithm.
Can be calculated in time $O((log k)^2)$
What about the modular exponentiation in RSA?
To calculate $x^b \Mod{n}$ we can
\begin{itemize}
    \item Very slow: $b \mns 1$ modular multiplications
    \item Faster: use the square and multiply algorithm
    \item Faster again: use Chinese remainder theorem (if you are Bob)
\end{itemize}

\subsection{RSA and Factoring}
Oscar, listening to communications will try to recover $x$ given $x^b$ $\Mod{n}$, $n$ and $b$.
This is called the RSA problem.
There is no efficient algorithm known to solve this problem.
The security of RSA rests on the fact that given $n$
it is hard to find primes $p,q$ such that $n = p \cdot q$.
If factoring is computationally feasible RSA is insecure
as one can just find $p$ and $q$ and then recreate the private key $a$.
A concern is that finding $\phi(n)$ is easier than factoring $n$;
given that $$1 = ab \Mod{\phi(n)}$$
Fortunately it is the case that calculating $\phi(n)$ is as computationally difficult as factoring $n$.

\textbf{Factoring as an attack}
\begin{definition}
    \textit{Splitting}
It suffices to study algorithms which split n, that is, 
    find $1 < a < n, 1 < b < n$ such that $n = ab$.
    b a and b are said to be non-trivial factors of n.
\end{definition}

Trial division

Before using a more expensive method on n we would usually trial divide by “small” primes.
 “small” is considered as a function of the size of n
In the extreme case we trial divide with all primes $p \leq abs(\sqrt{n})$
In the worst case n is a product of two primes of roughly the same size
Perfectly reasonable method if $n < 10^{12}$

There are two categories of factoring algorithm:
Special purpose factoring algorithms
General purpose factoring algorithms

General purpose algorithms include
\begin{itemize}
    \item The quadratic sieve
    \item The general number field sieve
\end{itemize}

Special purpose algorithms include
\begin{itemize}
    \item Pollard’s $p \mns 1$ method
    \item William’s $p + 1$ method
    \item The elliptic curve factoring method
\end{itemize}

Special purpose factoring algorithms and RSA?
Many authors have recommended that p, q be chosen to be strong primes.
\begin{definition}
    $p$ prime is strong if
    \begin{itemize}
        \item $p \mns 1$ has a large prime factor (denoted $r$)
        \item $p + 1$ has a large prime factor
        \item $r \mns 1$ has a large prime factor
    \end{itemize}
\end{definition}

However, strong primes offer no protection against the Elliptic Curve Method

\subsubsection{Elliptic curve method}
\begin{itemize}
    \item The Elliptic curve method (ECM) is a generalisation of Pollard’s $p \mns 1$ method.
    \item Depends on an integer “close to” $p$ having only “small” prime factors.
    \item This is more likely than the situation required by Pollard’s $p \mns 1$ method.
    \item Pollard’s method depends on a relation which holds in the group $\mathbb{Z}^*_p$.
    \item ECM depends on a relation which involves groups defined on elliptic curves modulo p.
    \item Tends to find smaller factors first.
\end{itemize}

\subsubsection{Factoring with Congruent Squares}
To factor $n$ look for $x,y \in \mathbb{Z}$ such that
$$x^2 \mns y^2 = n$$
If such $x, y$ can be found then 
$$d = gcd(x \mns y, n)$$
is a non trivial factor of $n$.

In the 1920’s Kraitchik modified the idea of difference of two squares: Congruent squares.
To factor n look for $x,y \in \mathbb{Z}$ such that
$$x^2 \equiv y^2 \Mod{n}$$

Proposition 4.
If we also have $x\not\equiv \pm y \Mod{n}$ then
$$gcd(x \mns y, n) \quad\textrm{and}\quad gcd(x + y, n)$$
are non trivial factors of $n$.

Given $a$ we may factor $n$.
Imagine Oscar has Bob’s public key $(n, b)$ and his decryption exponent $a$.
Oscar can factor $n$.
By definition $ba \equiv 1 \Modpn$
$$ba - 1 = k\phi(n)\quad\textrm{for some } k \in \mathbb{N}$$
Hence, $\forall x \in \mathbb{Z}_n^*$:
$$x^{ba \mns 1} \equiv \Mod{n}$$
$$y_1 = \sqrt{x^{ba \mns 1}} = x^{(ba\mns1)/2}$$
$$y_1^2 \equiv 1 \Mod{n}$$
We already have that
so we can again attempt to factor n.
Continue process until either
You find a non trivial factor of n
$(ba \mns 1)/2s$ is not divisible by 2
We factor $n$ with probability half.

\begin{example}
    $$n = 437 = 19 \times 23$$
    $$\phi(n) = (19\mns1) \times (23\mns1) = 396$$
    $$b = 7$$
    $$a \equiv b^{\mns1} \equiv 7^{\mns1} \equiv 283 \Modpn$$
    Assume we know $a$
    $$\frac{ba\mns1}{2} = \frac{(7\times283)\mns1}{2} = 990$$
    $$\textrm{Try } x = 2$$
\end{example}

\begin{definition}
    An integer $n \in \mathbb{N}$ is said to be \textit{B-Smooth} if
\end{definition}

\begin{example}
\end{example}
\subsubsection{Dixon’s Random Squares}

\subsection{RSA Security}
\subsubsection{Small Private Exponent}
\subsubsection{Small Public Exponent}
\subsubsection{Coppersmith’s Theorem}
\subsubsection{Hastad’s Broadcast Attack}
\subsubsection{Franklin-Reiter Related Messages}
\subsubsection{Padding}
\subsubsection{Crouch/Davenport}

\subsection{RSA in practice}
\subsubsection{RSA Optimal Asymmetric Encryption Padding}

\pagebreak
\subsection{Diffie-Hellman key exchange}
\textit{Diffie–Hellman key exchange} (DH) is a method of securely exchanging cryptographic keys over a public channel
and was one of the first public-key protocols as originally conceptualized
by Ralph Merkle and named after Whitfield Diffie and Martin Hellman.
DH is one of the earliest practical examples of public key exchange implemented within the field of cryptography.

\subsubsection{Cryptographic explanation}
The simplest and the original implementation of the protocol uses the multiplicative group of integers modulo p,
where p is prime, and $\alpha$ is a primitive root modulo p.
These two values are chosen in this way to ensure that the resulting shared secret can take on any value from 1 to $p - 1$.
Public parameters: $p$ prime, and $\alpha$, a primitive element modulo $p$ which has order $p − 1$.

\begin{enumerate}
    \item Alice chooses $1 \leq a \leq p − 1$ at random and computes and sends $A$ to Bob.
        $$A = \alpha^a \Mod{p}$$
        \item Bob chooses $1 \leq b \leq p - 1$ and sends $B$ to Alice
        $$B = \alpha^b \Mod{p}$$
        \item Alice receives B and computes
        $$B^a = \alpha^{ab} \Mod{p}$$
        \item Bob received A and computes
        $$A^b = \alpha^{ab} \Mod{p}$$
        \item Thus we have
            $$k = A^b = B^a = \alpha^{ab} \Mod{p}$$
\end{enumerate}
Here is an example of the protocol, with \textcolor{blue}{non-secret values in blue},
and \textcolor{red}{secret values in red}.

\begin{enumerate}
    \item Alice and Bob agree to use a modulus $\textcolor{blue}{p = 23}$ and base $\textcolor{blue}{\alpha = 5}$ (which is a primitive root modulo 23).
    \item Alice chooses a secret integer \textcolor{red}{$a = 4$}, then sends Bob \textcolor{blue}{$A$}.
        \begin{equation*}
            \begin{split}
                \textcolor{blue}{A} &= \textcolor{blue}{\alpha}^{\textcolor{red}{a}} \Mod{\textcolor{blue}{p}} \\
                \textcolor{blue}{A} &= \textcolor{blue}{5}^{\textcolor{red}{4}} \Mod{\textcolor{blue}{23}} \\
                \textcolor{blue}{A} &= \textcolor{blue}{4}
            \end{split}
        \end{equation*}
    \item Bob chooses a secret integer \textcolor{red}{$b = 3$}, then sends Alice \textcolor{blue}{$B$}
        \begin{equation*}
            \begin{split}
                \textcolor{blue}{B} &= \textcolor{blue}{\alpha}^{\textcolor{red}{b}} \Mod{\textcolor{blue}{p}} \\
                \textcolor{blue}{B} &= \textcolor{blue}{5}^{\textcolor{red}{3}} \Mod{\textcolor{blue}{23}} \\
                \textcolor{blue}{B} &= \textcolor{blue}{10}
            \end{split}
        \end{equation*}
    \item Alice computes
        \begin{equation*}
            \begin{split}
                \textcolor{red}{k} &= \textcolor{blue}{B}^{\textcolor{red}{a}} \Mod{\textcolor{blue}{p}} \\
                \textcolor{red}{k} &= \textcolor{blue}{10}^{\textcolor{red}{4}} \Mod{\textcolor{blue}{23}} \\
                \textcolor{red}{k} &= \textcolor{blue}{18}
            \end{split}
        \end{equation*}
    \item Bob computes
        \begin{equation*}
            \begin{split}
                \textcolor{red}{k} &= \textcolor{blue}{A}^{\textcolor{red}{b}} \Mod{\textcolor{blue}{p}} \\
                \textcolor{red}{k} &= \textcolor{blue}{4}^{\textcolor{red}{3}} \Mod{\textcolor{blue}{23}} \\
                \textcolor{red}{k} &= \textcolor{blue}{18}
            \end{split}
        \end{equation*}
    \item Alice and Bob now share a secret (the number 18).
\end{enumerate}

\subsubsection{Computational Diffie-Hellman Problem (CDH)}
Given $p$ prime, $\alpha$ a primitive element modulo $p$
and two elements $\beta,\gamma \in \mathbb{Z}^*_p$
find $\delta \in \mathbb{Z}^*_p$ such that
$$log_\alpha\delta \equiv log_\alpha \beta \times log_\alpha \gamma \Mod{p}$$
That is, given $\alpha^a$ and $\alpha^b$, find $\alpha^{ab}$.

\vspace{4mm}
\textbf{Man in the Middle Attack}
\begin{itemize}
    \item Oscar intercepts all messages between Alice and Bob and substitutes his own messages in their place.
    \item At the end of the exchanges Alice has established a secret key with Oscar and Bob has established a secret key with Oscar.
    \item Can this be avoided?
\end{itemize}
\textbf{Attempts to avoid the attack}
\begin{itemize}
    \item Can Alice and Bob detect the attack?
    \item What if Alice and Bob prove their identities to each other before they start?
    \item A more promising approach would be to design a key agreement scheme where the participants are authenticated at the same time as the key is established. These are called authenticated key agreement schemes and we won’t be studying these.
\end{itemize}

\pagebreak
\subsection{The ElGamal cryptosystem}
The ElGamal encryption system is an asymmetric key encryption algorithm for public-key cryptography which is based on the Diffie–Hellman key exchange. The system provides an additional layer of security by asymmetrically encrypting keys previously used for symmetric message encryption. It was described by Taher Elgamal in 1985.

ElGamal encryption can be defined over any cyclic group
G
G. Its security depends upon the difficulty of a certain problem in
G
G related to computing discrete logarithms (see below).

\subsubsection{Encryption and Decryption}
\begin{itemize}
    \item $P = \Z^*_p$
    \item $C = \Z^*_p \times \Z^*_p$
    \item $K = \{(p,\alpha,a,\beta)| \beta \equiv \alpha^a \Mod{p}\}$
    \item The public key is $(p, \alpha, \beta)$. The private key is $a$
\end{itemize}

For $k = (p,\alpha,a,\beta) \in K$
and for a secret random number $r \in \mathbb{Z}_{p - 1}$
calculate $y \in \mathbb{Z}^*_p \times \mathbb{Z}^*_p$.
\begin{align*}
    e_k(x,r) &= (\alpha^r \Mod{p}, x\beta^r \Mod{p}) &\quad d_k(y_1,y_2) &= y_2(y_1^a)^{-1} \Mod{p} \\
             &= (y_1,y_2)                            &                   &= x
\end{align*}

\subsubsection{Example}
\begin{enumerate}
    \item Let $p = 107$, $\alpha = 2$ (a primitive element modulo 107).
    \item Let $a = 63$ so
        $$\beta = \alpha^a = 2^{63} \Mod{107} = 46$$
    \item Alice wants to send $x = 12$ to Bob and selects random number $r = 80$
    \item She computes $(y_1,y_2)$
            \begin{align*}
                y_1 &= \alpha^r \Mod{p} &\quad y_2 &= x \times \beta^r \Mod{p} \\
                    &= 2^{80} \Mod{107} &          &= 12 \times 46^{80} \Mod{107} \\
                    &= 76               &          &= 44
            \end{align*}
    \item Bob receives $(76, 44)$ and computes
        \begin{equation*}
            \begin{split}
                x &= y_2 \times (y_1^\alpha)^{-1} \Mod{p}\\
                  &= 44 \times (76^{63})^{-1} \Mod{107} \\
                  &= 12
            \end{split}
        \end{equation*}
\end{enumerate}

Notice that the encryption operation is randomised. A single plaintext may encrypt to many different ciphertexts depending on the value of r. How important is it that a different random number is used each time?
Clearly for this to be% a public key cryptosystem we need p to be chosen such that the discrete logarithm problem in Z∗p is infeasible.
There is no known poly%nomial time algorithm for this version of the discrete loga- rithm problem
We need to
• Prove that decryption works (problem sheet)
• Check that it is feasible to compute the encryption and decryption functions (think about it yourselves)
%• Consider algorithms for solving the discrete logarithm problem (next week)
%• We have already seen that given p we can find an appropriate α

\subsection{Discrete Logarithm Problem}
%Discrete logarithm problem (DLog problem)
%Given p prime, α a primitive element modulo p and β ∈ Z∗p find an element a, 0 ≤ a ≤ p−2
%such that
%αa≡β (modp) Generalized discrete logarithm problem
%Given a finite cyclic group G of order n, α a generator of G and β ∈ G find an element a,
%0 ≤ a ≤ n − 1 such that
%αa = β
%
%Types of algorithms
%• Algorithms which work in arbitrary groups
%– Exhaustive search
%– Shanks’ algorithm
%– Pollard’s rho algorithm
%• Algorithms which work in arbitrary groups but require the order of the group to have only small prime factors
%– Pohlig-Hellman algorithm
%• Algorithms which only work in certain groups
%– Index-calculus method


\pagebreak
\section{Signature Schemes}
\pagebreak
\section{Public Key Infrastructures}
\end{document}
