\documentclass{article}

\usepackage[utf8]{inputenc}
\usepackage[english]{babel}
\usepackage{amsfonts}
\usepackage{amssymb}
\usepackage{amsthm}
\usepackage{mathtools}

\addtolength{\evensidemargin}{-.5in}
\addtolength{\oddsidemargin}{-.5in}
\addtolength{\textwidth}{0.8in}
\addtolength{\textheight}{0.8in}
\addtolength{\topmargin}{-.4in}

\newtheoremstyle{quest}{\topsep}{\topsep}{}{}{\bfseries}{}{ }{\thmname{#1}\thmnote{ #3}.}
\theoremstyle{quest}
\newtheorem*{definition}{Definition}
\newtheorem*{theorem}{Theorem}
\newtheorem*{question}{Question}

\setcounter{secnumdepth}{4}

\newcommand{\Hsh}{\mathcal{H}}
\newcommand{\K}{\mathcal{K}}
\newcommand{\X}{\mathcal{X}}
\newcommand{\Y}{\mathcal{Y}}
\newcommand{\Z}{\mathbb{Z}}

\begin{document}

\section{What is the problem}
Cryptography allows for the secure exchange of information.
It is important to begin by noting that cryptography does not mean security.
What cryptography does is provide us \textit{primitives} for security; it also
allows us to analyse how secure we are being.

\subsection{Areas of cryptography}
\begin{itemize}
  \item Encryption - Only the intended recipient can understand the message.
  \item Steganography - Only the intended recipient is aware that there is a message.
  \item Chaffing and winnowing - Only the intended recipient can find the real message.
\end{itemize}

We will focus on encryption.

\subsection{Goals}

\begin{itemize}
	\item Confidentiality: Keeping information a secret from those not authorised to have it.
	\item Data integrity: Ensuring information has not been altered by those not authorised to do so.
	\item Authentication: Confirmation of the identity of an entity.
	\item Message authentication: Confirmation of the source of information.
	\item Signature: A way of binding information to an entity.
	\item Certification: Endorsement of information by a trusted entity.
	\item Non-repudiation: Preventing an entity from denying previous actions or commitments.
	\item Revocation: Retracting certification or authorisation.
\end{itemize}

\subsection{Cryptosystem}
\begin{definition}
	A \textit{cryptosystem} (or encryption system) is a five-tuple $(P,C,K,E,D)$, where
	\begin{enumerate}
		\item $P$ is a finite set of possible plaintexts
		\item $C$ is a finite set of possible ciphertexts
		\item $K$ is a finite set of possible keys called the keyspace
		\item For each key $k \in K$
			there is an encryption rule $e_k \in E$ where $e_k : P \rightarrow C$
			and a corresponding decryption rule where $d_k \in D$, $d_k : C \rightarrow P$
			such that
			$$\forall\ x \in P\ d_k(e_k(x)) = x$$
		\end{enumerate}
\end{definition}

For a cryptosystem to be useful in practice, we need:
1. to be able to efficiently compute the encryption and the decryption functions
2. that an unauthorised party should not be able to determine the key or the plaintext
We also need each encryption function ek to be injective or one to one. Why?

\section{Classical Cryptography}
\begin{definition}
	The \textit{subsitution cipher} has
	\begin{itemize}
		\item $P = C = \Z_{26}$
		\item $K$ is the set of all permutations of $\Z_{26}$
		\item
			$
			\forall\ \pi \in K
			\ e_\pi(x) = \pi(x) = y
			\ d_\pi(y) = \pi^{-1}(y) = x
			$
	\end{itemize}
\end{definition}

Consider

\begin{definition}
	The \textit{shift cipher}, also called the Caesar cipher,
\end{definition}

\begin{definition}
	The \textit{Vigen\'ere cipher cipher}
\end{definition}

\begin{definition}
	The \textit{permutation cipher}
\end{definition}

\subsection{What do we mean by secure?}

\subsection{Kerckhoff's desiderata (1883)}

\subsection{Attack Models}
\subsubsection{Ciphertext only attack}
\subsubsection{Known plaintext attack}
\subsubsection{Chosen plaintext attack}
\subsubsection{Adaptive chosen plaintext attack}
\subsubsection{Chosen ciphertext attack}
\subsubsection{Non-cryptograph attacks}

\subsection{Security}
\subsubsection{Computational security}
A cryptosystem is computationally secure if the best algorithm for breaking it requires a com- putational effort which is greater than the computational resources of the assumed attacker.
• We need a measure of the computational effort to break the cryptosystem. • We can’t prove a system is computationally secure against all attacks.

\subsubsection{Provable security}
For some cryptosystems that provide confidentiality we can provide evidence that the system is secure by proving a theorem of the form:
If the cryptosystem can be broken then we can efficiently solve problem A, where problem A is
 Well studied
 Thought to be “difficult”
This is not an absolute proof of security but a proof of the security relative to another problem.

\subsubsection{Unconditional security}
A cryptosystem is said to be unconditionally secure if an attacker with infinite computational resources cannot break the system.

\begin{definition}
	\textit{Vernam cipher}
\end{definition}

\begin{definition}
	\textit{One time pad}
\end{definition}

\subsubsection{An unconditionally secure cryptosystem}
In 1949 Shannon proved that the one-time pad is unconditionally secure.
Theorem 1.
%%Suppose (P, C, K, E, D) is a cryptosystem where |K| = |C| = |P|. Then the cryptosystem provides perfect secrecy if and only if every key is used with equal probability 1/|K|, and for everyx∈P andeveryy∈C,thereisauniquekeysuchthatek(x)=y.
(as stated in Cryptography Theory and Practice)

\subsubsection{Unconditional security is not practical}

Instead of aiming for unconditional security researchers have tried to develop cryptosystems where one shared key can be used to send many messages between to entities while still main- taining computational security.

\section{Private Key Cryptography}

\subsection{Combining basic components}
\subsection{Substitution-permutation networks}
\subsection{Attacks on substitution-permutation networks}
\subsection{The Data Encryption Standard (DES)}
\subsection{Modes of operation}
\subsection{After DES: 3DES and the Advanced Encryption Standard (AES)}

\section{Cryptographic Hashes}

\textit{Hash functions} are functions which
map data of arbitrary size to a bit string of a fixed size,
the output being called a hash.
To qualify as hash functions these functions must have several qualities;
firstly that the function \textit{compresses} the input,
the output must always be the same size.
Secondly the function must be "easy" to compute;
i.e. computable algorithmically in polynomial time
with no massive hidden constants.

More formally we have
\begin{itemize}
	\item $\X$ is the set of messages.
	\item $\Y$ is the finite set of hashes,
		where each hash is of fixed size $n$ such that
		$\forall\ y \in \Y\ |y| = n$.
	\item $h : \X \rightarrow \Y$ is the hash function.
	\item A pair $(x,y) \in (\X,\Y)$ is valid iff $h(x) = y$.
\end{itemize}

\begin{definition}
	A \textit{keyed hash function} is the tuple $(\X,\Y,\K,\Hsh)$ where
	\begin{enumerate}
		\item $\X$ is the set of messages
		\item $\Y$ is the finite set of hashes.
		\item $\forall\ k \in \K$ there exists $h_k : \X \rightarrow \Y \in \Hsh$
			such that a pair $(x,y) \in (\X, \Y)$ is considered valid iff $h_k(x) = y$.
	\end{enumerate}
\end{definition}

\textit{Cryptographic hash functions} are a subset of hash functions
for which given $h(x) = y$ it is easy to verify that $h(x)$ is $y$
but given just $y$ it is not easy to find $x$.
The ideal cryptographic hash function has five main properties:
\begin{enumerate}
	\item it is deterministic so the same message always results in the same hash
	\item it is quick to compute the hash value for any given message
	\item it is infeasible to generate a message from its hash value except by trying all possible messages
	\item a small change to a message should change the hash value so extensively that the new hash value appears uncorrelated with the old hash value
	\item it is infeasible to find two different messages with the same hash value
\end{enumerate}
A hash function is considered to be secure if three problems are difficult to solve.
\begin{itemize}
	\item \textit{Preimage 1}
		Given $h : \X \rightarrow \Y$
		and $y \in Y$
		find $x \in X$
		such that $h(x) = y$.
	\item \textit{Preimage 2}
		Given $h : \X \rightarrow \Y$
		and $x \in \X$
		find $x\prime \in \X\ x \neq x\prime$
		such that $h(x) = h(x\prime)$.
	\item \textit{Collision}
		Given only $h : \X \rightarrow \Y$
		find $x, x\prime \in \X$
		such that $x \neq x\prime, h(x) = h(x\prime)$.
\end{itemize}

Na\"ive attack

Birthday attack problem

Summarily, a hash function is secure if
a preimage attack requires $2^n$ operations and
a collision attack requires attack $2^{n/2}$ operations.

\textit{Birthday paradox}

\subsection{Merkle-Damg\aa rd Construction}
The Merkle-Damg\aa rd construction is 
a method of building collision-resistant cryptographic hash functions 
from collision-resistant one-way compression functions.
A one-way compression function is a function
that transforms two fixed-length inputs into a fixed-length output
A hash function must be able to process an arbitrary-length message into a fixed-length output.
This can be achieved by breaking the input up into a series of equal-sized blocks,
and operating on them in sequence using a one-way compression function.
The Merkle–Damgård hash function first applies an MD-compliant padding function to create an input whose size is a multiple of a fixed number (e.g. 512 or 1024) — this is because compression functions cannot handle inputs of arbitrary size. The hash function then breaks the result into blocks of fixed size, and processes them one at a time with the compression function, each time combining a block of the input with the output of the previous round.

In order to make the construction secure, Merkle and Damgård proposed that messages be padded with a padding that encodes the length of the original message. This is called length padding or Merkle–Damgård strengthening.
The algorithm starts with an initial value, the initialization vector (IV). The IV is a fixed value (algorithm or implementation specific). For each message block, the compression (or compacting) function f takes the result so far, combines it with the message block, and produces an intermediate result. The last block is padded with zeros as needed and bits representing the length of the entire message are appended.

the padding scheme used in the Merkle–Damgård construction must be chosen carefully to ensure the security of the scheme
With these conditions in place, we find a collision in the MD hash function exactly when we find a collision in the underlying compression function. Therefore, the Merkle–Damgård construction is provably secure when the underlying compression function is secure


\subsection{Message Digest (MD) Hash functions}

\subsubsection{MD4}

\subsubsection{MD5}

\subsection{Secure Hash Algorithm (SHA)}

\subsubsection{SHA-1}

\subsubsection{SHA-2 Family}

\subsubsection{SHA-3}

\section{Key Distribution Problem}

\section{Public-Key Cryptography}

\section{Signature Schemes}

\section{Public Key Infrastructures}

\end{document}
