\documentclass{article}

\usepackage[utf8]{inputenc}
\usepackage[english]{babel}
\usepackage{amsfonts}
\usepackage{amssymb}
\usepackage{amsthm}
\usepackage{mathtools}
\usepackage{color}
\usepackage{tikz}
\usetikzlibrary{calc}
\usetikzlibrary{positioning}

\DeclareUnicodeCharacter{2212}{-}

\addtolength{\evensidemargin}{-.5in}
\addtolength{\oddsidemargin}{-.5in}
\addtolength{\textwidth}{0.8in}
\addtolength{\textheight}{0.8in}
\addtolength{\topmargin}{-.4in}

\newtheoremstyle{quest}{\topsep}{\topsep}{}{}{\bfseries}{}{ }{\thmname{#1}\thmnote{ #3}.}
\theoremstyle{quest}
\newtheorem*{definition}{Definition}
\newtheorem*{theorem}{Theorem}
\newtheorem*{question}{Question}
\newtheorem*{example}{Example}

\setcounter{secnumdepth}{4}

\newcommand{\Hsh}{\mathcal{H}}
\newcommand{\K}{\mathcal{K}}
\newcommand{\X}{\mathcal{X}}
\newcommand{\Y}{\mathcal{Y}}
\newcommand{\N}{\mathbb{N}}
\newcommand{\Z}{\mathbb{Z}}
\newcommand{\mns}{ - }
\newcommand{\Mod}[1]{\ (\mathrm{mod}\ #1)}
\newcommand{\Modpn}{\Mod{\phi(n)}}

\setlength{\parindent}{0pt}


\usetikzlibrary{crypto.symbols}
\tikzset{shadows=no}
\newcommand\feistel{\begin{tikzpicture}
    \foreach \z in {1, 2,...,3} {
        \node[draw,thick,minimum width=1cm] (f\z) at ($\z*(0,-1.5cm)$)  {$f_\z$};
        \node (xor\z) [XOR, left of = f\z, node distance = 2cm, scale=0.8] {};
        \draw[thick,-latex] (f\z) -- (xor\z);
    }
    \foreach \z in {1, 2} {
        \draw[thick,latex-latex] (f\z.east) -| +(1.5cm,-0.5cm) -- ($(xor\z) - (0,1cm)$) -- ($(xor\z.north) - (0,1.5cm)$);
        \draw[thick] (xor\z.south) -- ($(xor\z)+(0,-0.5cm)$) -- ($(f\z.east) + (1.5cm,-1cm)$) -- +(0,-0.5cm);
    }

    %% Inputs    
    \node (p0) [above of = f1, minimum width=5cm,minimum height=0.5cm,node distance=1cm] {}; 
    \node (l0) [above of = xor1,node distance=1cm] {$L_0$};
    \node (r0) [right of = l0, node distance = 4cm] {$R_0$};
    \draw[thick,-latex] (l0 |- p0.south) -- (xor1.north);
    \draw[thick] ($(f1.east)+(1.5cm,0)$) -- +(0,0.75cm);

    %% Outputs
    \node (p3) [below of = f3, minimum width=5cm,minimum height=0.5cm,node distance=1.75cm] {}; 
    \node (l3) [below of = xor3,node distance=1.75cm] {$L_3$};
    \node (r3) [right of = l3, node distance = 4cm] {$R_3$};
    \draw[thick,latex-latex] (f3.east) -| +(1.5cm,-0.5cm) -- ($(xor3) - (0,1cm)$) -- (xor3 |- p3.north);
    \draw[thick,-latex] (xor3.south) -- ($(xor3)+(0,-0.5cm)$) -- ($(f3.east) + (1.5cm,-1cm)$) -- +(0,-0.5cm);

\end{tikzpicture}}

\newcommand\des{\begin{tikzpicture}
% First two rounds
    \node[draw,thick,minimum width=1cm] (f1) at ($1*(0,-1.5cm)$)  {$f_1$};
    \node (xor1) [XOR, left of = f1, node distance = 2cm] {};
    \draw[thick,-latex] (f1) -- (xor1);

    \node[draw,thick,minimum width=1cm] (f2) at ($2*(0,-1.5cm)$)  {$f_2$};
    \node (xor2) [XOR, left of = f2, node distance = 2cm] {};
    \draw[thick,-latex] (f2) -- (xor2);

    \draw[thick,latex-latex] (f1.east) -| +(1.5cm,-0.5cm) -- ($(xor1) - (0,1cm)$) -- ($(xor1.north) - (0,1.5cm)$);
    \draw[thick] (xor1.south) -- ($(xor1)+(0,-0.5cm)$) -- ($(f1.east) + (1.5cm,-1cm)$) -- +(0,-0.5cm);

    \draw[thick,latex-] (f2.east) -| +(1.5cm,-0.5cm) -- ($(xor2) - (0,1cm)$);
    \draw[thick] (xor2.south) -- ($(xor2)+(0,-0.5cm)$) -- ($(f2.east) + (1.5cm,-1cm)$);

    \draw[thick, densely dotted] ($(f2.east) + (1.5cm,-1cm)$) -- +(0,-0.5cm);
    \draw[thick, densely dotted] ($(xor2) - (0,1cm)$) -- ($(xor2.north) - (0,1.5cm)$);

    % Middle text
    \node at (0,-4.5cm) {\scriptsize{for 16 rounds}};

    % Last two rounds
    \node[draw,thick,minimum width=1cm] (f3) at ($3*(0,-1.5cm) + (0, -.75cm)$)  {$f_{15}$};
    \node (xor3) [XOR, left of = f3, node distance = 2cm] {};
    \draw[thick,-latex] (f3) -- (xor3);

    \node[draw,thick,minimum width=1cm] (f4) at ($4*(0,-1.5cm) + (0, -.75cm)$)  {$f_{16}$};
    \node (xor4) [XOR, left of = f4, node distance = 2cm] {};
    \draw[thick,-latex] (f4) -- (xor4);

    \draw[thick,latex-latex] (f3.east) -| +(1.5cm,-0.5cm) -- ($(xor3) - (0,1cm)$) -- ($(xor3.north) - (0,1.5cm)$);
    \draw[thick] (xor3.south) -- ($(xor3)+(0,-0.5cm)$) -- ($(f3.east) + (1.5cm,-1cm)$) -- +(0,-0.5cm);

    \draw[thick, densely dotted] ($(f3.east) + (1.5cm,0cm)$) -- +(0cm,0.5cm);
    \draw[thick, densely dotted] (xor3.north) -- +(0cm,0.35cm);

    %% Inputs    
    \node (p0) [draw,thick,above of = f1, minimum width=5cm,minimum height=0.5cm,node distance=1cm] {$IP$}; 
    \node (l0) [above of = xor1,node distance=2cm] {$L$};
    \node (r0) [right of = l0, node distance = 4cm] {$R$};
    \draw[thick,-latex] (l0 |- p0.south) -- (xor1.north);
    \draw[thick] ($(f1.east)+(1.5cm,0)$) -- +(0,0.75cm);

    \draw[thick,latex-] (l0 |- p0.north) -- (l0);
    \draw[thick,latex-] (r0 |- p0.north) -- (r0);

    %% Outputs
    \node (p4) [draw,thick,below of = f4, minimum width=5cm,minimum height=0.5cm,node distance=1cm] {$IP^{-1}$}; 
    \node (l4) [below of = xor4,node distance=2cm] {$L'$};
    \node (r4) [right of = l4, node distance = 4cm] {$R'$};
    \draw[thick,latex-latex] (f4.east) -| +(1.5cm,-0.75cm);
    \draw[thick,-latex] (xor4.south) -- ($(xor4)+(0,-0.75cm)$);

    \draw[thick,-latex] (l4 |- p4.south) -- (l4);
    \draw[thick,-latex] (r4 |- p4.south) -- (r4);

\end{tikzpicture}
}

\newcommand\desf{\begin{tikzpicture}[scale=1]

	%% Subkey XORs
    \foreach \z in {0,...,47} {
        \node[XOR, scale=0.8] (xor\z) at ($\z*(0.8em, 0)$) {};
    }
    \node[scale=0.6] (xor-2) at ($-2*(0.8em, 0)$) {};
    \node[scale=0.6] (xor-1) at ($-1*(0.8em, 0)$) {};
    \node[scale=0.6] (xor48) at ($48*(0.8em, 0)$) {};
    \node[scale=0.6] (xor49) at ($49*(0.8em, 0)$) {};

    %% Nodes positions
    \foreach \z in {0,...,47} {
        \node (i\z) [above = 0.75em of xor\z] {};
        \node (ii\z) [above = 2em of i\z] {};
        \node (o\z) [below = 2.5em of xor\z] {};
        \node (oo\z) [below = 5em of o\z] {};
        \draw[thick] (i\z.center) -- (xor\z) -- (o\z);
	}
    

    \node (i-2) [above = 0.75em of xor-2] {};
    \node (ii-2) [above = 2em of i-2] {};
    \node (o-2) [below = 2.5em of xor-2] {};

    \node (i-1) [above = 0.75em of xor-1] {};
    \node (ii-1) [above = 2em of i-1] {};
    \node (o-1) [below = 2.5em of xor-1] {};
    	
    \node (i48) [above = 0.75em of xor48] {};
    \node (ii48) [above = 2em of i48] {};
    \node (o48) [below = 2.5em of xor48] {};
    	
    \node (i49) [above = 0.75em of xor49] {};
    \node (ii49) [above = 2em of i49] {};
    \node (o49) [below = 2.5em of xor49] {};
    

	\draw[thick,densely dashed] ($(ii-2.south)-(0,0.25em)$) -- ($(i0.north)+(0,0.25em)$) -- (i0.center);
	\draw[thick,densely dashed] ($(ii1.south)-(0,0.25em)$) -- ($(i-1.north)+(0,0.25em)$) -- (i-1.center);
	
	\draw[thick] (ii1.center) -- (i1.center);
	\draw[thick] (ii2.center) -- (i2.center);
	\draw[thick] (ii3.center) -- (i3.center);
	\draw[thick] (ii4.center) -- (i4.center);

	\draw[thick] ($(ii4.south)-(0,0.25em)$) -- ($(i6.north)+(0,0.25em)$) -- (i6.center);
	\draw[thick] ($(ii7.south)-(0,0.25em)$) -- ($(i5.north)+(0,0.25em)$) -- (i5.center);
	
	\draw[thick] (ii7.center) -- (i7.center);
	\draw[thick] (ii8.center) -- (i8.center);
	\draw[thick] (ii9.center) -- (i9.center);
	\draw[thick] (ii10.center) -- (i10.center);

	\draw[thick] ($(ii10.south)-(0,0.25em)$) -- ($(i12.north)+(0,0.25em)$) -- (i12.center);
	\draw[thick] ($(ii13.south)-(0,0.25em)$) -- ($(i11.north)+(0,0.25em)$) -- (i11.center);

	\draw[thick] (ii13.center) -- (i13.center);
	\draw[thick] (ii14.center) -- (i14.center);
	\draw[thick] (ii15.center) -- (i15.center);
	\draw[thick] (ii16.center) -- (i16.center);

	\draw[thick] ($(ii16.south)-(0,0.25em)$) -- ($(i18.north)+(0,0.25em)$) -- (i18.center);
	\draw[thick] ($(ii19.south)-(0,0.25em)$) -- ($(i17.north)+(0,0.25em)$) -- (i17.center);

	\draw[thick] (ii19.center) -- (i19.center);
	\draw[thick] (ii20.center) -- (i20.center);
	\draw[thick] (ii21.center) -- (i21.center);
	\draw[thick] (ii22.center) -- (i22.center);

	\draw[thick] ($(ii22.south)-(0,0.25em)$) -- ($(i24.north)+(0,0.25em)$) -- (i24.center);
	\draw[thick] ($(ii25.south)-(0,0.25em)$) -- ($(i23.north)+(0,0.25em)$) -- (i23.center);

	\draw[thick] (ii25.center) -- (i25.center);
	\draw[thick] (ii26.center) -- (i26.center);
	\draw[thick] (ii27.center) -- (i27.center);
	\draw[thick] (ii28.center) -- (i28.center);

	\draw[thick] ($(ii28.south)-(0,0.25em)$) -- ($(i30.north)+(0,0.25em)$) -- (i30.center);
	\draw[thick] ($(ii31.south)-(0,0.25em)$) -- ($(i29.north)+(0,0.25em)$) -- (i29.center);

	\draw[thick] (ii31.center) -- (i31.center);
	\draw[thick] (ii32.center) -- (i32.center);
	\draw[thick] (ii33.center) -- (i33.center);
	\draw[thick] (ii34.center) -- (i34.center);

	\draw[thick] ($(ii34.south)-(0,0.25em)$) -- ($(i36.north)+(0,0.25em)$) -- (i36.center);
	\draw[thick] ($(ii37.south)-(0,0.25em)$) -- ($(i35.north)+(0,0.25em)$) -- (i35.center);

	\draw[thick] (ii37.center) -- (i37.center);
	\draw[thick] (ii38.center) -- (i38.center);
	\draw[thick] (ii39.center) -- (i39.center);
	\draw[thick] (ii40.center) -- (i40.center);

	\draw[thick] ($(ii40.south)-(0,0.25em)$) -- ($(i42.north)+(0,0.25em)$) -- (i42.center);
	\draw[thick] ($(ii43.south)-(0,0.25em)$) -- ($(i41.north)+(0,0.25em)$) -- (i41.center);
	
	\draw[thick] (ii43.center) -- (i43.center);
	\draw[thick] (ii44.center) -- (i44.center);
	\draw[thick] (ii45.center) -- (i45.center);
	\draw[thick] (ii46.center) -- (i46.center);

	\draw[thick,densely dashed] ($(ii46.south)-(0,0.25em)$) -- ($(i48.north)+(0,0.25em)$);
	\draw[thick,densely dashed] ($(ii49.south)-(0,0.25em)$) -- ($(i47.north)+(0,0.25em)$) -- (i47.center);
    
	
	%%%%% To Permutation P
	\draw[thick,->] (o1.north) -- (oo1.center);
	\draw[thick,->] (o2.north) -- (oo2.center);
	\draw[thick,->] (o3.north) -- (oo3.center);
	\draw[thick,->] (o4.north) -- (oo4.center);

	\draw[thick,->] (o7.north) -- (oo7.center);
	\draw[thick,->] (o8.north) -- (oo8.center);
	\draw[thick,->] (o9.north) -- (oo9.center);
	\draw[thick,->] (o10.north) -- (oo10.center);

	\draw[thick,->] (o13.north) -- (oo13.center);
	\draw[thick,->] (o14.north) -- (oo14.center);
	\draw[thick,->] (o15.north) -- (oo15.center);
	\draw[thick,->] (o16.north) -- (oo16.center);

	\draw[thick,->] (o19.north) -- (oo19.center);
	\draw[thick,->] (o20.north) -- (oo20.center);
	\draw[thick,->] (o21.north) -- (oo21.center);
	\draw[thick,->] (o22.north) -- (oo22.center);

	\draw[thick,->] (o25.north) -- (oo25.center);
	\draw[thick,->] (o26.north) -- (oo26.center);
	\draw[thick,->] (o27.north) -- (oo27.center);
	\draw[thick,->] (o28.north) -- (oo28.center);

	\draw[thick,->] (o31.north) -- (oo31.center);
	\draw[thick,->] (o32.north) -- (oo32.center);
	\draw[thick,->] (o33.north) -- (oo33.center);
	\draw[thick,->] (o34.north) -- (oo34.center);

	\draw[thick,->] (o37.north) -- (oo37.center);
	\draw[thick,->] (o38.north) -- (oo38.center);
	\draw[thick,->] (o39.north) -- (oo39.center);
	\draw[thick,->] (o40.north) -- (oo40.center);


	\draw[thick,->] (o43.north) -- (oo43.center);
	\draw[thick,->] (o44.north) -- (oo44.center);
	\draw[thick,->] (o45.north) -- (oo45.center);
	\draw[thick,->] (o46.north) -- (oo46.center);



	%% SBoxes
    \foreach \z in {0,...,7} {
    		\node[draw,thick,minimum width=4.5em,minimum height=2em,fill=white!25] (s\z) at ($\z*(4.8em,0) + (2em,-2em)$) {$S_\z$};
	}
	
    \node[draw,thick,minimum width=38em,minimum height=3.5em,fill=white!25,below = 4em of s0.west,anchor=west] (p) {$P\ (8 \times 4\ bits)$};

\end{tikzpicture}
}


\begin{document}
\tableofcontents

\pagebreak
\include{basics}

\pagebreak
\section{Private Key Cryptography: Block Ciphers}
\begin{definition}
    A \textit{product cipher} combines two or more components with the intention of producing a cipher that is more secure than the basic parts of which it is made.
\end{definition}
\begin{example}
    Let $e_1(x) = x \oplus k_1$ and $e_2(x) = \pi(x)$,
    a product cipher of $e_1$ and $e_2$ is $e_1(e_2(x))$, or $e_2(e_1(x))$.
\end{example}

The product cipher combines a sequence of simple
transformations such as substitution (S-box),
permutation (P-box),
and modular arithmetic.
Shannon suggested using a combination of S-box and P-box transformation,
the combination could yield a cipher system more powerful than either one alone.
A product cipher that uses only substitutions and permutations is called a SP-network.

\begin{definition}
    A \textit{iterated cipher} is a block cipher is made up from:
    \begin{itemize}
        \item A key schedule of $Nr$ round keys: $(k_1, k_2, \dots, k_{Nr})$, derived, using a fixed public algorithm, from the key $k$.
        \item A round function $f$ which takes a round key and a current state.
    \end{itemize}
    The encryption (and decryption) functions consist of repetition of the round function $Nr$ times.
\end{definition}

\subsection{Substitution-permutation networks (SPN)}
\begin{definition}
    A \textit{substitution–permutation network}
    takes a block of the plaintext and the key as inputs,
    and applies several alternating "rounds" or "layers" of
    substitution boxes (S-boxes) and permutation boxes (P-boxes)
    to produce the ciphertext block.\\

    Decryption is done by simply reversing the process
    (using the inverses of the S-boxes and P-boxes and applying the round keys in reversed order).
\end{definition}

A good S-box will have the property that changing one input bit will change about half of the output bits (or an avalanche effect).
It will also have the property that each output bit will depend on every input bit.
A good P-box has the property that the output bits of any S-box are distributed to as many S-box inputs as possible.

At each round, the round key (obtained from the key with some simple operations, for instance, using S-boxes and P-boxes)
is combined using some group operation, typically XOR.

A well-designed SP network with several alternating rounds of S and P-boxes already satisfies Shannon's confusion and diffusion properties.

\begin{example}
    Let $l,m,Nr$ be positive integers.
    \begin{itemize}
        \item $\pi_S : \{0,1\}^l \rightarrow \{0,1\}^l$ is an S-box
        \item $\pi_P : \{1,...,lm\} \rightarrow {1,...,lm}$ is a permutation
        \item $P=C=\{0,1\}^{lm}$
        \item $K \subseteq (\{0,1\}^{lm})^{Nr+1}$ is the set of all key schedules that can be derived from an initial key $k$
        \item For a key schedule $(k^1,...,k^{Nr+1})$ we encrypt using an iterated cipher composed of substitution, permutation and x-or operations.
    \end{itemize}
\end{example}

\subsection{Feistel Network}
\begin{definition}
    A \textit{feistel cipher} is an iterated cipher
    in which the state on round $i$ is divided into two halves of equal length: $L_i$ and $R_i$.
    The round function $g$ has the form $g(L_{i−1}, R_{i−1}, k_i) \rightarrow (L_i, R_i)$ and is computed:
    \begin{align*}
        L_i &= R_{i−1} \\
        R_i &= L_{i−1} \oplus f(R_{i-1},k_i)
    \end{align*}
    for some function f.
\end{definition}
\begin{center}\feistel\end{center}
The Feistel structure has the advantage that encryption and decryption operations are very similar,
even identical in some cases, requiring only a reversal of the key schedule.
Therefore, the size of the code or circuitry required to implement such a cipher is nearly halved.
Although a Feistel network that uses S-boxes (such as DES) is quite similar to SP networks,
there are some differences that make either this or that more applicable in certain situations.
For a given amount of confusion and diffusion,
an SP network has more "inherent parallelism" and so — given a CPU with a large number of execution units — can be computed faster than a Feistel network.
CPUs with few execution units — such as most smart cards — cannot take advantage of this inherent parallelism.
Also SP ciphers require S-boxes to be invertible (to perform decryption);
Feistel inner functions have no such restriction and can be constructed as one-way functions.

\subsection{The Data Encryption Standard (DES)}
\begin{center}\des\end{center}
\begin{definition}

DES is a 16 round Feistel cipher where:
• m = 64, Li and Ri are bitstrings of length 32.
Ciphertext
• k is 56 bits long, from this sixteen 48 bit round keys are produced consisting of a selection of bits of k, permuted.
• There is a fixed initial permutation L0R0 = IP(x) before the first round. 20
• The inverse permutation IP−1(R16L16) is applied after the last round. • f :{0,1}32 ×{0,1}48 →{0,1}32.
• f consists of a substitution (S-box) followed by a fixed permutation.
DES f function
• Expand Ri−1 to 48 bits and x-or with ki: state = E(Ri−1) \oplus ki
• Apply substitutions to state: map 6-bit substrings to 4-bit substrings • Permute state: state = P (state)

DES f function
• Expand Ri−1 to 48 bits and x-or with ki: state = E(Ri−1) \oplus ki
• Apply substitutions to state: map 6-bit substrings to 4-bit substrings • Permute state: state = P (state)
\end{definition}

\textbf{Security of DES}

\subsubsection{Triple DES (3DES)}
Currently DES is insecure, however triple DES is still secure.
Simply encrypt via DES three times. Yup that simple

\subsection{Modes of Operation}
Electronic codebook mode (ECB)
Cipher block chaining mode (CBC)
Output feedback mode (OFB)
Cipher feedback mode (CFB)

\subsection{Advanced Encryption Standard (AES)}
Galois Fields

\subsection{Attacks on substitution-permutation networks}
\textbf{Evaluating security of block ciphers}
\begin{itemize}
    \item Key size
    \item Block size
    \item Estimated security level
\end{itemize}
\textbf{Computational security against exhaustive key search}
• A fixed key size defines an upper bound on the security of the cipher.
• If the key k is a bitstring of length lk then there are 2lk keys.
• Given a small number of plaintext-ciphertext pairs the attack has complexity of 2lk−1 operations.
\textbf{Computational security against exhaustive data analysis}
• Text dictionary attack: For block size m, if an attacker has collected 2m plaintext-
ciphertext pairs then they have a complete dictionary of the cipher.
• Matching ciphertext attack: If you have collected 2m/2 blocks you expect to find matching ciphertext blocks within them.
\textbf{Known attacks against private-key block ciphers}
In the next lecture we will consider
• differential cryptanalysis, a chosen plaintext attack and
• linear cryptanalysis, a known plaintext attack.
These are attacks on SPNs which are also relevant to the design of DES and AES.

\subsection{Differential Cryptanalysis}
Objective
Find targeted key bits.
Basic idea
Exploit a situation whereby a particular difference between ciphertexts $y\prime$ occurs, given a par- ticular difference between plaintexts $x\prime$, with a very much higher probability than is ideal.
\subsection{Linear Cryptanalysis}
A related attack is linear cryptanalysis
• Objective: Find targeted key bits.
• Known plaintext attack: attacker has a set of plaintext-ciphertext pairs encrypted with
the same key k.
• Using probabilistic analysis we find biased linear approximations for the S-boxes. We construct a linear approximation, with a large bias, of the SPN (excepting the final round) in terms of plaintext bits and state bits.
• For each candidate key we partially decrypt each ciphertext and see if the linear approx- imation holds for state, incrementing a counter for the key if it does.
• The candidate key with largest bias (from |input pairs|/2) should contain the targeted key bits.


\pagebreak
\section{Cryptographic Hashes}

\textit{Hash functions} are functions which
map data of arbitrary size to a bit string of a fixed size,
the output being called a \textit{hash}.\\

To qualify as hash functions these functions must have several qualities;
firstly the output must always be the same size.
Secondly the function must be easy to compute;
i.e. computable algorithmically in polynomial time
with no massive hidden constants.\\

\begin{definition}
    More formally we have
    \begin{itemize}
        \item $\X$ is the set of messages.
        \item $\Y$ is the finite set of hashes,
            where each hash is of fixed size $n$ such that
            $\forall\ y \in \Y\ |y| = n$.
        \item $h : \X \rightarrow \Y$ is the hash function.
        \item A pair $(x,y) \in (\X,\Y)$ is valid iff $h(x) = y$.
    \end{itemize}
\end{definition}

\begin{definition}
    A \textit{keyed hash function} is the tuple $(\X,\Y,\K,\Hsh)$ where
    \begin{enumerate}
        \item $\X$ is the set of messages
        \item $\Y$ is the finite set of \textbf{authentication tags}
        \item $\forall\ k \in \K$ there exists $h_k : \X \rightarrow \Y \in \Hsh$
            such that a pair $(x,y) \in (\X, \Y)$ is considered valid iff $h_k(x) = y$.
    \end{enumerate}
\end{definition}

\begin{definition}
    \textit{Cryptographic hash functions} are a subset of hash functions
    for which given $h(x) = y$ it is easy to verify that $h(x)$ is $y$
    but given just $y$ it is not easy to find $x$.\\

    The ideal cryptographic hash function has five main properties:
    \begin{enumerate}
        \item It is \textbf{deterministic} so the same message always results in the same hash
        \item It is \textbf{quick to compute} the hash value for any given message
        \item It is infeasible to generate a message from its hash value except by trying all possible messages,
            i.e. \textbf{one-way}.
        \item A small change to a message should change the hash value so extensively that the new hash value appears uncorrelated with the old hash value,
            i.e. \textbf{diffusion}.
        \item It is infeasible to find two different messages with the same hash value,
            i.e. \textbf{no hash collisions}.
    \end{enumerate}
\end{definition}

A hash function is considered to be \textbf{secure} if three problems are difficult to solve.
\begin{itemize}
    \item \textit{Preimage 1}
        Given $h : \X \rightarrow \Y$
        and $y \in Y$
        find $x \in X$
        such that $h(x) = y$.
    \item \textit{Preimage 2}
        Given $h : \X \rightarrow \Y$
        and $x \in \X$
        find $x\prime \in \X\ x \neq x\prime$
        such that $h(x) = h(x\prime)$.
    \item \textit{Collision}
        Given only $h : \X \rightarrow \Y$
        find $x, x\prime \in \X$
        such that $x \neq x\prime, h(x) = h(x\prime)$.
\end{itemize}

\textbf{Na\"ive attack}\\
\begin{itemize}
    \item $h$ produces hashes of length $n$ bits
    \item Let $y = h(x)$ for some message $x$
    \item For random bitstrings $x\prime$ of bounded bitlength,
        calculate $h(x\prime)$ and check if $h(x\prime) = y$
\end{itemize}

\textbf{Birthday Attack problem}\\
In a group of 23 randomly chosen people,
at least two will share a birthday with probability at least $0.5$.
How is this relevant anyway?
Define $h$ : humans $\rightarrow$ days of year by setting $h(x)$ equal to the birth day of person $x$.
Finding two people with the same birthday is the same as finding a collision for this hash.

What is the probability that k people don’t share a birthday?
$$
    p = 1 \times \frac{364}{365} \times \frac{363}{365} \times \dots \times \frac{365 - (k - 1)}{365}
$$
Probability that there is at least one shared birthday amongst k people is 1 − p
Let h : X → Y produce hashes y of length n bits and hence |Y| = 2n.
We expect to find a collision by brute force in around 2n/2 operations.

Summarily, a hash function is secure if
\begin{itemize}
    \item a preimage attack requires $2^n$ operations and
    \item a collision attack requires attack $2^{n/2}$ operations.
\end{itemize}


\subsection{Merkle-Damg\aa rd Construction}
The Merkle-Damg\aa rd construction is 
a method of building collision-resistant cryptographic hash functions 
from collision-resistant one-way compression functions.
A one-way compression function is a function
that transforms two fixed-length inputs into a fixed-length output
A hash function must be able to process an arbitrary-length message into a fixed-length output.
This can be achieved by breaking the input up into a series of equal-sized blocks,
and operating on them in sequence using a one-way compression function.
The Merkle–Damgård hash function first applies an MD-compliant padding function to create an input whose size is a multiple of a fixed number (e.g. 512 or 1024) — this is because compression functions cannot handle inputs of arbitrary size. The hash function then breaks the result into blocks of fixed size, and processes them one at a time with the compression function, each time combining a block of the input with the output of the previous round.

In order to make the construction secure, Merkle and Damgård proposed that messages be padded with a padding that encodes the length of the original message. This is called length padding or Merkle–Damgård strengthening.
The algorithm starts with an initial value, the initialization vector (IV). The IV is a fixed value (algorithm or implementation specific). For each message block, the compression (or compacting) function f takes the result so far, combines it with the message block, and produces an intermediate result. The last block is padded with zeros as needed and bits representing the length of the entire message are appended.

the padding scheme used in the Merkle–Damgård construction must be chosen carefully to ensure the security of the scheme
With these conditions in place, we find a collision in the MD hash function exactly when we find a collision in the underlying compression function. Therefore, the Merkle–Damgård construction is provably secure when the underlying compression function is secure

Merkle-Damg ̊ard (MD) Strengthening
Definition 17.
Let x = x1x2 . . . xr of bitlength b be a message which has been broken into r blocks of length t. MD strengthening is defined to be the addition of a length-block xr+1 containing, say, the binary representation of b (b < 2t).

\subsection{Message Digest (MD) Hash functions}

\textbf{MD5}\\
MD5 is no longer considered to be secure:
Preprocessing
Let x be a b bit message. Padding (always performed):
• Pad x so that it has a bitlength congruent to 448 (mod 512) • 1 bit is appended then 0 bits to the correct length
Append length:
• Append a 64 bit representation of b
• If b > 264 the low-order 64 bits of b are appended Initialisation:
• A four word state (A, B, C, D) is initialised with fixed public values • TableT: Ti =abs(sini)×232,1≤i≤64,iinradians

Auxiliary functions
Define four auxiliary functions that each take as input three 32-bit words and produce as output one 32-bit word:
F(X,Y,Z) = (X∧Y)∨(¬X∧Z) G(X,Y,Z) = (X∧Z)∨(Y∧¬Z)
H(X,Y,Z) = X⊕Y⊕Z I(X,Y,Z) = Y⊕(X∨¬Z)

Iteration and output
Let y be the padded input.
Iteration of MD5 compression function:
for each 512 bit block of y do
Modify the 128 bit state (A, B, C, D):
Round inputs: current block, state and table T.
Four similar rounds each of 16 operations based on one of the non-linear functions F, G, H or I, modular arithmetic and left rotation.
end do
Output h(x) = A∥B∥C∥D


\subsection{Secure Hash Algorithm (SHA)}
\textbf{SHA-1}
Preprocessing
Letxbeab<264 bitmessage. Padding:
• Identical to MD5 Append length:
• Append a 64 bit representation of b Initialisation:
• A five word state (A, B, C, D, E) is initialised with fixed public values • A sequence T of 80 elements is initialised with fixed public values
Auxiliary functions
Define four auxiliary functions that each take as input three 32-bit words and produce as output one 32-bit word:
F(X,Y,Z) = (X∧Y)∨(¬X∧Z) G(X,Y,Z) = X⊕Y⊕Z
H(X,Y,Z) = (X ∧Y)∨(X ∧Z)∨(Y ∧Z) I(X,Y,Z) = X⊕Y⊕Z
Iteration and output
Let y be the padded input.
Iteration of SHA-1 compression function:
for each 512 bit block of y do:
Modify the 160 bit state (A, B, C, D, E):
Round inputs: current block, state and sequence T.
Using the current block define 80 32-bit words using ⊕ and left rotations.
Four similar rounds each of 20 operations based on one of the non-linear functions F, G, H or I, modular arithmetic and left rotation.
end do
Output h(x) = A∥B∥C∥D∥E

\textbf{SHA-2 Family}

\textbf{SHA-3}

\textbf{Using unkeyed hash functions}\\

\subsection{Message Digest Codes {MDC}}
A message digest code is an unkeyed hash function.
Let compress : {0, 1}m+t → {0, 1}m be a compression function (t ≥ 1). We construct an iterated hash function:
h: \bigcup ∞ {0,1}i →{0,1}n i=m+t+1

Let g : {0, 1}m → {0, 1}n be a (public) optional transformation function. Steps:
1. Preprocessing
2. Iteration
3. Optional transformation

\subsection{Message Authentication Codes {MAC}}
\begin{definition}
    A family of keyed hash functions is a four-tuple $(X,Y,K,H)$ where:
    1. $X$ is the set of possible messages
    2. $Y$ is a finite set of possible authentication tags
    3. $K$ is the keyspace, a finite set of possible keys
    4. For each $k\inK$, thereisahashfunctionhk $\in H$
    $$hk : X \rightarrow Y$$
    A pair $(x,y)$, is valid under key $k$ if $h_k(x) = y$
\end{definition}

\textbf{What do we mean by secure?}
Computation-resistance:
without prior knowledge of k, given zero or more pairs
$(x_i,h_k(x_i))$ it is computationally infeasible to compute $(x,h_k(x))$ for a new input $x \neq x_i$

MAC forgery: severity of attack
The attacker is able to produce a new valid pair $(x,h_k(x))$
• for his choice (or partial choice) of x
• but has no control over the value of x
Key recovery by exhaustive search
Input: (x,hk(x)), key of length l Output: k
Compute n-bit MAC using all possible keys
• This requires 2l MAC operations.
• In an ideal MAC, 2l−n keys will remain as candidates • Further valid pairs can test the remaining candidates
A MAC key should not be recoverable in fewer than 2l operations. Cost of guessing a MAC
For an n-bit MAC we expect to guess a correct MAC with probability 2−n.
But we cannot check guesses without either the key or an (adaptive) chosen-text scenario.
We should not be able to produce MAC forgeries with probability higher than
max(2−l, 2−n)
How to create a MAC?
I have a cunning plan...
Why not recycle?
Lets construct a MAC as follows:
• Take your favourite secure unkeyed iterated hash function h which uses compress: {0, 1}m+t → {0, 1}m as the compression function
• For simplicity assume there is no preprocessing step and no output transformation
• Hence every message x will need to have length a multiple t
• Let hk be created by setting IV = k and keeping the IV secret.
...but Oscar knows better
Suppose Oscar has a valid pair (x,hk(x)) and a message extension x′ of length t.
• Oscar can calculate a valid pair for the extended message x∥x′ without knowledge of k
Problem solved?
We try to stop this forgery by re-introducing the preprocessing step (padding): Oscar has valid pair (x,hk(x)). The preprocessing step for x produces:
y = x∥pad(x), |y| = rt for some r ∈ Z Let w be of length t and define
x′ = y∥w = x∥pad(x)∥w The preprocessing step for x′ produces:
where |y′| = r′t, r′ > r
y′ = x′∥pad(x′) = x∥pad(x)∥w∥pad(x′) 38
• Oscar wishes to find hk(x′) without knowledge of k. He has hk(x) which would be the current state at round r in this computation. He computes:
compress(hk(x)∥yr′ +1)
compress(zr+1∥yr′ +2) . .
Perhaps we could set
This will fail for the same reasons as above! Or we could set
hk(x) = h(x∥k) • Susceptible to a birthday attack
• MAC value depends only on the last chaining value, key is only used in one step
The above examples should provide some motivation for the ideas in the following section. I
also formally rename this section “How not to create a MAC”.
How to create a MAC (for real this time)
HMAC
Suffix and postfix key
The above attacks suggest that the MAC key should be used as both a suffix and a postfix: hk(x) = h(k∥p∥x∥k)
with padding to ensure that there is at least two iterations in the computation of h. HMAC
Algorithm 7.
Inputs: key k, MDC h
Define ipad and opad each of length 512 bits: ipad = 3636...36
opad = 5C5C ...5C
HMACk (x) = h􏰇 (k ⊕ opad􏰈 􏰎􏰎 h((k ⊕ ipad)∥x) 􏰈
• Keyed-Hash Message Authentication Code, FIPS standard 198, 2002
• h can be any approved unkeyed hash, examples in the standard use SHA-1. • Argument for security is given in course book
hk(x′) = zr′ =
What about incorporating the key in other ways?
zr+1 = zr+2 =
compress(zr′−1∥yr′ ′ ) hk(x) = h(k∥x)
CBC-MAC


\pagebreak
\section{RSA Cryptosystem}
\textit{Public key cryptography}, or asymmetrical cryptography,
is any cryptographic system that uses pairs of keys: public keys which may be disseminated widely,
and private keys which are known only to the owner.
The essential advantage of public key cryptography is that it provides secure communication between two people who had not met or exchanged securely a secret key.
This accomplishes two functions: authentication, where the public key verifies that a holder of the paired private key sent the message,
and encryption, where only the paired private key holder can decrypt the message encrypted with the public key.

\textbf{1976: Diffie and Hellman}
“We stand today on the brink of a revolution in cryptography.”
\begin{itemize}
    \item Gave an abstract way of providing secure communication between two people who had not met or exchanged securely a secret key.
    \item Argued how such a system could also provide secure digital signatures.
    \item Gave a practical method by which two people, without the aid of a trusted authority, can
        establish a shared secret key using an insecure channel.
\end{itemize}

\begin{definition}
    The essence of a \textit{public key cryptosystem} is defined as the tuple
    $(P,C,K,E,D)$
    \begin{itemize}
        \item For every key $k \in K$, $e_k$ is the inverse of $d_k$
        \item For every key $k \in K$,
            and for every $x \in P$ and $y \in C$
            $e_k(x)$ and $d_k(y)$ are easy to compute
        \item It is computationally infeasible for almost all $k \in K$
            to derive $d_k$ from $e_k$.
        \item For every $k \in K$ it is feasible to derive $e_K$ and $d_k$
    \end{itemize}
    We have $e_k$ as the public key which allows \textbf{anyone} to encrypt a message
    which only the holder of $d_k$ will be able to decrypt.
\end{definition}

Clearly a public-key cryptosystem can never be unconditionally secure.
\begin{itemize}
    \item Alice looks up Bob’s public key function $e_k$ and encrypts $x : y = e_k(x)$.
    \item Oscar encrypts each possible message in turn until he finds the unique $x$ such that $y = e_k(x)$.
\end{itemize}
Note that Oscar can always launch a chosen-plaintext attack.

\begin{definition}
    A \textit{one-way function} is a function $f : X \rightarrow Y$ such that 
    for all $x \in X$ it is easy to compute $f(x)$ but for (almost) all $y \in Y$ it is computationally
    infeasible to find an $x$ where $f(x) = y$.
\end{definition}
\begin{definition}
    A \textit{trap-door one-way function} is a one way function $f$ such that
    given some additional trap-door information it becomes feasible, 
    for all $y \in Y$ to find $x \in X$ such that $y = f(x)$.
\end{definition}

Given a public-key cryptosystem in which $P = C$
using the fact that $e_k$ and $d_k$ are inverses
we can define a secure digital signature scheme as such
we might define a mechanism to allow secure digital signatures:
\begin{itemize}
    \item
        Alice wishes to sign message $x$.
        Using her private decryption function she obtains $y$ where $y = d_k(x)$;
        sending $y$ to Bob.
    \item Bob can then encrypt $y$ with Alice’s public encryption function: $e_k(y) = x$
    \item Only Alice could have computed $y$ such that $e_k(y) = x$ hence Bob is convinced that
    Alice signed the message.
    Furthermore anyone could have checked Alice’s signature, not just Bob.
\end{itemize}

\subsection{Mathematical background}
\textbf{The Euler function}\\
\begin{definition}
    The \textit{Euler function} $\phi(x)$ is
\end{definition}
It has the following properties
\begin{itemize}
    \item If $p$ is prime then $\phi(p)$ is $p - 1$
    \item $\phi$ is multiplicative; that is,
        if $gcd(m,n) = 1$ then $\phi(m \cdot n) = \phi(m) \cdot \phi(n)$
    \item If $\phi(n) \geq 2$ then $\phi(n)$ is even.
\end{itemize}

\textbf{Fermat's Little Theorem}\\
\begin{definition}
    \textit{Fermat's little theorem} states that if $p$ is a prime number,
    then for any integer $a$ the number $a^p - a$ is an integer multiple of $p$.
    In the notation of modular arithmetic this is expressed as
    $$ a^p \equiv a \Mod p $$
\end{definition}
For example, if $a = 2$ and $p = 7$, then $2^7 = 128$, and 128 - 2 = 126 = 7 $\times$ 18.

\textbf{Chinese Remainder Theorem}\\
\begin{definition}
    \textit{Chinese Remainder Theorem} is a way of solving certain systems of congruences.
    %%Letm1,m2,...,mr bepairwiserelativelyprimepositiveintegers. Supposea1,a2,...,ar ∈Zandconsider:
    $$
    x \equiv a_1 (\Mod m_1) \\
    x \equiv a_2 (\Mod m_2) \\
    x \equiv a_r (\Mod m_r)
    $$
    %%The Chinese remainder theorem states that this system has a unique solution modulo M = m1m2 ···mr
\end{definition}

\textbf{The Euclidean Algorithm}\\

\textbf{The Extended Euclidean Algorithm}\\

\subsection{RSA Cryptosystem}
The Rivest-Shamir-Adleman cryptosystem,
named after its authors, was one of the first public key cryptosystems.
RSA is a relatively slow algorithm, and because of this,
it is less commonly used to directly encrypt user data.
More often, RSA passes encrypted shared keys for symmetric key cryptography
which in turn can perform bulk encryption-decryption operations at much higher speed.

\begin{definition}
    The \textit{RSA cryptosystem} is defined as such
    \begin{itemize}
        \item Let $p$ and $q$ be large primes, (in some implementations 1024 bits each)
        \item Let $n = p \cdot q$
        \item Let $P = C = \Z_n$
        \item $K = \{(n,p,q,a,b) | ba \equiv 1 \Mod{\phi(n)}\}$
        \item $(n,a)$ is the public key
        \item $(n,b)$ is the public key
        \item For $K = (n,p,q,a,b)$ we have
            $$e_k(x) = x^b \Mod n = y$$
            $$d_k(y) = y^a \Mod n = x$$
    \end{itemize}
\end{definition}

\subsection{RSA Toy Example}
Firstly Bob chooses $p = 127$ and $q = 131$ and calculates $n$ and $\phi(n)$
\begin{align*}
    n &= 127  \times 131  &\phi(n) &= (p \mns 1) \times (q \mns 1)\\
      &= 16637            &        &= 126        \times 130       \\
      &                   &        &= 16380
\end{align*}
Bob selects some $b$ such that $gcd(b, \phi(n)) = 1$;
we shall use $b = 4057$ and calculate $a$
\begin{align*}
    1      &= ab \Mod{\phi(n)}\\
    b^{-1} &= a \Mod{\phi(n)}\\
           &= 10453 \Mod{\phi(n)}\\
         a &= 10453
\end{align*}
Thus Bob has the public key $(n,b) = (16637,4057)$ and the private key $(p,q,a) = (127, 131, 10453)$.
Alice has Bob’s public key and she wishes to encrypt the plaintext $x = 9031$.
She computes:
\begin{align*}
   y &= x^b \Mod{n} \\
     &= 90314057 \Mod{16637} \\
     &= 1870
\end{align*}
So Alice sends $y = 1870$ over the insecure channel. Bob now uses his decryption exponent a to compute:
\begin{align*}
   x &= y^a \Mod{n} \\
     &= 187010453 \Mod{16637} \\
     &= 9031
\end{align*}

\subsection{RSA Implementation Concerns}
To be ready to use RSA we have to
\begin{enumerate}
    \item Generate two large primes $p, q$ such that $p \neq q$
    \item Calculate $n = p \cdot q$ and $\phi(n) = (p \mns 1) (q \mns 1)$
    \item Choose a random $1 < b < \phi(n)$ such that $gcd(b, \phi(n)) = 1$
    \item Calculate $a = b^{-1} \Mod{\phi(n)}$
    \item Set the public key to $(n, b)$ and the private key to $(p, q, a)$
\end{enumerate}
For RSA to be a practical algorithm we need these steps to be computationally feasible.
Firstly the issue of generating large primes
\begin{itemize}
    \item Basic method: generate a large random number, test for primality.
    \item PRIMES is in P: In 2002 Agrawal, Kayal and Saxena proved that
        there is a polynomial time deterministic algorithm for primality testing; but we don’t use this.
    \item Instead we use one of a variety of randomised polynomial time algorithms. The Miller-Rabin algorithm is the most prominent.
\end{itemize}

The other large cost involved in the parameter generation is the calculation of
$a = b^{-1} \Mod{\phi(n)}$
We use extended Euclidean algorithm.
Can be calculated in time $O((log k)^2)$
What about the modular exponentiation in RSA?
To calculate $x^b \Mod{n}$ we can
\begin{itemize}
    \item Very slow: $b \mns 1$ modular multiplications
    \item Faster: use the square and multiply algorithm
    \item Faster again: use Chinese remainder theorem (if you are Bob)
\end{itemize}

\subsection{RSA and Factoring}
Oscar, listening to communications will try to recover $x$ given $x^b$ $\Mod{n}$, $n$ and $b$.
This is called the RSA problem.
There is no efficient algorithm known to solve this problem.
The security of RSA rests on the fact that given $n$
it is hard to find primes $p,q$ such that $n = p \cdot q$.
If factoring is computationally feasible RSA is insecure
as one can just find $p$ and $q$ and then recreate the private key $a$.
A concern is that finding $\phi(n)$ is easier than factoring $n$;
given that $$1 = ab \Mod{\phi(n)}$$
Fortunately it is the case that calculating $\phi(n)$ is as computationally difficult as factoring $n$.

\textbf{Factoring as an attack}
\begin{definition}
    \textit{Splitting}
It suffices to study algorithms which split n, that is, 
    find $1 < a < n, 1 < b < n$ such that $n = ab$.
    b a and b are said to be non-trivial factors of n.
\end{definition}

Trial division

Before using a more expensive method on n we would usually trial divide by “small” primes.
 “small” is considered as a function of the size of n
In the extreme case we trial divide with all primes $p \leq abs(\sqrt{n})$
In the worst case n is a product of two primes of roughly the same size
Perfectly reasonable method if $n < 10^{12}$

There are two categories of factoring algorithm:
Special purpose factoring algorithms
General purpose factoring algorithms

General purpose algorithms include
\begin{itemize}
    \item The quadratic sieve
    \item The general number field sieve
\end{itemize}

Special purpose algorithms include
\begin{itemize}
    \item Pollard’s $p \mns 1$ method
    \item William’s $p + 1$ method
    \item The elliptic curve factoring method
\end{itemize}

Special purpose factoring algorithms and RSA?
Many authors have recommended that p, q be chosen to be strong primes.
\begin{definition}
    $p$ prime is strong if
    \begin{itemize}
        \item $p \mns 1$ has a large prime factor (denoted $r$)
        \item $p + 1$ has a large prime factor
        \item $r \mns 1$ has a large prime factor
    \end{itemize}
\end{definition}

However, strong primes offer no protection against the Elliptic Curve Method

\subsubsection{Elliptic curve method}
\begin{itemize}
    \item The Elliptic curve method (ECM) is a generalisation of Pollard’s $p \mns 1$ method.
    \item Depends on an integer “close to” $p$ having only “small” prime factors.
    \item This is more likely than the situation required by Pollard’s $p \mns 1$ method.
    \item Pollard’s method depends on a relation which holds in the group $\mathbb{Z}^*_p$.
    \item ECM depends on a relation which involves groups defined on elliptic curves modulo p.
    \item Tends to find smaller factors first.
\end{itemize}

\subsubsection{Factoring with Congruent Squares}
To factor $n$ look for $x,y \in \mathbb{Z}$ such that
$$x^2 \mns y^2 = n$$
If such $x, y$ can be found then 
$$d = gcd(x \mns y, n)$$
is a non trivial factor of $n$.

In the 1920’s Kraitchik modified the idea of difference of two squares: Congruent squares.
To factor n look for $x,y \in \mathbb{Z}$ such that
$$x^2 \equiv y^2 \Mod{n}$$

Proposition 4.
If we also have $x\not\equiv \pm y \Mod{n}$ then
$$gcd(x \mns y, n) \quad\textrm{and}\quad gcd(x + y, n)$$
are non trivial factors of $n$.

Given $a$ we may factor $n$.
Imagine Oscar has Bob’s public key $(n, b)$ and his decryption exponent $a$.
Oscar can factor $n$.
By definition $ba \equiv 1 \Modpn$
$$ba - 1 = k\phi(n)\quad\textrm{for some } k \in \mathbb{N}$$
Hence, $\forall x \in \mathbb{Z}_n^*$:
$$x^{ba \mns 1} \equiv \Mod{n}$$
$$y_1 = \sqrt{x^{ba \mns 1}} = x^{(ba\mns1)/2}$$
$$y_1^2 \equiv 1 \Mod{n}$$
We already have that
so we can again attempt to factor n.
Continue process until either
You find a non trivial factor of n
$(ba \mns 1)/2s$ is not divisible by 2
We factor $n$ with probability half.

\begin{example}
    $$n = 437 = 19 \times 23$$
    $$\phi(n) = (19\mns1) \times (23\mns1) = 396$$
    $$b = 7$$
    $$a \equiv b^{\mns1} \equiv 7^{\mns1} \equiv 283 \Modpn$$
    Assume we know $a$
    $$\frac{ba\mns1}{2} = \frac{(7\times283)\mns1}{2} = 990$$
    $$\textrm{Try } x = 2$$
\end{example}

\begin{definition}
    An integer $n \in \mathbb{N}$ is said to be \textit{B-Smooth} if
\end{definition}

\begin{example}
\end{example}
\subsubsection{Dixon’s Random Squares}

\subsection{RSA Security}
\subsubsection{Small Private Exponent}
\subsubsection{Small Public Exponent}
\subsubsection{Coppersmith’s Theorem}
\subsubsection{Hastad’s Broadcast Attack}
\subsubsection{Franklin-Reiter Related Messages}
\subsubsection{Padding}
\subsubsection{Crouch/Davenport}

\subsection{RSA in practice}
\subsubsection{RSA Optimal Asymmetric Encryption Padding}
\subsubsection{Existential Forgery}


\end{document}
