\section{Private Key Cryptography: Block Ciphers}
\begin{definition}
    A \textit{product cipher} combines two or more components with the intention of producing a cipher that is more secure than the basic parts of which it is made.
\end{definition}
\begin{example}
    Let $e_1(x) = x \oplus k_1$ and $e_2(x) = \pi(x)$,
    a product cipher of $e_1$ and $e_2$ is $e_1(e_2(x))$, or $e_2(e_1(x))$.
\end{example}

The product cipher combines a sequence of simple
transformations such as substitution (S-box),
permutation (P-box),
and modular arithmetic.
Shannon suggested using a combination of S-box and P-box transformation,
the combination could yield a cipher system more powerful than either one alone.
A product cipher that uses only substitutions and permutations is called a SP-network.

\begin{definition}
    A \textit{iterated cipher} is a block cipher is made up from:
    \begin{itemize}
        \item A key schedule of $Nr$ round keys: $(k_1, k_2, \dots, k_{Nr})$, derived, using a fixed public algorithm, from the key $k$.
        \item A round function $f$ which takes a round key and a current state.
    \end{itemize}
    The encryption (and decryption) functions consist of repetition of the round function $Nr$ times.
\end{definition}

\subsection{Substitution-permutation networks (SPN)}
\begin{definition}
    A \textit{substitution–permutation network}
    takes a block of the plaintext and the key as inputs,
    and applies several alternating "rounds" or "layers" of
    substitution boxes (S-boxes) and permutation boxes (P-boxes)
    to produce the ciphertext block.\\

    Decryption is done by simply reversing the process
    (using the inverses of the S-boxes and P-boxes and applying the round keys in reversed order).
\end{definition}

A good S-box will have the property that changing one input bit will change about half of the output bits (or an avalanche effect).
It will also have the property that each output bit will depend on every input bit.
A good P-box has the property that the output bits of any S-box are distributed to as many S-box inputs as possible.

At each round, the round key (obtained from the key with some simple operations, for instance, using S-boxes and P-boxes)
is combined using some group operation, typically XOR.

A well-designed SP network with several alternating rounds of S and P-boxes already satisfies Shannon's confusion and diffusion properties.

\begin{example}
    Let $l,m,Nr$ be positive integers.
    \begin{itemize}
        \item $\pi_S : \{0,1\}^l \rightarrow \{0,1\}^l$ is an S-box
        \item $\pi_P : \{1,...,lm\} \rightarrow {1,...,lm}$ is a permutation
        \item $P=C=\{0,1\}^{lm}$
        \item $K \subseteq (\{0,1\}^{lm})^{Nr+1}$ is the set of all key schedules that can be derived from an initial key $k$
        \item For a key schedule $(k^1,...,k^{Nr+1})$ we encrypt using an iterated cipher composed of substitution, permutation and x-or operations.
    \end{itemize}
\end{example}

\subsection{Feistel Network}
\begin{definition}
    A \textit{feistel cipher} is an iterated cipher
    in which the state on round $i$ is divided into two halves of equal length: $L_i$ and $R_i$.
    The round function $g$ has the form $g(L_{i−1}, R_{i−1}, k_i) \rightarrow (L_i, R_i)$ and is computed:
    \begin{align*}
        L_i &= R_{i−1} \\
        R_i &= L_{i−1} \oplus f(R_{i-1},k_i)
    \end{align*}
    for some function f.
\end{definition}
\begin{center}\feistel\end{center}
The Feistel structure has the advantage that encryption and decryption operations are very similar,
even identical in some cases, requiring only a reversal of the key schedule.
Therefore, the size of the code or circuitry required to implement such a cipher is nearly halved.
Although a Feistel network that uses S-boxes (such as DES) is quite similar to SP networks,
there are some differences that make either this or that more applicable in certain situations.
For a given amount of confusion and diffusion,
an SP network has more "inherent parallelism" and so — given a CPU with a large number of execution units — can be computed faster than a Feistel network.
CPUs with few execution units — such as most smart cards — cannot take advantage of this inherent parallelism.
Also SP ciphers require S-boxes to be invertible (to perform decryption);
Feistel inner functions have no such restriction and can be constructed as one-way functions.

\subsection{The Data Encryption Standard (DES)}
\begin{center}\des\end{center}
\begin{definition}

DES is a 16 round Feistel cipher where:
• m = 64, Li and Ri are bitstrings of length 32.
Ciphertext
• k is 56 bits long, from this sixteen 48 bit round keys are produced consisting of a selection of bits of k, permuted.
• There is a fixed initial permutation L0R0 = IP(x) before the first round. 20
• The inverse permutation IP−1(R16L16) is applied after the last round. • f :{0,1}32 ×{0,1}48 →{0,1}32.
• f consists of a substitution (S-box) followed by a fixed permutation.
DES f function
• Expand Ri−1 to 48 bits and x-or with ki: state = E(Ri−1) \oplus ki
• Apply substitutions to state: map 6-bit substrings to 4-bit substrings • Permute state: state = P (state)

DES f function
• Expand Ri−1 to 48 bits and x-or with ki: state = E(Ri−1) \oplus ki
• Apply substitutions to state: map 6-bit substrings to 4-bit substrings • Permute state: state = P (state)
\end{definition}

\textbf{Security of DES}

\subsubsection{Triple DES (3DES)}
Currently DES is insecure, however triple DES is still secure.
Simply encrypt via DES three times. Yup that simple

\subsection{Modes of Operation}
Electronic codebook mode (ECB)
Cipher block chaining mode (CBC)
Output feedback mode (OFB)
Cipher feedback mode (CFB)

\subsection{Advanced Encryption Standard (AES)}
Galois Fields

\subsection{Attacks on substitution-permutation networks}
\textbf{Evaluating security of block ciphers}
\begin{itemize}
    \item Key size
    \item Block size
    \item Estimated security level
\end{itemize}
\textbf{Computational security against exhaustive key search}
• A fixed key size defines an upper bound on the security of the cipher.
• If the key k is a bitstring of length lk then there are 2lk keys.
• Given a small number of plaintext-ciphertext pairs the attack has complexity of 2lk−1 operations.
\textbf{Computational security against exhaustive data analysis}
• Text dictionary attack: For block size m, if an attacker has collected 2m plaintext-
ciphertext pairs then they have a complete dictionary of the cipher.
• Matching ciphertext attack: If you have collected 2m/2 blocks you expect to find matching ciphertext blocks within them.
\textbf{Known attacks against private-key block ciphers}
In the next lecture we will consider
• differential cryptanalysis, a chosen plaintext attack and
• linear cryptanalysis, a known plaintext attack.
These are attacks on SPNs which are also relevant to the design of DES and AES.

\subsection{Differential Cryptanalysis}
Objective
Find targeted key bits.
Basic idea
Exploit a situation whereby a particular difference between ciphertexts $y\prime$ occurs, given a par- ticular difference between plaintexts $x\prime$, with a very much higher probability than is ideal.
\subsection{Linear Cryptanalysis}
A related attack is linear cryptanalysis
• Objective: Find targeted key bits.
• Known plaintext attack: attacker has a set of plaintext-ciphertext pairs encrypted with
the same key k.
• Using probabilistic analysis we find biased linear approximations for the S-boxes. We construct a linear approximation, with a large bias, of the SPN (excepting the final round) in terms of plaintext bits and state bits.
• For each candidate key we partially decrypt each ciphertext and see if the linear approx- imation holds for state, incrementing a counter for the key if it does.
• The candidate key with largest bias (from |input pairs|/2) should contain the targeted key bits.
