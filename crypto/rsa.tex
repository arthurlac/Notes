\section{RSA Cryptosystem}
\textit{Public key cryptography}, or asymmetrical cryptography,
is any cryptographic system that uses pairs of keys: public keys which may be disseminated widely,
and private keys which are known only to the owner.
The essential advantage of public key cryptography is that it provides secure communication between two people who had not met or exchanged securely a secret key.
This accomplishes two functions: authentication, where the public key verifies that a holder of the paired private key sent the message,
and encryption, where only the paired private key holder can decrypt the message encrypted with the public key.

\textbf{1976: Diffie and Hellman}
“We stand today on the brink of a revolution in cryptography.”
\begin{itemize}
    \item Gave an abstract way of providing secure communication between two people who had not met or exchanged securely a secret key.
    \item Argued how such a system could also provide secure digital signatures.
    \item Gave a practical method by which two people, without the aid of a trusted authority, can
        establish a shared secret key using an insecure channel.
\end{itemize}

\begin{definition}
    The essence of a \textit{public key cryptosystem} is defined as the tuple
    $(P,C,K,E,D)$
    \begin{itemize}
        \item For every key $k \in K$, $e_k$ is the inverse of $d_k$
        \item For every key $k \in K$,
            and for every $x \in P$ and $y \in C$
            $e_k(x)$ and $d_k(y)$ are easy to compute
        \item It is computationally infeasible for almost all $k \in K$
            to derive $d_k$ from $e_k$.
        \item For every $k \in K$ it is feasible to derive $e_K$ and $d_k$
    \end{itemize}
    We have $e_k$ as the public key which allows \textbf{anyone} to encrypt a message
    which only the holder of $d_k$ will be able to decrypt.
\end{definition}

Clearly a public-key cryptosystem can never be unconditionally secure.
\begin{itemize}
    \item Alice looks up Bob’s public key function $e_k$ and encrypts $x : y = e_k(x)$.
    \item Oscar encrypts each possible message in turn until he finds the unique $x$ such that $y = e_k(x)$.
\end{itemize}
Note that Oscar can always launch a chosen-plaintext attack.

\begin{definition}
    A \textit{one-way function} is a function $f : X \rightarrow Y$ such that 
    for all $x \in X$ it is easy to compute $f(x)$ but for (almost) all $y \in Y$ it is computationally
    infeasible to find an $x$ where $f(x) = y$.
\end{definition}
\begin{definition}
    A \textit{trap-door one-way function} is a one way function $f$ such that
    given some additional trap-door information it becomes feasible, 
    for all $y \in Y$ to find $x \in X$ such that $y = f(x)$.
\end{definition}

Given a public-key cryptosystem in which $P = C$
using the fact that $e_k$ and $d_k$ are inverses
we can define a secure digital signature scheme as such
we might define a mechanism to allow secure digital signatures:
\begin{itemize}
    \item
        Alice wishes to sign message $x$.
        Using her private decryption function she obtains $y$ where $y = d_k(x)$;
        sending $y$ to Bob.
    \item Bob can then encrypt $y$ with Alice’s public encryption function: $e_k(y) = x$
    \item Only Alice could have computed $y$ such that $e_k(y) = x$ hence Bob is convinced that
    Alice signed the message.
    Furthermore anyone could have checked Alice’s signature, not just Bob.
\end{itemize}

\subsection{Mathematical background}
\textbf{The Euler function}\\
\begin{definition}
    The \textit{Euler function} $\phi(x)$ is
\end{definition}
It has the following properties
\begin{itemize}
    \item If $p$ is prime then $\phi(p)$ is $p - 1$
    \item $\phi$ is multiplicative; that is,
        if $gcd(m,n) = 1$ then $\phi(m \cdot n) = \phi(m) \cdot \phi(n)$
    \item If $\phi(n) \geq 2$ then $\phi(n)$ is even.
\end{itemize}

\textbf{Fermat's Little Theorem}\\
\begin{definition}
    \textit{Fermat's little theorem} states that if $p$ is a prime number,
    then for any integer $a$ the number $a^p - a$ is an integer multiple of $p$.
    In the notation of modular arithmetic this is expressed as
    $$ a^p \equiv a \Mod p $$
\end{definition}
For example, if $a = 2$ and $p = 7$, then $2^7 = 128$, and 128 - 2 = 126 = 7 $\times$ 18.

\textbf{Chinese Remainder Theorem}\\
\begin{definition}
    \textit{Chinese Remainder Theorem} is a way of solving certain systems of congruences.
    %%Letm1,m2,...,mr bepairwiserelativelyprimepositiveintegers. Supposea1,a2,...,ar ∈Zandconsider:
    $$
    x \equiv a_1 (\Mod m_1) \\
    x \equiv a_2 (\Mod m_2) \\
    x \equiv a_r (\Mod m_r)
    $$
    %%The Chinese remainder theorem states that this system has a unique solution modulo M = m1m2 ···mr
\end{definition}

\textbf{The Euclidean Algorithm}\\

\textbf{The Extended Euclidean Algorithm}\\

\subsection{RSA Cryptosystem}
The Rivest-Shamir-Adleman cryptosystem,
named after its authors, was one of the first public key cryptosystems.
RSA is a relatively slow algorithm, and because of this,
it is less commonly used to directly encrypt user data.
More often, RSA passes encrypted shared keys for symmetric key cryptography
which in turn can perform bulk encryption-decryption operations at much higher speed.

\begin{definition}
    The \textit{RSA cryptosystem} is defined as such
    \begin{itemize}
        \item Let $p$ and $q$ be large primes, (in some implementations 1024 bits each)
        \item Let $n = p \cdot q$
        \item Let $P = C = \Z_n$
        \item $K = \{(n,p,q,a,b) | ba \equiv 1 \Mod{\phi(n)}\}$
        \item $(n,a)$ is the public key
        \item $(n,b)$ is the public key
        \item For $K = (n,p,q,a,b)$ we have
            $$e_k(x) = x^b \Mod n = y$$
            $$d_k(y) = y^a \Mod n = x$$
    \end{itemize}
\end{definition}

\subsection{RSA Toy Example}
Firstly Bob chooses $p = 127$ and $q = 131$ and calculates $n$ and $\phi(n)$
\begin{align*}
    n &= 127  \times 131  &\phi(n) &= (p \mns 1) \times (q \mns 1)\\
      &= 16637            &        &= 126        \times 130       \\
      &                   &        &= 16380
\end{align*}
Bob selects some $b$ such that $gcd(b, \phi(n)) = 1$;
we shall use $b = 4057$ and calculate $a$
\begin{align*}
    1      &= ab \Mod{\phi(n)}\\
    b^{-1} &= a \Mod{\phi(n)}\\
           &= 10453 \Mod{\phi(n)}\\
         a &= 10453
\end{align*}
Thus Bob has the public key $(n,b) = (16637,4057)$ and the private key $(p,q,a) = (127, 131, 10453)$.
Alice has Bob’s public key and she wishes to encrypt the plaintext $x = 9031$.
She computes:
\begin{align*}
   y &= x^b \Mod{n} \\
     &= 90314057 \Mod{16637} \\
     &= 1870
\end{align*}
So Alice sends $y = 1870$ over the insecure channel. Bob now uses his decryption exponent a to compute:
\begin{align*}
   x &= y^a \Mod{n} \\
     &= 187010453 \Mod{16637} \\
     &= 9031
\end{align*}

\subsection{RSA Implementation Concerns}
To be ready to use RSA we have to
\begin{enumerate}
    \item Generate two large primes $p, q$ such that $p \neq q$
    \item Calculate $n = p \cdot q$ and $\phi(n) = (p \mns 1) (q \mns 1)$
    \item Choose a random $1 < b < \phi(n)$ such that $gcd(b, \phi(n)) = 1$
    \item Calculate $a = b^{-1} \Mod{\phi(n)}$
    \item Set the public key to $(n, b)$ and the private key to $(p, q, a)$
\end{enumerate}
For RSA to be a practical algorithm we need these steps to be computationally feasible.
Firstly the issue of generating large primes
\begin{itemize}
    \item Basic method: generate a large random number, test for primality.
    \item PRIMES is in P: In 2002 Agrawal, Kayal and Saxena proved that
        there is a polynomial time deterministic algorithm for primality testing; but we don’t use this.
    \item Instead we use one of a variety of randomised polynomial time algorithms. The Miller-Rabin algorithm is the most prominent.
\end{itemize}

The other large cost involved in the parameter generation is the calculation of
$a = b^{-1} \Mod{\phi(n)}$
We use extended Euclidean algorithm.
Can be calculated in time $O((log k)^2)$
What about the modular exponentiation in RSA?
To calculate $x^b \Mod{n}$ we can
\begin{itemize}
    \item Very slow: $b \mns 1$ modular multiplications
    \item Faster: use the square and multiply algorithm
    \item Faster again: use Chinese remainder theorem (if you are Bob)
\end{itemize}

\subsection{RSA and Factoring}
Oscar, listening to communications will try to recover $x$ given $x^b$ $\Mod{n}$, $n$ and $b$.
This is called the RSA problem.
There is no efficient algorithm known to solve this problem.
The security of RSA rests on the fact that given $n$
it is hard to find primes $p,q$ such that $n = p \cdot q$.
If factoring is computationally feasible RSA is insecure
as one can just find $p$ and $q$ and then recreate the private key $a$.
A concern is that finding $\phi(n)$ is easier than factoring $n$;
given that $$1 = ab \Mod{\phi(n)}$$
Fortunately it is the case that calculating $\phi(n)$ is as computationally difficult as factoring $n$.

\textbf{Factoring as an attack}
\begin{definition}
    \textit{Splitting}
It suffices to study algorithms which split n, that is, 
    find $1 < a < n, 1 < b < n$ such that $n = ab$.
    b a and b are said to be non-trivial factors of n.
\end{definition}

Trial division

Before using a more expensive method on n we would usually trial divide by “small” primes.
 “small” is considered as a function of the size of n
In the extreme case we trial divide with all primes $p \leq abs(\sqrt{n})$
In the worst case n is a product of two primes of roughly the same size
Perfectly reasonable method if $n < 10^{12}$

There are two categories of factoring algorithm:
Special purpose factoring algorithms
General purpose factoring algorithms

General purpose algorithms include
\begin{itemize}
    \item The quadratic sieve
    \item The general number field sieve
\end{itemize}

Special purpose algorithms include
\begin{itemize}
    \item Pollard’s $p \mns 1$ method
    \item William’s $p + 1$ method
    \item The elliptic curve factoring method
\end{itemize}

Special purpose factoring algorithms and RSA?
Many authors have recommended that p, q be chosen to be strong primes.
\begin{definition}
    $p$ prime is strong if
    \begin{itemize}
        \item $p \mns 1$ has a large prime factor (denoted $r$)
        \item $p + 1$ has a large prime factor
        \item $r \mns 1$ has a large prime factor
    \end{itemize}
\end{definition}

However, strong primes offer no protection against the Elliptic Curve Method

\subsubsection{Elliptic curve method}
\begin{itemize}
    \item The Elliptic curve method (ECM) is a generalisation of Pollard’s $p \mns 1$ method.
    \item Depends on an integer “close to” $p$ having only “small” prime factors.
    \item This is more likely than the situation required by Pollard’s $p \mns 1$ method.
    \item Pollard’s method depends on a relation which holds in the group $\mathbb{Z}^*_p$.
    \item ECM depends on a relation which involves groups defined on elliptic curves modulo p.
    \item Tends to find smaller factors first.
\end{itemize}

\subsubsection{Factoring with Congruent Squares}
To factor $n$ look for $x,y \in \mathbb{Z}$ such that
$$x^2 \mns y^2 = n$$
If such $x, y$ can be found then 
$$d = gcd(x \mns y, n)$$
is a non trivial factor of $n$.

In the 1920’s Kraitchik modified the idea of difference of two squares: Congruent squares.
To factor n look for $x,y \in \mathbb{Z}$ such that
$$x^2 \equiv y^2 \Mod{n}$$

Proposition 4.
If we also have $x\not\equiv \pm y \Mod{n}$ then
$$gcd(x \mns y, n) \quad\textrm{and}\quad gcd(x + y, n)$$
are non trivial factors of $n$.

Given $a$ we may factor $n$.
Imagine Oscar has Bob’s public key $(n, b)$ and his decryption exponent $a$.
Oscar can factor $n$.
By definition $ba \equiv 1 \Modpn$
$$ba - 1 = k\phi(n)\quad\textrm{for some } k \in \mathbb{N}$$
Hence, $\forall x \in \mathbb{Z}_n^*$:
$$x^{ba \mns 1} \equiv \Mod{n}$$
$$y_1 = \sqrt{x^{ba \mns 1}} = x^{(ba\mns1)/2}$$
$$y_1^2 \equiv 1 \Mod{n}$$
We already have that
so we can again attempt to factor n.
Continue process until either
You find a non trivial factor of n
$(ba \mns 1)/2s$ is not divisible by 2
We factor $n$ with probability half.

\begin{example}
    $$n = 437 = 19 \times 23$$
    $$\phi(n) = (19\mns1) \times (23\mns1) = 396$$
    $$b = 7$$
    $$a \equiv b^{\mns1} \equiv 7^{\mns1} \equiv 283 \Modpn$$
    Assume we know $a$
    $$\frac{ba\mns1}{2} = \frac{(7\times283)\mns1}{2} = 990$$
    $$\textrm{Try } x = 2$$
\end{example}

\begin{definition}
    An integer $n \in \mathbb{N}$ is said to be \textit{B-Smooth} if
\end{definition}

\begin{example}
\end{example}
\subsubsection{Dixon’s Random Squares}

\subsection{RSA Security}
\subsubsection{Small Private Exponent}
\subsubsection{Small Public Exponent}
\subsubsection{Coppersmith’s Theorem}
\subsubsection{Hastad’s Broadcast Attack}
\subsubsection{Franklin-Reiter Related Messages}
\subsubsection{Padding}
\subsubsection{Crouch/Davenport}

\subsection{RSA in practice}
\subsubsection{RSA Optimal Asymmetric Encryption Padding}
\subsubsection{Existential Forgery}
