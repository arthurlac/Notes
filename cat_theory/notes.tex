\documentclass{article}

\usepackage[utf8]{inputenc}
\usepackage[english]{babel}
\usepackage{amsfonts}
\usepackage{amssymb}
\usepackage{amsthm}
\usepackage{mathtools}

\theoremstyle{definition}
\newtheorem{definition}{Definition}[section]
\newtheorem{example}{Example}[section]

\usepackage{tikz-cd}
\usetikzlibrary{arrows}
\tikzset{
	commutative diagrams/.cd,
	arrow style=tikz,
	diagrams={>=space}
 }
\tikzcdset{
	arrow style=tikz,
	diagrams={>={Straight Barb[scale=0.8]}}
}

\begin{document}

\section{The most important definition}
\begin{definition}
	A \textit{functor} $U : \mathcal{C} \rightarrow \mathcal{D}$ has a \textit{left adjoint} if
		$$
		\forall\ X \in \mathcal{D}
		\quad\exists\quad
		FX \in \mathcal{C} \ \textrm{and}\ \eta_X : X \rightarrow UFX \in \mathcal{D}
		$$
		such that
		$$
		\forall\ A \in \mathcal{C}\ \textrm{and}\ \forall\ f : X \rightarrow UA
		\quad\exists\quad
		!g : FX \rightarrow A
		$$
		such that the following diagram commutes
\end{definition}

\begin{center}
	\begin{tikzcd}
		X \drar[swap]{f} \rar{\eta_X} & UFX \dar{Ug} \\
		& UA
	\end{tikzcd}
\end{center}

\section{Categories and Functors}

\begin{definition}
	A \textit{category} $\mathcal{C}$ consists of
	\begin{itemize}
		\item A set $Ob\,\mathcal{C}$, elements of which are called \textit{objects} of $\mathcal{C}$.
		\item For each $X, Y \in Ob\,\mathcal{C}$
			a set $\mathcal{C}(X,Y)$ called the \textit{homset} from $X$ to $Y$.
		\item For each $X, Y, Z \in Ob\,\mathcal{C}$ a \textit{composition function} $\circ$
			$$\mathcal{C}(X,Y) \circ \mathcal{C}(Y,Z) \rightarrow \mathcal{C}(X,Z)$$
		\item For all $X \in Ob\,\mathcal{C}$ an element $\iota_X$ of $\mathcal{C}(X,\ X)$,
			or equivalently a function $\iota_X : 1 \rightarrow \mathcal(X,\ X)$.
			Such that
			$$\forall f \in \mathcal{C}(X,Y)\quad f \circ \iota_X = \iota_Y \circ f = f$$
		\item Composition is associative.
			$$\forall
			f \in \mathcal{C}(X,Y),\ 
			g \in \mathcal{C}(Y,Z),\ 
			h \in \mathcal{C}(X,Z)\quad
			f \circ (g \circ h) = (f \circ g) \circ h
			$$
			%%\begin{center}
				%%\begin{tikzcd}
					%%X\ar[r,"f"]\ar[rr,out=-30,in=210,swap,"h"] & Y\ar[r,"g"] & Z
				%%\end{tikzcd}
			%%\end{center}
	\end{itemize}
\end{definition}

\par
An element $f$ of $\mathcal{C}(X,Y)$ is called an \textit{arrow}, or a \textit{function}
or a \textit{morphism}. The object $X$ is called the \textit{domain} of $f$ and $Y$ is
the \textit{codomain}.

\begin{example}
	The foremost example of a category is the category of \textit{small sets}.
	We say small sets due to Russel's paradox, conversely phrased a category is small
	if its objects and arrow consitute sets; otherwise it is large.

	The objects in $Ob\,\mathcal{C}$ are sets,
	a morphism from $X$ to $Y$ is a function $f : X \rightarrow Y$.
	The composition of Set is given by composition of functions,
	and the identity maps are given by the identity functions.
\end{example}





\begin{definition}
	A \textit{functor} $U : \mathcal{C} \rightarrow \mathcal{D}$ consists of
	\begin{itemize}
		\item a function $Ob\,U : Ob\,\mathcal{C} \rightarrow Ob\,\mathcal{D}$.
		\item For each $X, Y \in Ob\,\mathcal{C}$ a function
			$$U_{X,Y} : \mathcal{C}(X,Y) \rightarrow \mathcal{D}(UX, UY)$$
			such that $U$ respects both composition and identity.
	\end{itemize}
\end{definition}




\section{Monoids and Groups}

\begin{definition}
	A \textit{monoid} is a set $M$ together with an associative binary operation $\cdot$ such that
	$$\forall\ x, y, z \in M\ (x \cdot y) \cdot z = x \cdot (y \cdot z)$$
	and a unit element $e \in M$ where
	$$\forall\ m \in M\ m \cdot e = e \cdot m = m$$
	Thus a monoid could be a considered a triple $(M,\ \cdot,\ e)$.
\end{definition}

At this point we can define the "forgetful functor", consider a functor from the
category of monoids to the category of sets. For this functor we can simply prise the
set $M$ from the monoid and "forget" the binary associative operation $\cdot$ as well as
the unit $e$. It is called the forgetful functor because it forgets some of the structure of
the domain category.

\begin{definition}
	Given two monoids $M$ and $N$, a \textit{monoid map} is a function $f : M \rightarrow N$
	which preserves multiplication and the unit where
	$$f(e_M) = e_N\ \textrm{and}\ \forall x, y \in M f(x \cdot y) = z, z \in N$$
\end{definition}

\begin{example}
	Perhaps the simplest example of a monoid is taking the natural numbers $\mathbb{N}$
	(including zero), addition, and zero.
	Thus the monoid is $(\mathbb{N}, +, 0)$. We can see that axioms
	are obeyed. Addition is associative and $\forall x \in \mathbb{N}\ x + 0 = x$.
	A further example is multiplication with unit element being 1.
\end{example}

\begin{definition}
	A \textit{commutative monoid} is a monoid $(M,\ \cdot,\ e)$ where
	$$\forall x, y \in M\ x \cdot y = y \cdot x$$
\end{definition}

Because both addition and multiplication are commutative, the monoids $(\mathbb{N},\ +,\ 0)$
and $(\mathbb{N},\ \times,\ 1)$ are commutative monoids.
$$ \forall x, y \in \mathbb{N}$$
$$x + y = y + x$$
$$x \times y = y \times x$$

\begin{example}
	An example of a monoid which is not commutative is that of strings and concatenation.
	Consider the alphabet $\Sigma = \{a,\ b,\ \epsilon\}$ where $\epsilon$ is the empty string.
	Strings in this alphabet include $\epsilon,\ aa,\ ab,\ bbb$ and so on.
	Our binary associative operation is concatenation ($+$).
	The unit element is $\epsilon$.
	From this we can see that
	$$aa + (bb + ab) = (aa + bb) + aa = aabbaa$$
	$$aa + \epsilon = aa = \epsilon + aa$$
	Furthermore that
	$$aa + bb \ne bb + aa$$
\end{example}

A further example of a non-commutative monoid is that of matrix multiplication.

\begin{definition}
	A \textit{idempotent monoid} is one where $\forall\ x \in M\ x \cdot x = x$.
\end{definition}

To understand this definition consider the monoid of sets and union, $(P_f(X), \cup, \varnothing)$.
$M$ is the finite power set of $X$ (set of all finite subsets), $\cdot$ is union, and $e$ is
the empty set $\varnothing$.
This is an example of an idempotent monoid because
$$\forall\ x \in P_f(X)\ x \cup x = x$$
It is also a commutative monoid because
$$\forall\ x,\ y \in P_f(X)\ x \cup y = y \cup x$$

\begin{definition}
	A \textit{group} is a monoid where there exists an inverse element such that
	$$\forall\ m \in M\ \exists\ n \in M\ m \cdot n = e$$
\end{definition}

\begin{definition}
	A \textit{Abelian group} is a commutative group, i.e.
	$$\forall\ x, y \in M\ x \cdot y = y \cdot x$$
\end{definition}

\begin{example}
	An exmaple of an Abelian group
\end{example}

\end{document}
