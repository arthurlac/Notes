\documentclass{article}

\usepackage[utf8]{inputenc}
\usepackage[english]{babel}
\usepackage{amsthm}
\usepackage{mathtools}

\theoremstyle{definition}
\newtheorem{definition}{Definition}[section]
\newtheorem{example}{Example}[section]

\usepackage{tikz-cd}
\usetikzlibrary{arrows}
\tikzset{
	commutative diagrams/.cd,
	arrow style=tikz,
	diagrams={>=space}
 }
\tikzcdset{
	arrow style=tikz,
	diagrams={>={Straight Barb[scale=0.8]}}
}

\begin{document}

\section{The most important definition}
\begin{definition}
	A \textit{functor} $U : \mathcal{C} \rightarrow \mathcal{D}$ has a \textit{left adjoint} if
		$$
		\forall\ X \in \mathcal{D}
		\quad\exists\quad
		FX \in \mathcal{C} \ \textrm{and}\ \eta_X : X \rightarrow UFX \in \mathcal{D}
		$$
		such that
		$$
		\forall\ A \in \mathcal{C}\ \textrm{and}\ \forall\ f : X \rightarrow UA
		\quad\exists\quad
		!g : FX \rightarrow A
		$$
		such that the following diagram commutes
\end{definition}

\begin{center}
	\begin{tikzcd}
		X \drar[swap]{f} \rar{\eta_X} & UFX \dar{Ug} \\
		& UA
	\end{tikzcd}
\end{center}

\section{Categories and Functors}

\begin{definition}
	A \textit{category} $\mathcal{C}$ consists of
	\begin{itemize}
		\item a set $Ob\,\mathcal{C}$, elements of which are called \textit{objects} of $\mathcal{C}$.
		\item for each $X, Y \in Ob\,\mathcal{C}$ a set called the \textit{homset} from $X$ to $Y$.
		\item for each $X, Y, Z \in Ob\,\mathcal{C}$ a \textit{composition function} $\circ$
			$$\mathcal{C}(X,Y) \circ \mathcal{C}(Y,Z) \rightarrow \mathcal{C}(X,Z)$$
		\item for each $X \in Ob\,\mathcal{C}$ a \textit{identity function} $\iota_X$
			$$\iota_X \mathcal{C}(X) \rightarrow \mathcal{C}(X)$$
			\begin{center}
				\begin{tikzcd}
					X\ar[r,"\iota_X"] & X
				\end{tikzcd}
			\end{center}
		\item Composition is associative.
			$$\forall
			f \in \mathcal{C}(X,Y),\ 
			g \in \mathcal{C}(Y,Z),\ 
			h \in \mathcal{C}(X,Z)\quad
			f \circ (g \circ h) = (f \circ g) \circ h
			$$
			\begin{center}
				\begin{tikzcd}
					X\ar[r,"f"]\ar[rr,out=-30,in=210,swap,"h"] & Y\ar[r,"g"] & Z
				\end{tikzcd}
			\end{center}
		\item $\forall f \in \mathcal{C}(X,Y)\quad f \circ \iota_X = \iota_Y \circ f = f$
	\end{itemize}
\end{definition}

\par
An element $f$ of $\mathcal{C}(X,Y)$ is called an \textit{arrow}, or a \textit{function}
or a \textit{morphism}. The object $X$ is called the \textit{domain} of $f$ and $Y$ is
the \textit{codomain}.

\begin{example}
	The foremost example of a category is the category of \textit{small sets}.
	The objects in $Ob\,\mathcal{C}$ are sets,
	a morphism from $X$ to $Y$ is a function $f : X \rightarrow Y$.
	The composition of Set is given by composition of functions,
	and the identity maps are given by the identity functions.
\end{example}





\begin{definition}
	A \textit{functor} $U : \mathcal{C} \rightarrow \mathcal{D}$ consists of
	\begin{itemize}
		\item a function $Ob\,U : Ob\,\mathcal{C} \rightarrow Ob\,\mathcal{D}$.
		\item For each $X, Y \in Ob\,\mathcal{C}$ a function
			$$U_{X,Y} : \mathcal{C}(X,Y) \rightarrow \mathcal{D}(UX, UY)$$
			such that $U$ respects both composition and identity.
	\end{itemize}
\end{definition}

\section{Monoids}

\begin{definition}
	A \textit{monoid} is a set $M$ together with an associative binary operation $\cdot$ and
	a unit element $e \in M$ where
	$$\forall\: m \in M\: \quad m \cdot e = e \cdot m = m$$
\end{definition}

\begin{definition}
    A \textit{commutative monoid} is 
\end{definition}

\begin{definition}
	A \textit{group} is a monoid where there exists an inverse operation where
	$$\forall\: m \in M\: \exists\: n \in M\: \quad m \cdot n = e$$
\end{definition}

\begin{definition}
    A \textit{monoid map} is
\end{definition}

\end{document}
