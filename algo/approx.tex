\section{Approximate Algorithms}
Many important practical problems are in NP-hard,
we need a tractable solution for them.
It may be the case that we can find an algorithm for our relevant cases.
Alternatively we may need to solve the problem for the input
data of a relatively small size, so that even an exponential-time algorithm is acceptable.
Finally, we can try to solve the problem approximately;
i.e. solve the problem as close to optimally as possible but accept some degree of error
or incorrectness.
This last approach is suitable for optimization search problems.\\

\textbf{Examples of optimization search problems}
\begin{itemize}
    \item Travelling Salesperson (find the smallest tour)
    \item Find the largest clique in a graph
    \item Vertex Covering (find the smallest covering)
\end{itemize}

\subsection{Ratio bound and relative error}
For approximation algorithms we need an \textit{objective function}
which will allow us to measure the difference of the approximate and optimal algorithms.\\

For example in the traveling sales person problem,
the objective function is the length of the tour produced by the algorithm.
We want to determine how "close" our approximate algorithm is to the optimal one.
We will assume that the objective function is always positive.

\begin{definition}
    Let $C$ denote the value of the objective function at our approximate solution,
    and $C\prime$ at an optimal solution.
\end{definition}

\textbf{Ratio Bound}
\begin{definition}
    An approximation algorithm has \textit{ratio bound} $\rho(n)$
    if for any input of the size $n$ the value $C$ of the objective function
    at the approximate solution satisfies the inequality
    $$max(C/C\prime, C\prime/C) \leq \rho(n)$$
\end{definition}

Observe that
$$ 0 < C \leq C\prime $$
thus for maximization problems
$$ max(C/C\prime, C\prime/C) = C\prime/C $$
and for minimization problems
$$ max(C/C\prime, C\prime/C) = C/C\prime $$

Observe that $\rho(n) \geq 1$,
and secondly that when $\rho(n) = 1$ we have an optimal solution.
By definition a large  ratio bound (might) means that the approximate
solution is much worse than an optimal one.
Usually $\rho(n)$ is a constant,
i.e. the ratio bound does not grow with the size of the input.\\

\textbf{Relative Error Bound}
\begin{definition}
    For $C, C\prime$ and $n$  the \textit{relative error bound $\epsilon(n)$} is defined as
    $$\frac{abs(C - C\prime)}{C\prime} \leq \epsilon(n)$$
\end{definition}

\textbf{Approximation algorithm for Vertex Covering}\\
Recall that the vertex covering problem is defined as such
\begin{itemize}
    \item \textbf{Input}:
        a graph $\G,\ (V,E)$
    \item \textbf{Output}:
        a set $S \subset V$ such that $\forall\ (v,w) \in E$
        either $v$ or $w$ are in $S$.
        I.e. every vertex is reachable,
        and the fewest possible vertices are in $S$.
\end{itemize}

\textbf{Algorithm}
\begin{enumerate}
    \item Let $S$ be $\varnothing$
    \item Let $E\prime$ be the set of edges $E$
    \item While $E\prime$ is non-empty
        \begin{enumerate}
            \item
                Let $(v,w)$ be an arbitrary edge of $E\prime$
            \item
                Let $S$ be $S \cup \{v,w\}$
            \item
                Remove from $E\prime$ every edge having either $v$ or $w$ as a vertex
        \end{enumerate}
    \item Return the set $S$ as the vertex covering

\end{enumerate}

\textbf{Properties of the Algorithm}
\begin{enumerate}
    \item The algorithm is correct
    \item The algorithm has polynomial complexity
    \item The algorithm $\rho(n) = 2$
\end{enumerate}
\textbf{Proof}
\begin{proof}
    Let $A$ denote the set of all edges that are chosen by the algorithm.
    No two edges of $A$ have a common vertex,
    since once the edge $(v, w)$ is chosen,
    all edges having either $v$ or $w$ as vertices are deleted.
    Thus, each new execution adds exactly two new vertices to $S$, thus $|S| = 2|A|$.
    The optimal covering $S\prime$, by definition, must cover every edge in $A$.
    Hence
    \begin{align*}
           S\prime &\supset A \\
         |S\prime| &\geq |A|\\
         |S\prime| &\geq |S|/2\\
        2|S\prime| &\geq |S|\\
                 2 &\leq |S|/|S\prime|\\
                   &\leq C/C\prime \\
           \rho(n) &= 2
    \end{align*}
\end{proof}

\textbf{Approximation algorithm for Travelling Salesperson with triangle inequality}\\
\begin{itemize}
    \item \textbf{Input}:
        A matrix $d_{ij}$ such that $\forall\ i,j\, d(i,j) > 0$ and $\forall\ i,j,k$
        $$d(i,j) \leq d(i,k) + d(k,j)$$
    \item \textbf{Output}:
        A permutation $i_1, i_2,\dots i_m$ (called “tour”) of such that the following sum is minimised.
        $$\sum_{1\leq x < m} d(x,x+1)$$
\end{itemize}
Travelling Salesman with triangle inequality is NP-hard.\\

\textbf{Algorithm for $\Delta$Traveling Salesperson}\\
Given a connected graph $\mathcal{G}\ (V,E)$,
\begin{enumerate}
    \item
        Choose a vertex $v \in V$, to serve as the root of a tree.
        Build a spanning tree $T$ beginning at $v$;
        note that this can be done in polynomial time using Prim's or Kruskal's algorithm.
    \item
        Perform a pre-order walk on the tree $T$ returning the list of nodes as a tour.
\end{enumerate}

We can now propose three properties of this algorithm:
\begin{enumerate}
    \item It is correct
    \item It has polynomial complexity
    \item It has a ratio bound $\rho(n) = 2$
\end{enumerate}

\begin{proof}
    Let $L*$ denote an optimal tour,
    $d(L)$ be the length of $L$,
    $d(L*)$ be the length of $L*$.
    Observe that a spanning tree for $\G$ can be obtained by deleting any edge from any tour.
    Thus, for a minimal spanning tree $T$,
    with sum of weights on all edges $d(T)$,
    the following inequality holds
    $$d(T) \leq d(L*)$$
    Let W be a Full Walk of $T$,
    and $d(W)$ be its length.
    Since $W$ visits every edge of $T$ exactly twice,
    $$d(W ) = 2d(T )$$
    It follows that
    $$d(W) \leq 2d(L*)$$
    By the triangle inequality,
    Preorder Walk L is not longer than $W$:
    $$d(L) \leq d(W)$$
    As a result, we have:
    \begin{align*}
        d(L) &\leq 2d(L*)\\
        d(L) &\leq 2d(L*), d(L)/d(L*) wu 2 \\
        \rho(n) &= 2
    \end{align*}
\end{proof}

\textbf{No Ratio Bound for Traveling Salesperson}\\
\begin{theorem}
    If P $\neq$ NP then there is no
    approximation algorithm for general Travelling Salesman with polynomial complexity
    and constant ratio bound.
\end{theorem}

\begin{proof}
%We show how to solve Hamiltonian Cycle (HC) problem in polynomial time using a hypothetical approximation algorithm, with polynomial complexity and constant ratio bound ρ, for general Travelling Salesman (TS), as a subroutine.
%Let G = (V, E) be an input of HC. Construct an input of TS with the set of cities V and the distance function:
%d(u, v) = 1 if (u, v) ∈ E
%d(u, v) = ρ|V | + 1 if (u, v) ∈/ E.
%Clearly, this input of TS can be constructed from G in polynomial time.
%If G contains a hamiltonian cycle, then its length will be |V |. If G does not contain a hamiltonian cycle, then there is a pair (u,v) of subsequent cities in any tour such that (u, v) ∈/ E, thus the size of any tour will be at least
%(ρ|V | + 1) + (|V | − 1) = (ρ + 1)|V | > ρ|V |.
%Thus, we can decide whether G contains a hamiltonian cycle by approximately solving the instance of TS. If the size of the solution tour is ≤ ρ|V |, then G contains a hamiltonian cycle, else G has no hamiltonian cycles.
\end{proof}

