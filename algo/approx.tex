\section{Approximate Algorithms}
Many problems in NP-hard are problems for which we need a polynomial algorithm;
thus we are left with few, but not none, options.
It may be the case that we can find an algorithm for our relevant cases.
Alternatively we may need to solve the problem for the input
data of a relatively small size, so that even an exponential-time algorithm is acceptable.
Finally, we can try to solve the problem approximately;
i.e. solve the problem as close to optimally as possible but accept some degree of error
or incorrectness.
This last approach is suitable for optimization search problems.\\

\textbf{Examples of optimization search problems}
\begin{itemize}
    \item Travelling Salesperson (find the smallest tour)
    \item Find the largest clique in a graph
    \item Vertex Covering (find the smallest covering)
\end{itemize}

\subsection{Ratio bound and relative error}
For approximation algorithms we need an \textit{objective function}
which will allow us to measure the difference of the approximate and optimal algorithms.
For example in the traveling sales person problem,
the objective function is the length of the tour produced by the algorithm.
We want to determine how "close" our approximate algorithm is to the
optimal one.  We define "closeness" as \textit{ratio bound and relative error bound}.

We will assume that the objective function is always positive.

Let $C$ denote the value of the objective function at our approximate solution,
and $C\prime$ at an optimal solution.

\textbf{Ratio Bound}\\
\begin{definition}
    An approximation algorithm has \textit{ratio bound $\rho(n)$}
    if for any input of the size $n$ the value $C$ of the objective function
    at the approximate solution satisfies the inequality
    $$
    max(C/C\prime, C\prime/C) \leq \rho(n)
    $$
\end{definition}

Observe that
$$ 0 < C \leq C\prime $$
thus for maximization problems
$$ max(C/C\prime, C\prime/C) = C\prime/C $$
and for minimization problems
$$ max(C/C\prime, C\prime/C) = C/C\prime $$

Firstly observe that $\rho(n) \geq 1$,
and secondly that when $\rho(n) = 1$ we have an optimal solution.
By definition a large ratio bound (might) means that the approximate
solution is much worse than an optimal one.
This is an issue of trading speed for correctness.

\textbf{Relative Error Bound}\\
\begin{definition}
    For $C, C\prime$ and $n$  the \textit{relative error bound $\epsilon(n)$} is defined as
    $$\frac{abs(C - C\prime)}{C\prime} \leq \epsilon(n)$$
\end{definition}

Observe that $\rho(n) - 1 \leq abs(C\prime - C)/C\prime$,
i.e., $\rho(n) - 1$ is a relative error bound.

\textbf{Approximation algorithm for Vertex Covering}\\
Recall that the vertex covering problem is defined as such
\begin{itemize}
    \item \textbf{Input}:
        a graph $\G,\ (V,E)$
    \item \textbf{Output}:
        a set $S \subset V$ such that $\forall\ (v,w) \in E$
        either $v$ or $w$ are in $S$.
        I.e. every vertex is reachable,
        and the fewest possible vertices are in $S$.
\end{itemize}

\textbf{Algorithm}\\
\begin{enumerate}
    \item Let $S$ be $\varnothing$
    \item Let $E\prime$ be the set of edges $E$
    \item While $E\prime$ is non-empty
        \begin{enumerate}
            \item
                Let $(v,w)$ be an arbitrary edge of $E\prime$
            \item
                Let $S$ be $S \cup \{v,w\}$
            \item
                Let $E\prime$ be $E\prime \setminus (v,w)$
        \end{enumerate}
    \item Return the set $S$ as the vertex covering

\end{enumerate}

\textbf{Properties of the Algorithm}\\
\begin{enumerate}
    \item The algorithm is correct
    \item The algorithm has polynomial complexity
    \item The algorithm $\rho(n) = 2$
\end{enumerate}
\textbf{Proof}
\begin{enumerate}
    \item
        The correctness is trivial
        as the algorithm simply loops until each edge has been covered.
    \item
        Polynomial time complexity is also trivial for same reason.
    \item
        Let $A$ denote the set of all edges that are chosen during the execution of the item (3) of the algorithm.
        No two edges of A have a common vertex, since once the edge (v, w) is chosen, all edges having either v or w as vertices are deleted.
        Thus, each execution of (3) adds exactly two new vertices to S and |S| = 2|A|.
        Any covering, in particular optimal $S\prime$, must cover every edge in $A$, thus any covering has at least $|A|$ vertices.
        So because we have $S\prime \supset A$ we also have
        $$|S\prime| \geq |A| = \frac{|S|}{2}$$
        $$\frac{S}{|S\prime|} \leq 2$$
\end{enumerate}

\textbf{Approximation algorithm for Travelling Salesperson with triangle inequality}\\
The problem is formulated as such
\begin{itemize}
    \item \textbf{Input}:
        A matrix $d_{ij}$ such that $\forall\ i,j\, d(i,j) > 0$
        $$
        i,j,k d(i,j) \leq d(i,k) + d(k,j)
        $$
    \item \textbf{Output}:
        A tour such that every 
\end{itemize}

\textbf{Algorithm for $\Delta$Traveling Salesperson}\\
Given a connected graph $\mathcal{G}\ (V,E)$,
\begin{enumerate}
    \item
        Choose a vertex $v \in V$, to serve as the root of a tree.
        Build a spanning tree $T$ beginning at $v$;
        note that this can be done in polynomial time using Prim's or Kruskal's algorithm.
    \item
        Perform a pre-order walk on the tree $T$ returning the list of nodes as a tour.
\end{enumerate}

We can now propose three properties of this algorithm
\begin{enumerate}
    \item It is correct
    \item It has polynomial complexity
    \item It has a ratio bound $\rho(n) = 2$
\end{enumerate}
\textbf{Proof}
\begin{enumerate}
    \item
        The correctness is trivial
    \item
        Proof of polynomial complexity is trivial.
    \item
        The ratio bound
\end{enumerate}

