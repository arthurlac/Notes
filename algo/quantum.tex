\section{Quantum Complexity}
\subsection{Complex Numbers}
Complex numbers play a very special role in quantum computing.
A complex number may have three principal representations.\\

\textbf{Pair Representation}\\
As a pair of real numbers $(a, b)$, traditionally written as $a + ib$,
where $a, b \in R$ and $i$ is \textit{a posteriori} interpreted as $\sqrt{-1}$.
Arithmetic operations of $+$ and $\times$ are defined as
\begin{align*}
    (a+ib) +      (c+id) &= (a+c)   + i(b+d)  \\
    (a+ib) \times (c+id) &= (ac−bd) + i(ad+bc)
\end{align*}
Observe this is consistent with our definition $i=\sqrt{-1}$.
$$i^2 = (0 + i)(0 + i) = −1$$

\textbf{Polar Representation}\\
A complex number $z = a + ib$ has a natural representation
as a vector $(a,b)$ in the real plane $R^2$.
The norm of this vector, $|z| = \sqrt{a^2 +b^2}$,
is called the \textit{modulus} or \textit{magnitude} of $z$,
while the angle $\varphi = arctan(b/a)$ between vector $(a, b)$
and the positive direction of the horizontal axis is called the argument of $z$.
The pair $(m,\varphi)$ completely determines z and is called polar representation of z.
Note that in physics literature modulus is usually called magnitude, while argument is called phase.
We pass from polar representation (m, $\varphi$) to the original representation a + ib via formulas:
a = m cos $\varphi$, b = m sin $\varphi$.
A formula for multiplication of two complex numbers is particularly
easy in the polar representation:
$$(m_1, \varphi_1) \times (m_2, \varphi_2) = (m_1 \times m_2, \varphi_1 + \varphi_2)$$

\textbf{Exponential Representation}\\
From $a = m cos \varphi$, $b = m sin \varphi$ we deduce that
$$a + ib = m(cos\varphi + isin\varphi)$$
According to the Euler formula,
$$cos\varphi + isin\varphi = e^{i\varphi}$$
where $e = 2.71828\dots$
This leads us to the exponential representation:
$a + ib = me^{i\varphi}$
Multiplication formula in this case is
$$m_1e^{i\varphi_1} \times m_2e^{i\varphi_2} = m_1m_2e^{i(\varphi_1+\varphi_2)}$$
For a complex number $z = a + ib$ its complex conjugate is the number $\bar{z} = a − ib$.
It is clear that if $z \in R$, then $\bar{z} = z$, i.e. $b = 0$.

\subsection{Vectors and matrices}
We consider an $n$-dimensional vector space $C^n$ over complex numbers.
Its elements are column vectors $z = (z_1,\dots,z_n)^T$,
where every $z_i \in C$.

In physics (and quantum computing) vector $z$ is also denoted by \textit{ket-notation},
$|z\rangle$.
There is also a sister \textit{bra-notation}, $\langle z|$,
standing for the row vector $(\bar{z_1}, \dots , \bar{z_n})$.
The \textit{bra-ket (bracket) notation} was introduced by P. Dirac.

The notation $\langle x|y \rangle$,
for two vectors $x = (x1, . . . xn)$ and $y = (y1, . . . yn)$,
means the dot (inner) product of these vectors,
namely the number x1y1 + · · · + xnyn.
The notation $|y\rangle\langle x|$ also has some meaning.
%There is also a sister bra-notation, ⟨z|, standing for the row vector (z ̄1, . . . , z ̄n). The bra-ket (bracket) notation was introduced by P. Dirac. The notation ⟨x|y⟩, for two vectors x = (x1, . . . xn) and y = (y1, . . . yn), means the dot (inner) product of these vectors, namely the number x ̄1y1 + · · · + x ̄nyn. The notation |y⟩⟨x| also has some meaning, discussed later.
As usual, linear maps between vector spaces are described by matrices,
in our case having complex numbers as elements.
We will need the following special types of matrices.
\begin{itemize}
    \item Given a matrix A,
    its adjoint matrix,
    A†, is defined as (A)T = AT ,
    where the bar denotes element-wise complex conjugation.
    \item A matrix A is Hermitian if A† = A. Observe that A is a square matrix, and all its diagonal elements are real numbers.
    \item A square matrix U is unitary if UU† = U†U = I,
    where I is the identity matrix of the appropriate size. In other words, a matrix is unitary if its adjoint is inverse.
\end{itemize}

$$
\begin{pmatrix}
    cos\ \alpha & -sin\ \alpha  \\
    sin\ \alpha & cos\ \alpha
\end{pmatrix}
$$
$$
\frac{1}{\sqrt{2}}
\begin{pmatrix}
    1 & 1  \\
    1 & -1
\end{pmatrix}
$$
\subsection{Qubits}
\subsection{Bloch sphere}
\subsection{Multiple Qubits}
\subsection{Functions from bits to bits}
\subsection{Deutsch’s Algorithm}
