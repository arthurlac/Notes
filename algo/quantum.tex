\section{Quantum Complexity}
\textbf{Complex Numbers}\\
Complex numbers play a very special role in quantum computing.
A complex number may have three principal representations.\\

\textbf{Pair Representation}\\
As a pair of real numbers $(a, b)$, traditionally written as $a + ib$,
where $a, b \in R$ and $i$ is \textit{a posteriori} interpreted as $\sqrt{-1}$.
Arithmetic operations of $+$ and $\times$ are defined as
\begin{align*}
    (a+ib) +      (c+id) &= (a+c)   + i(b+d)  \\
    (a+ib) \times (c+id) &= (ac−bd) + i(ad+bc)
\end{align*}
Observe this is consistent with our definition $i=\sqrt{-1}$.
$$i^2 = (0 + i)(0 + i) = −1$$

\textbf{Polar Representation}\\
A complex number $z = a + ib$ has a natural representation
as a vector $(a,b)$ in the real plane $R^2$.
The norm of this vector, $|z| = \sqrt{a^2 +b^2}$,
is called the \textit{modulus} or \textit{magnitude} of $z$,
while the angle $\varphi = arctan(b/a)$ between vector $(a, b)$
and the positive direction of the horizontal axis is called the \textit{argument} or \textit{phase} of $z$.
We pass from polar representation $(m, \varphi)$ to the original representation $a + ib$ via formulas:
$$a = m \cos \varphi \quad b = m \sin \varphi$$
A formula for multiplication of two complex numbers is particularly easy in the polar representation:
$$(m_1, \varphi_1) \times (m_2, \varphi_2) = (m_1 \times m_2, \varphi_1 + \varphi_2)$$

\textbf{Exponential Representation}\\
From Euler's formula and the formula from polar representation to pair representation we show that
\begin{align*}
          e^{ix} &= \cos x + i\sin x \\
          a + ib &= m(\cos\varphi + i\sin\varphi) \\
                 &= me^{i\varphi}
\end{align*}
Hence deriving a new representation for complex numbers: $me^{i\varphi}$.
Multiplication formula in this case is
$$m_1e^{i\varphi_1} \times m_2e^{i\varphi_2} = m_1m_2e^{i(\varphi_1+\varphi_2)}$$\\

For a complex number $z = a + ib$ its complex conjugate is the number $\bar{z} = a − ib$.
It is clear that if $z \in R$, then $\bar{z} = z$, i.e. $b = 0$.\\

\textbf{Vectors and matrices}\\
We consider an $n$-dimensional vector space in $\C^n$.
Its elements are column vectors $z = (z_1,\dots,z_n)^T$,
where every $z_i \in \C$.\\

In quantum computing vector $z$ is also denoted by \textit{ket-notation}, $|z\rangle$.
There is also a sister \textit{bra-notation}, $\langle z|$,
standing for the row vector $(\bar{z}_1, \dots , \bar{z}_n)$.\\

The notation $\langle x|y \rangle$,
for two vectors $x = (x_1, \dots x_n)$ and $y = (y_1, \dots y_n)$,
means the dot (inner) product of these vectors.
$$\langle x|y \rangle = \bar{x}_1y_1 + \dots + \bar{x}_ny_n$$
The notation $|y\rangle\langle x|$ also has some meaning.\\

\textbf{Types of Matrices}
\begin{definition}
    Given a matrix $A$, its \textit{adjoint matrix} $A\dagger$.
    Where the bar denotes element-wise complex conjugation,
    an adjoint matrix $A$ is defined as
    $$(\overline{A})^T = \overline{A^T}$$
\end{definition}
\begin{definition}
    A matrix $A$ is \textit{Hermitian} if $A\dagger = A$.
    Observe that $A$ is a square matrix,
    and all its diagonal elements are real numbers.
\end{definition}
\begin{definition}
    A square matrix $U$ is \textit{unitary} if
    $UU\dagger = U\dagger U = I$
    where $I$ is the identity matrix of the appropriate size.
    In other words, a matrix is unitary if its adjoint is inverse.
\end{definition}

\begin{example}
    The following matrix with real elements is unitary:
    $$ \begin{pmatrix}
        cos\ \alpha & -sin\ \alpha  \\
        sin\ \alpha & cos\ \alpha
    \end{pmatrix}
    $$
    Another important example of a unitary matrix is the \textit{Hadamard matrix},
    note that the Hadamard matrix is Hermitian:
    $$
    \frac{1}{\sqrt{2}}
    \begin{pmatrix}
        1 & 1  \\
        1 & -1
    \end{pmatrix}
    $$
\end{example}

\textbf{Types of Matrix Multiplications}\\
Now we consider two types of matrix multiplications.

The first type is the “usual” matrix multiplication of two matrices
applicable in case when the sizes of matrices “match”
with complex elements.
$$A = ||a_{ij}|| \quad\textrm{and}\quad B = ||b_{ij}||$$

The latter means that $A$ is $(m \times n)$-matrix
while $B$ is $(n \times k)$-matrix
for some positive integers $m$, $n$, $k$.
The product $AB$ is an $(m\times k)$-matrix whose $(i,j)$-element is
$$\sum_{0\leq \nu \leq n} a_{i\nu} b_{\nu j}$$

In quantum computing we need also another type of matrix multiplication, called tensor (or Kronecker) product.
\begin{definition}
    Let $A = ||a_{ij}||$ and $B = ||b_{ij}||$
    be two matrices of sizes $m\times n$ and $p\times q$ respectively,
    with complex elements.
    Their \textit{tensor product},
    $A \otimes B$, is the $(mp \times nq)$-matrix:

    $$
    A \otimes B =
    \begin{pmatrix}
        a_{11}B & \dots  & a_{1n}B \\
          \dots & \dots  & \dots   \\
        a_{m1}B & \dots  & a_{mn}B
    \end{pmatrix}
    $$

    Here are the main properties of tensor product,
    given matrices $A,B,C,D$,
    vectors $u,x,y$ of appropriate dimensions,
    and numbers $a, b$
    \begin{align*}
        (A \otimes B)(C \otimes D) &= AC \otimes BD \\
        (A \otimes B)(x \otimes y) &= Ax \otimes By \\
        (x + y) \otimes u &= x \otimes u + y \otimes u \\
        u \otimes (x + y) &= u \otimes x + u \otimes y \\
        ax \otimes by &= ab(x \otimes y) \\
        (A \otimes B)\dagger &= A\dagger \otimes B\dagger
    \end{align*}

    Tensor product is not commutative.
    $$
    \begin{pmatrix} 1 \\ 0 \end{pmatrix}
    \otimes
    \begin{pmatrix} 0 \\ 1 \end{pmatrix}
    =
    \begin{pmatrix} 0 \\ 1 \\ 0 \\ 0 \end{pmatrix}
    \neq
    \begin{pmatrix} 0 \\ 0 \\ 1 \\ 0 \end{pmatrix}
    =
    \begin{pmatrix} 0 \\ 1 \end{pmatrix}
    \otimes
    \begin{pmatrix} 1 \\ 0 \end{pmatrix}
    $$
\end{definition}

\textbf{Qubits}\\
A classical bit is one of the numbers 0 or 1.
\begin{definition}
    \textit{Quantum bit (qubit)} is a unit vector in $\C^2$
    for which a particular orthonormal basis is fixed.
    Basis is denoted by $|0\rangle, |1\rangle$, it might be, e.g.,
    $$ |0\rangle = \begin{pmatrix} 1 \\ 0 \end{pmatrix}
        \quad
       |1\rangle = \begin{pmatrix} 0 \\ 1 \end{pmatrix}$$
    A basis is fixed throughout the theory.
    Basis vectors are identified with classical bits, 0 and 1, respectively.\\

    Qubit, therefore, is $a|0\rangle + b|1\rangle$,
    it might be
    $$a\begin{pmatrix} 1 \\ 0 \end{pmatrix} + b\begin{pmatrix} 0 \\ 1 \end{pmatrix}$$
    for some $a,b \in \C$ such that $|a|^2 + |b|^2 = 1$.\\

    When a qubit $a|0\rangle + b|1\rangle$ is \textit{measured (observed)}
    with respect to basis $|0\rangle, |1\rangle$,
    the qubit collapses either
    to $|0\rangle$ (with probability $|a|^2$) or
    to $|1\rangle$ (with probability $|b|^2$).
\end{definition}

\textbf{Canonical Representation}\\
Consider a qubit $a|0\rangle + b|1\rangle$ where $a,b \in \C$ such that $|a|^2 + |b|^2 = 1$,
each of complex numbers $a$ and $b$ is defined by two real numbers:
$$a = a_1 + ia_2 \quad b = b_1 +ib_2$$
$$a_1, a_2, b_1, b_2 \in \R$$
Represent $a$ and $b$ in polar form:
$$a = re^{i\varphi} \quad b = se^{i\psi}$$
Thus our qubit is
$$re^{i\varphi}|0\rangle + se^{i\psi}|1\rangle$$
Since we get an equivalent qubit by multiplying given qubit by a constant,
we can get
$$r|0\rangle + se^{i(\psi - \varphi)}|1\rangle$$

Using the fact that $a|^2 + |b|^2 = 1$,
we can rewrite our quibit as $r^2|e^{i\varphi}|^2 + s^2|e^{i\psi}|^2$.
Finally we obtain $r^2 + s^2 = 1$ (since $|e^{i\varphi}|^2 = |e^{i\psi}|^2 = 1$).
There exist $\alpha$, $0 \leq \alpha \leq \pi$ such that we can
express $r$ and $s$ in terms of $alpha$.

\begin{align*}
    1 &= |a|^2 + |b|^2 \\
      &= r^2|e^{i\varphi}|^2 + s^2|e^{i\psi}|^2 \\
      &= |e^{i\varphi}|^2 = |e^{i\psi}|^2 \\
      &= r^2 + s^2 \\
    r &= \cos(\alpha/2) \\
    s &= \sin(\alpha/2)
\end{align*}

Let $\beta = \psi − \varphi$. It follows that the qubit,
where $0 \leq \beta \leq 2\pi$,
can be written as
$$\cos(\alpha/2)|0\rangle + \sin(\alpha/2)e^{i\beta} |1\rangle$$

This is called \textbf{canonical representation} of qubit.
It follows that just two real numbers, $\alpha$ and $\beta$,
completely represent the qubit.
Canonical representation is unique except the case where the qubit is one of $|0\rangle, |1\rangle$.\\

\textbf{Bloch sphere}\\
We can re-interpret $\alpha$, $\beta$ as polar coordinates on a sphere,
and hence our qubit can be interpreted as a point on a sphere:
$\alpha$ is co-latitude with respect to $z$-axis,
while $\beta$ is longitude with respect to $x$-axis.
This interpretation is known as the \textbf{Bloch sphere}.
Each point on Bloch sphere (i.e., each qubit) is expressed via $\alpha$, $\beta$ as follows:
\begin{align*}
    x &= \sin\alpha \cos\beta \\
    y &= \sin\alpha \sin\beta \\
    z &= \cos\alpha
\end{align*}
The following calculation shows that two antipodal points on Bloch sphere correspond to orthogonal qubits.
Consider a qubit in canonical representation:
$$|x\rangle = \cos{\frac{\alpha}{2}}|0\rangle + (\sin{\frac{\alpha}{2}})e^{i\beta}|1\rangle$$
Its antipodal qubit on the Bloch sphere is
\begin{align*}
    |y\rangle &= \cos\left(\frac{\pi - \alpha}{2}\right) |0\rangle +
                 \sin\left(\frac{\pi - \alpha}{2}\right) e^{\beta + \pi} |1\rangle \\
              &= \cos\left(\frac{\pi - \alpha}{2}\right) |0\rangle -
                 \sin\left(\frac{\pi - \alpha}{2}\right) e^{\beta} |1\rangle \\
    \langle y|x \rangle &= \cos\left(\frac{\alpha}{2}\right) \cos\left(\frac{\pi - \alpha}{2}\right) -
                           \sin\left(\frac{\alpha}{2}\right) \sin\left(\frac{\pi - \alpha}{2}\right)
\end{align*}
Here we use that, by definition,
$\langle y|x \rangle$ means $\bar{y}^T x$,
hence $e^{−i\beta}$ will be multiplied by $e^{i\beta}$ which gives 1.
But $\cos(a + b) = \cos a \cos b − \sin a \sin b$,
so $\langle y|x \rangle = \cos(\pi/2) = 0$.
Hence antipodal points correspond to orthogonal qubits.\\

\textbf{Multiple Qubits}\\



\pagebreak
\textbf{Functions from bits to bits}\\
\textbf{Deutsch’s Algorithm}\\
