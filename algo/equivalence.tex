\section{Polynomial Transformations and Equivalence}
\textbf{Polynomial Transformations}
\begin{definition}
    Consider two languages $\mathcal{L_1} \subset A_1$ and $\mathcal{L_2} \subset A_2$,
    $\mathcal{L_1}$ is said to be \textit{polynomially transformable} to $\mathcal{L_2}$ if
    $$\exists\ f : A_1 \rightarrow A_2$$
    such that $f$ can be computed with polynomially complexity and $f$ preserves the language. I.e.
    $$\forall\ x \in A_1\ x \in \mathcal{L_1}\ \mathrm{iff}\ f(x)\in \mathcal{L_2}$$
\end{definition}

Practically $\mathcal{L_1} \propto \mathcal{L_2}$ means that
$\mathcal{L_2}$ is not harder than $\mathcal{L_1}$;
i.e. if $\mathcal{L_2} \in P$ then $\mathcal{L_1} \in P$.
Furthermore this relation is transitive.
$$
  \mathcal{L_1} \propto \mathcal{L_2},\ 
  \mathcal{L_2} \propto \mathcal{L_3},\Rightarrow
  \mathcal{L_1} \propto \mathcal{L_3}
$$

\textbf{Polynomial Equivalence}
\begin{definition}
    If we have two languages such $\mathcal{L_1} \propto \mathcal{L_2}$ and
    $\mathcal{L_2} \propto \mathcal{L_1}$, then they are said to be \textit{polynomially equivalent}.
    Thus we have $\mathcal{L_1} \approx \mathcal{L_2}$.
\end{definition}
This is an equivalence as such it is reflexive, symmetric, and transitive;
as such $\forall\ R,\ S,\ T$:
\begin{itemize}
    \item $R \approx R$
    \item if $R \approx S$ then $S \approx R$.
    \item $
        R \approx S,\ 
        S \approx T \Rightarrow
        R \approx T
        $
\end{itemize}

%%TODO
In practise what this means is that we can see what problem a class is in if we can
polynomially translate it to another problem in a known class; for example:
$$
  \forall\ \mathcal{L_1} \in P\ 
  \forall\ \mathcal{L_2} \in P\ 
  \mathcal{L_1} \approx \mathcal{L_2}
$$
