\section{Polynomial Transformations and Equivalence}
\textbf{Polynomial Transformations}
\begin{definition}
    Consider two languages $L_1 \subset A_1$ and $L_2 \subset A_2$,
    $L_1$ is said to be \textit{polynomially transformable} to $L_2$ if
    $$\exists\ f : A_1 \rightarrow A_2$$
    such that $f$ can be computed with polynomially complexity and $f$ preserves the language. I.e.
    $$\forall\ x \in A_1\ x \in L_1\ \mathrm{iff}\ f(x)\in L_2$$
\end{definition}

Practically $L_1 \propto L_2$ means that
$L_2$ is not harder than $L_1$;
i.e. if $L_2 \in P$ then $L_1 \in P$.
Furthermore this relation is transitive.
$$
  L_1 \propto L_2,\ 
  L_2 \propto L_3,\Rightarrow
  L_1 \propto L_3
$$

\textbf{Polynomial Equivalence}
\begin{definition}
    If we have two languages such $L_1 \propto L_2$ and
    $L_2 \propto L_1$, then they are said to be \textit{polynomially equivalent}.
    Thus we have $L_1 \approx L_2$.
\end{definition}
This is an equivalence as such it is reflexive, symmetric, and transitive;
as such $\forall\ R,\ S,\ T$:
\begin{itemize}
    \item $R \approx R$
    \item if $R \approx S$ then $S \approx R$.
    \item $
        R \approx S,\ 
        S \approx T \Rightarrow
        R \approx T
        $
\end{itemize}

%%TODO
In practise what this means is that we can see what problem a class is in if we can
polynomially translate it to another problem in a known class; for example:
$$
  \forall\ L_1 \in P\ 
  \forall\ L_2 \in P\ 
  L_1 \approx L_2
$$
