\section{Cook’s Theorem}

\subsection{Formulation}
\begin{theorem}
    Boolean Satisfiability problem is NP-complete.
\end{theorem}
Actually we will prove that CNF-satisfiability problem is NP-complete.

\subsection{Proof}
\subsection{Formulation}

The following is in conjunctive normal form:
$$(x_1 \lor x_2) \land \neg x_3$$
The following is not in conjunctive normal form;
however it is in \textit{disjunctive normal form}
$$(x_1 \lor x_2) \lor \neg x_3$$
This is because there is the negation on the left hand side of the logical or.

\begin{definition}
    Given an undirected graph $\mathcal{G}$ which is a pair of vertices and edges $(V,E)$,
    a subset of vertices $S \subset V$ is a \textit{clique} iff
    $$\forall\ x, y \in S\ \exists\ \textrm{an edge}\ (x,y) \in E$$
\end{definition}

\begin{definition}
    If $S$ is a clique in $\mathcal{G}$ and $|S| = q$ then $S$ is called a \textit{$q$-clique}.
\end{definition}

An obvious consequence of this definition is that for any edge $(x,y)$ there is a 2-clique.

\begin{definition}
    The \textit{$q$-clique problem} is finding if there exists a $q$-clique in a given graph
    $\mathcal{G}$ where $q \in \mathbb{N}$.
\end{definition}
