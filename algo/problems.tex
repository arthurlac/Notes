\textit{Traveling Sales Person}\\
\begin{itemize}
    \item \textbf{Inputs}:
    \item \textbf{Output}:
\end{itemize}
Consider a finite set $N$ of cities; $\{1, 2 \dots n\}$. The distance, or cost,
of traveling between cities $i$ and $j$ is given by $D(i,j)$ where $D$
is a symmetric matrix.

$$\forall\ i, j\quad D(i,J) \in \mathbb{N}\quad D(i,j) = D(j,i)$$

The problem is to find a path, or permutation, of
cities where each city appears exactly once in the path and the total
cost is minimised.

The brute force solution is to consider all paths, doing this would require
considering $!n$ paths.

\textit{Traveling Sales Person (Yes-No Version)}\\
\begin{itemize}
    \item \textbf{Inputs}:
    \item \textbf{Output}:
\end{itemize}
Consider a finite set $N$ of cities; $\{1, 2 \dots n\}$. The distance, or cost,
of traveling between cities $i$ and $j$ is given by $D(i,j)$ where $D$
is a symmetric matrix.

$$\forall\ i, j\quad D(i,J) \in \mathbb{N}\quad D(i,j) = D(j,i)$$

The problem is to find a path, or permutation, of
cities where each city appears exactly once in the path and the total
cost is minimised.

\textit{Boolean Formula Satisfiability}\\
\begin{itemize}
    \item \textbf{Inputs}:
    \item \textbf{Output}:
\end{itemize}
A boolean formula consists of the logical binary operators $(\land, \lor, \neg)$
and a finite set of variables $X$ where each variable can be either true or false.
The problem consists of deciding if there exists a set of values for the variables $X$
which results in the boolean formula being true.

$$(x_1 \land x_2) \land \neg x_3$$

$$(x_1 \land x_2) \land \neg x_1$$

A brute force solution would be to consider all $2^{|X|}$ possible assignments.

The link between these two problems is the na\"ive brute force solution is not
tractable for large values of $n$.

\textit{Knapsack Problem}\\
\begin{itemize}
    \item \textbf{Input}: Real numbers $x_1,\dots x_n$
    \item \textbf{Output}: Yes iff there exists a subset of $x_i$ which sum to 1.
\end{itemize}

The following is in conjunctive normal form:
$$(x_1 \lor x_2) \land \neg x_3$$
The following is not in conjunctive normal form;
however it is in \textit{disjunctive normal form}
$$(x_1 \lor x_2) \lor \neg x_3$$
This is because there is the negation on the left hand side of the logical or.

\begin{definition}
    Given an undirected graph $\mathcal{G}$ which is a pair of vertices and edges $(V,E)$,
    a subset of vertices $S \subset V$ is a \textit{clique} iff
    $$\forall\ x, y \in S\ \exists\ \textrm{an edge}\ (x,y) \in E$$
\end{definition}

\begin{definition}
    If $S$ is a clique in $\mathcal{G}$ and $|S| = q$ then $S$ is called a \textit{$q$-clique}.
\end{definition}

An obvious consequence of this definition is that for any edge $(x,y)$ there is a 2-clique.

\begin{definition}
    The \textit{$q$-clique problem} is finding if there exists a $q$-clique in a given graph
    $\mathcal{G}$ where $q \in \mathbb{N}$.
\end{definition}

\textbf{Distinctness problem: upper bound}\\
\begin{itemize}
    \item \textbf{Input}: $(x_0, x_1,\dots x_n) \in \mathbb{R}^n$
    \item \textbf{Output}: Yes iff $\forall\ i,j$ where $0 \leq i,j \leq n, i \neq j$ we have $x_i \neq x_j$
\end{itemize}
