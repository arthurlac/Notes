\section{Turing Machines}
\subsection{Deterministic Turing Machines}
%TODO
What do Turing machines do? Why are we looking at them?
Conceptually a Turing machine is a \textit{tape} which is infinite in both directions,
the input is writen on this tape with the rest of the tape being blank. We have a head
which can read and write to the tape. The machine moves across the tape acording to
a transition function until it reaches a terminal state which produces a yes or no result.

\begin{center}
    \begin{tikzpicture}[
            start chain=1 going right,start chain=2 going below,node distance=-0.15mm
        ]
        \node [on chain=1] at (-1.5,-.4) {\ldots};
        \foreach \x in {1,2,...,11} {\x, \node [draw,on chain=1] {}; }
        \node [
            name=k,
            arrow box,
            on chain=2,
            arrow box arrows={north:.25cm},
            draw=black,thick
            ] at (-0.335,-1) {Head};
        \node [name=r,on chain=1] {\ldots};
    \end{tikzpicture}
\end{center}

\begin{definition}
    Formally a Turing machine consists of
    \begin{itemize}
        \item A \textit{tape alphabet} $\Gamma$ $\{a_1, a_2 \dots a_n, \beta\}$,
            where $a_i$ is a symbol, and $\beta$ is the blank or empty symbol.
        \item An \textit{input alphabet} $\Sigma$
            where $\Sigma \subset \Gamma \setminus \{\beta\}$.
        \item A set of states $Q$ $\{q_0, q_1 \dots q_m, q_y, q_n\}$ where $q_0$ is the
            initial state, $q_y$ is the terminal yes state, and $q_n$ is the terminal no state.
            $Q\prime = Q \setminus \{q_y,q_n\}$, i.e. the non-terminal states.
        \item A transition function $\delta$,
            where $\delta\ :\ (Q\prime \times \Gamma) \rightarrow
            (Q \times \Sigma \times \{L,S,R\})$.
            What this means is we look at a pair of a non-terminal state
            and a symbol from the alphabet;
            given this we enter a new state $q \in Q$,
            we write to the tape a symbol $a \in \Gamma$,
            and we stay or move the head left or right $(\{L,S,R\})$.
    \end{itemize}
\end{definition}

\begin{definition}
    Given an input $n$ and a Turing machine,
    $T(n)$ is the amount of times the transition function 
    $\delta$ must be applied to reach a terminal state.
\end{definition}

\textbf{Even or odd}
\begin{example}
    In this example we will present a Turing machine to decide if a number is even.
    \begin{itemize}
        \item Input: a number $x \in \mathbb{Z}$ in unary form; i.e. $\{1 = X, 2 = XX\dots\}$,
            thus $\Sigma = \{X\}$.
        \item Output: yes iff $x$ is even, otherwise no. I.e. the Turing machine will finish
            in state $q_y$ iff $x$ is even.
        \item Transition function $\delta$ is described as a matrix below with the states
            $Q\prime = \{q_0,q_1\}$ and $\Sigma = \{X,\beta\}$,
            output is $(Q \times \Sigma \times \{L,S,R\})$.
            \begin{center}
                \begin{tabular}{ c c c }
                    & $q_0$           & $q_1$           \\
                    X    & ($q_1,\beta,R$) & ($q_0,\beta,R$) \\
                    $\beta$ & ($q_y,\beta,S$) & ($q_n,\beta,S$) \\
                \end{tabular}
            \end{center}
    \end{itemize}
    For this Turing machine $T(n) = n + 1$.
\end{example}

\textbf{Recognising Palindromes}\\
A palindrome is any word such that the reversed word is equal to the word itself,
consider the alphabet $\Gamma = \{a,b\}$, examples of palindromes would be $a, aa, aba$ etc.
Can we construct a Turing machine which recognises palindromes for the alphabet $\Gamma$?

We can, we simply move forward and backward checking each letter at opposite ends is the same,
until the entire word has been checked.
This means for palindromes $T(n)$ is equal to $n + (n - 1) \dots + 1$. Thus
$$T(n) = \frac{n^2 + n}{2}$$

However in non palindrome cases, say $aaa \dots ab$, then $T(n)$ is very small.

\subsection{Non-Deterministic Turing Machines}
\begin{definition}
    \textit{Nondeterministic Turing Machine (NTM)} has
    the same constituent elements as a deterministic Turing Machine,
    the difference between TM and NTM is in the way they work.
    NTM has a “guessing module” with write-only head attached to it.
    Before NTM starts working, an input word is written on a tape;
    the working head observes cell 1; guessing head observes cell −1.
    NTM works in two stages:
    \begin{enumerate}
        \item \textbf{Guessing stage}
            Guessing head moves along the tape from right to left
            shifting to the next cell on each step.
            On each step guessing head writes a letter from $\Gamma$ on the tape.
            The process may or may not terminate.
            During this stage the control module of NTM and its working head remain passive;
            NTM is not in any “state”. When (if ever) the guessing state terminates, NTM moves to the second stage.
        \item \textbf{Verifying stage}
            NTM adopts the initial state $q_0$ and works as ordinary TM
            with the combination of the “guessed word” and the original input word, as its input.
            During this second stage the guessing module and its guessing head are passive.
    \end{enumerate}
    \textbf{Observe}
    \begin{itemize}
        \item For a fixed input word $x$, the sequence of steps from stages (1) and (2) is called computation for $x$.
        \item NTM accepts an input $x$ if there exists a computation (for $x$) ending with the state $q_Y$.
        \item If NTM accepts $x$ there might also be computations (for $x$) ending with $q_N$.
        \item There might be many computations (for $x$) leading to $q_Y$.
    \end{itemize}
\end{definition}
\begin{definition}
    Let NTM accept $x$.
    Complexity of accepting $x$ is the number of steps in the shortest accepting computation for x.
\end{definition}
\begin{definition}
    Complexity of NTM is the function
    \begin{equation}
        T(n) = max\ \left\{
            \begin{array}{@{}ll@{}}
                m \textrm{ such that there exists }x\textrm{ where} |x| = n\\
                \textrm{ such that the complexity of accepting }x\textrm{ is }m
            \end{array}\right\}.
    \end{equation}
    If such $x$ does not exist let $T(n) = 1$.
\end{definition}

\subsection{$O, \Omega, \Theta$ Notation}
\begin{definition}
    Given a computation $T(x)$ where $|x| = n$,
    constants $c > 0$ and $n_0 > 0$,
    and functions $f_1, f_2, f_3$ defined on $n$,
    we have
    \begin{itemize}
        \item $O(f_1(n))$ where $\forall\ |x| \geq n_0$ we have $f_1(n) \leq c\,T(x)$
        \item $O(f_1(n))$ describes the upper bound for $T(x)$.
        \item $\Omega(f_2(n))$ where $\forall\ |x| \geq n_0$ we have $f_2(n) \geq c\,T(x)$
        \item $\Omega(f_2(n))$ describes the lower bound for $T(x)$.
        \item $\Theta(f_3(n))$ where we have both $O(f_3(n))$ and $\Omega(f_3(n))$,
            or relying on the fact that $O(g(n))$ and $\Omega(g(n))$ are sets of functions, we can define
            $$\Theta(g(n)) \equiv O(g(n)) \cap \Omega(g(n))$$
        \item $\Theta(f_3(n))$ describes the exact bound of for $T(x)$.

    \end{itemize}
\end{definition}
