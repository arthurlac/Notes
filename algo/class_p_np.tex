\section{Complexity Classes}
\subsection{Class P}
\begin{definition}
    \textit{Class P} is the class of all Yes - No computational problems
    (languages) $L$ such that there exists $TM$ and a polynomial $p(n)$ such that
    $TM$ solves $L$ with complexity $T(n) \leq p(n)$ for all $n \geq 1$.
\end{definition}
\textbf{Examples}
\begin{itemize}
    \item \textit{Even unary non-negative numbers} is in P , take $n + 1$ as polynomial $p(n)$ from the definition
    \item \textit{Palindromes} is in P, $p(n) = cn^2$ for a large enough positive constant $c$
    \item \textit{Primality} is in P . This is a problem of deciding for a given integer number $n > 0$,
        represented in decimal (or binary) form, whether $n$ is prime
\end{itemize}

\subsection{Class NP}
\begin{definition}
    \textit{Class NP} consists of languages $L$ such that
    there exist polynomial $p(n)$ and NTM T for solving $L$
    with complexity $T(n) \leq p(n)$.\\

    \textbf{Examples}
    \begin{itemize}
        \item \textit{Travelling Salesman} $\in$ NP.
            We need to construct a NTM for solving this problem with polynomial complexity.
            On the guessing stage the NTM produces an arbitrary permutation of cities,
            and the verifying stage, working like an “ordinary” TM,
            checks whether the length of the guessed tour does no exceed the boundary $B$.
            An accepting computation for an input $x$ exists if and only if there is a short tour.
            This computation consists in writing out a guess and finding the corresponding sum of all distances.
            Clearly the complexity of this computation is polynomial.
        \item \textit{Boolean satisfiability} $\in$ NP.
            NTM first guesses a satisfying assignment of truth values to variables,
            and then verifies the guess in polynomial time.
    \end{itemize}
\end{definition}

\begin{theorem}
    $P \subset NP$
\end{theorem}
\begin{proof}
    Let $L$ be a Yes-No problem from P,
    and $D$ be a (deterministic) TM for solving $L$ in polynomial time.
    We get a NTM for solving $L$ in polynomial time by imitating $D$ on the verification stage
    and ignoring the guessing stage.
\end{proof}

\begin{center}
The question whether P = NP, is a famous open problem.
\end{center}

\begin{theorem}
    Let $L \in NP$,
    then there exist a polynomial $p(n)$ and TM for solving $L$ with complexity $O(2^{p(n)})$.
\end{theorem}
\begin{proof}
    Let $N$ be an NTM for solving $L$ with complexity $T_N(n) \leq r(n)$
    where $r(n)$ is a polynomial.
    For every accepted input $x$ (with $|x| = n$)
    there is a guess word of length $\leq r(n)$
    such that $N$ adopts the state $q_Y$ in no more than $r(n)$ steps.
    The total number of possible guesses is less than
    $$|\Gamma|^{r(n)+1}$$

    Now we construct a (deterministic) TM $M$ as follows.
    $M$ examines every possible guess in turn,
    and for each of them runs the verification stage of $N$ up to $r(n)$ steps.
    That will take $$O(r(n)|\Gamma|^{r(n)})$$ steps.
    By the definition of O-symbol, that’s the same as $O(2p(n))$ for a certain polynomial $p(n)$.
\end{proof}

\subsection{Class NP-Complete}
\begin{definition}
    $L$ is \textit{NP-Complete} if $L \in NP$ and
    $\forall\ L\prime \in NP\ L\prime \propto L$.
\end{definition}

\textbf{What does this definition mean?}\\
Firstly, any NP-Complete problem is at least as "hard" as any other problem in NP.
Secondly, any two NP-complete languages are polynomially equivalent.

\begin{theorem}
    Given two languages $L_1 \in NP$ and $L_2 \in NPC$,
    if $L_2 \propto L_1$ then $L_1 \in NPC$.
\end{theorem}

This theorem is rather useful for proving that a language is NP-Complete,
consider that all we have to prove of a candidate NPC language $L$ is that, firstly,
it is in NP and, secondly, that it is polynomially transformable to a problem known to be in NPC.
The catch, however, is proving a first language is NP-Complete.

\begin{definition}
    A boolean formula $F$ is in \textit{conjunctive normal form} if it is in the form
    \begin{equation}
        \begin{split}
            F &= f_1 \land f_2 \dots \land f_n \\
            f_i &= y_1 \lor y_2 \dots \lor y_m \\
            y_j &= x\ |\ \neg x
        \end{split}
    \end{equation}
\end{definition}

\subsection{Class co-P and co-NP}
\textbf{Complements to languages}
\begin{definition}
    Let $L \subset \Sigma\ast$ be a language over an alphabet $\Sigma$.
    The set of words $\bar{L} = \Sigma\ast \setminus L$ is called the complement to $L$.
\end{definition}

Observe that $\bar{L}$ is not necessarily a language,
i.e. a set of words generated by a grammar (or accepted by a TM).
However, for all languages $\in$ NP, the complement $\bar{L}$is indeed a language.
It is not at all obvious that if $L \in NP$, then $\bar{L} \in NP$.

\begin{example}
    Consider the complement to CNF-SAT:
    \begin{itemize}
        \item \textbf{Input}: Boolean formula $F$ in conjunctive normal form.
        \item \textbf{Output}: Yes iff $F$ is not satisfiable (i.e. false).
    \end{itemize}
    There is no apparent way of using non-determinism to solve this problem.
    Brute force is our approach here, just examining, one by one, every true/false assignment.
\end{example}

\begin{definition}
    Let \textit{co−NP} be the set of languages $L$ such that $\bar{L} \in NP$.
\end{definition}
Similarly we also have co-P.
\begin{definition}
    Let \textit{co−P} be the set of languages $L$ such that $\bar{L} \in P$.
\end{definition}

\begin{theorem}
    P $=$ co-P, observe that if we have a TM for a language $L$
    then simply negating our answer provides us with a TM for $\bar{L}$.
\end{theorem}

\begin{hypothesis}
    NP $\neq$ co−NP\\
    This hypothesis is \textit{stronger} than P $\neq$ NP
    as it implies if P = NP, that NP = co−NP.
\end{hypothesis}


\subsection{Class NP-Intermediate}
Recall that it is not known whether or not $P = NP$,
the widely accepted hypothesis being that this equality is wrong.
Let NP-Intermediate (NPI) be defined as $NP \setminus (NPC \cup P)$,
i.e. the class of problems in NP but not in NPC.
\begin{theorem}
    If P $\neq$ NP then:
    \begin{itemize}
        \item NPI $\neq \varnothing$
        \item There exist the problems $A,B \in$ NPI such that
            neither $A \propto B$, nor $B \propto A$
            (proof of this is nontrivial).
            This states that under the hypothesis P $\neq$ NP
            the set NPI is divided into more that one equivalence classes
            with respect to polynomial transformation.
    \end{itemize}
\end{theorem}

\textbf{Some famous candidates for problems in NPI}\\

\textit{Graph Isomorphism}
\begin{itemize}
    \item \textbf{Input}:
        Two graphs $\mathcal{G} = (V, E)$
        and $\mathcal{G}\prime = (V\prime, E\prime)$.
    \item \textbf{Output}:
        Yes iff there is a bijective (one-to-one) map $f: V \rightarrow V\prime$
        such that $(v,w) \in E$ is equivalent to $(f(v),f(w)) \in E\prime$.
\end{itemize}
The status of Graph Isomorphism is unknown.
Despite many efforts no polynomial time algorithm was found.
On the other hand, the problem seems to be too “rigid” to be NP-complete.
Compare it with the following NP-complete problem.\\

\textit{Isomorphism to Subgraph}:
\begin{itemize}
    \item \textbf{Input}:
        Two graphs $\mathcal{G} = (V, E)$
        and $\mathcal{G}\prime = (V\prime, E\prime)$.
    \item \textbf{Output}:
        Yes iff there is a subgraph in $\mathcal{G}$ which is isomorphic to $\mathcal{G}\prime$.
\end{itemize}

Isomorphism to Subgraph is an NP-complete problem
which is proved by polynomially transforming to it to the k-clique problem
(take as $\mathcal{G}\prime$ the complete graph with $k$ vertices).

\textit{Linear Programming}
\begin{itemize}
    \item \textbf{Input}: System of linear inequalities in several variables with integer coefficients.
    \item \textbf{Output}: Yes iff the system has a solution (in real numbers).
\end{itemize}

The status of Linear Programming was open for a long time
until in late seventies a polynomial-time algorithm was discovered.
Even before that it seemed quite unlikely for the problem to be NP-complete
since the complement to Linear Programming is in NP
(follows from so-called Duality Theorem in linear programming).
If Linear Programming was NP-complete then NP = co−NP, which is probably wrong.

\subsection{Class NP-Hard}
\textbf{Search Problems}\\
The computational problems we considered so far were Yes - No problems,
however there is a large set of useful problems which are not of the form Yes-No.
Consider the traditional Travelling Salesman problem is an optimisation problem.
Observe that what we shall call search problems are “harder” than the corresponding Yes - No problems
in the sense that any algorithm for a search problem automatically solves the Yes - No version.\\

The theory we had developed so far is not very well adjusted for search problems.
Firstly, the concept of nondeterministic Turing Machine is formulated for Yes-No problems.
Secondly, and most importantly, the definition of $\propto$ relation is very specific,
formulated essentially in terms of languages.\\

\textbf{Turing Transformation Informally}\\
Recall that for two languages $L_1 , L_2$,
the relation $L_1 \propto L_2$ informally means that there is an algorithm $f$ such that:
\begin{enumerate}
    \item
        For given input word $x_1$ of $L_1$ the algorithm computes $x_2$ of $L_2$,
        herewith $x_1 \in L_1$ if and only if $x_2 \in L_2$
    \item the algorithm uses a “subroutine” to solve the problem 2 on the input $x_2$ and finds output $y_2$;
    \item $y_2$ is also the answer for problem 1.
\end{enumerate}
The algorithm uses polynomial time for all its work,
except the “subroutine”.
Now we generalize this informal definition for search problems $L_1$ and $L_2$:
We say that $L_1$ is Turing transformable to $L_2$
if there exists an algorithm which for a given input $x_1$ of $L_1$
computes the required output using (zero, or one, or more times) an algorithm
for the problem $L_2$ as a subroutine.
The running time of the algorithm is polynomial
if each call of the subroutine is counted as one elementary step.\\

\textbf{Oracle Turing Machine}\\
Oracle Turing Machine (OTM) differs from ordinary (deterministic) TM
with arbitrary output in the following way.
Apart from the usual working tape,
OTM has an additional oracle tape equipped with the oracle read-write head
which is attached to the control module.
OTM works like ordinary TM,
but can at an arbitrary moment adopt (via “asking” state) a special asking mode in which:
\begin{enumerate}
    \item the oracle head writes a word (which was generated during the previous normal work of the machine) on the oracle tape (oracle input);
    \item after (1) is done,
        the oracle head writes a word on the oracle tape (oracle output).
        Passage from (1) to (2) is determined by a function
        $$g: \Gamma\ast \rightarrow \Gamma\ast$$
        where $\Gamma$ is the tape alphabet.
\end{enumerate}
Each application of asking mode counts as one step for complexity.
Since OTM depends on a concrete build-in function $g$,
the notation OTMg might be more appropriate.
The formal definition of OTM and its complexity is straightforward.\\

\textbf{Turing transformation: Definition}\\
\begin{definition}
    Let L1,L2 be search problems,
    herewith L2 is a problem of computing
    a function g of the form $(\ast)$.
    We say that L1 is Turing transformable to L2
    (notation: $L1 \propto T L2$)
    if there exists OTMg for solving L1
    using as oracle the function g and having polynomial complexity.
\end{definition}

\begin{theorem}
    For any two Yes - No problems %(languages L1 , L2 ), if L1 ∝ L2 , then L1 ∝T L2.
\begin{theorem}

Trivial, since as we had seen bef%ore, ∝ is a particular case of ∝T .
It is not known whether the statement inverse to Theorem 1 is true or not for languages.

\begin{theorem}
    The %relation ∝T is transitive.
    Proof is straightforward.
\end{theorem}

\textbf{Definition of NP-Hard}\\
\begin{definition}
    A search problem L is %NP-hard if for any language L′ ∈ NP the following holds: L′ ∝T L. (∗∗)
\end{definition}
\begin{theorem}
    A search problem L is NP-hard if and only if for at least one (thus, for any)
    NP-complete language %L′ relation (∗∗) is true.
\end{theorem}
\begin{proof}
    If L is NP-hard, then% for any language from NP, in particular, for any NP- complete language L′ the relation (∗∗) is true.
    Conversely, if there %exists an NP-complete language L′ such that (∗∗) is true, then for any language L′′ ∈ NP we have
    %L′′ ∝ L′ ∝T L,
    %which, by Theorems 1 and 2 of Section 8, implies that L′′ ∝T L, thus L is NP-hard by
    the definition.
\end{proof}
\begin{example}
    Consider a modification of Travelling Salesman problem
    as a search problem in which the output is the tour of the smallest length.
    Call this modification Optimal Travelling Salesman (OptTS),
    as opposed to Yes - No modification (TS) which involves a threshold.
\end{example}

\end{theorem}
    T%S ∝T OptTS, therefore OptTS is N P -hard.
\end{theorem}

\begin{proof}
    The OTM for TS with threshold B% first uses the oracle to solve the OptTS with the same set of cities and distances as the input of TS (one step of computation), then in polynomial time finds the length l of the produced optimal tour and checks whether B ≥ l. If the latter inequality is true, then the answer is Yes, else – No.
Observe that this example does not u%se the relation ∝T in full force, addressing the oracle only once.
\end{proof}
